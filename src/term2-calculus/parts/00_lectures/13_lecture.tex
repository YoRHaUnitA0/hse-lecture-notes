\SectionLecture{Лекция 13}{Игорь Энгель}
\begin{example} \thmslashn

    $f(x) = \begin{cases}
        e^{-\frac{1}{x^2}} & x \neq 0\\
        0 & x = 0
    \end{cases}$ 

    \[ f^{(n)}(x \neq 0) = \frac{p_{n}(x)}{x^{3n}}e^{-\frac{1}{x^2}} .\] 

    (можно по индукции)

    \[ f^{(n)}(0) = \lim\limits_{x \to 0} \frac{f^{(n-1)}(x) - f^{(n-1)}(0)}{x} = 0.\]

    (тоже по индукции и предыдущему свойству)

    Её ряд тейлора нулевой, но сама функция не нулевая. Значит, бесконечная дифференцируемость не является достаточным условием аналитичности.
\end{example}
\begin{theorem}[Разложение элементарных функций в ряд Тейлора] \thmslashn

    \begin{enumerate}
        \item $e^{x} = \sum\limits_{n=0}^{\infty} \frac{x^{n}}{n!}$
        \item $\cos(x) = \sum\limits_{n=0}^{\infty} (-1)^{n} \frac{x^{2n}}{(2n)!}$
        \item $\sin(x) = \sum\limits_{n=0}^{\infty} (-1)^{n} \frac{x^{2n+1}}{(2n+1)!}$
        \item $\log(1+x) = \sum\limits_{n=1}^{\infty} (-1)^{n-1} \frac{x^{n}}{n} \impliedby x\in (-1, 1)$.
        \item $\arctan(x) = \sum\limits_{n=0}^{\infty} (-1)^{n} \frac{x^{2n+1}}{2n+1} \impliedby x\in [-1, 1]$
        \item \[ (1+x)^{p} = 1 + \sum\limits_{n=0}^{\infty} \frac{p(p-1) \ldots (p-n+1)}{n!} x^{n} \impliedby x\in (-1, 1) .\] 
    \end{enumerate}

    Ряды $1$-$3$ сходятся во всей плоскости.

    Их можно использовать, чтобы доопределить эти функции на всём $\mathbb{C}$.
    \begin{proof} \thmslashn
    
        \[ \log'(1+x) = \frac{1}{1+x} = \sum\limits_{n=0}^{\infty} (-1)^{n}x^{n} \impliedby x \in (-1, 1) .\]
        \[ \log(1+x) = \int\limits_{0}^{x} \frac{dt}{1+t} = \int\limits_{0}^{x} \sum\limits_{n=0}^{\infty} (-1)t^{n}dt = \sum\limits_{n=0}^{\infty} (-1)^{n} \int\limits_{0}^{x} t^ndt = \sum\limits_{n=0}^{\infty} (-1)^{n} \frac{x^{n+1}}{n+1} = \sum\limits_{n=1}^{\infty} (-1)^{n-1} \frac{x^{n}}{n}     .\]

        \[ \arctan'(x) = \frac{1}{1+x^2} = \sum\limits_{n=0}^{\infty} (-1)x^{2n} \impliedby x\in (-1, 1) .\]
        \[ \arctan(x) = \int\limits_{0}^{x} \frac{1}{1+t^2}dt = \sum\limits_{n=0}^{\infty} (-1)^{n} \int\limits_{0}^{x} t^{2n}dt = \sum\limits_{n=0}^{\infty} (-1)^{n} \frac{x^{2n+1}}{2n+1}   .\]

        Докажем что разложение арктангенса подходит для $x=1$:

        По признаку Лейбница, ряд сходится.

        По теореме Абеля:
        \[ \sum\limits_{n=0}^{\infty} (-1)^{n} \frac{1}{2n+1} = \lim\limits_{x \to 1-} \sum\limits_{n=0}^{\infty} (-1)^{n} \frac{x^{2n+1}}{2n+1} = \lim\limits_{x \to 1-} \arctan(x) = \arctan(1) .\]

        \[ (1+x)^{p} = T_{n}(x) + \frac{1}{n!}\int\limits_{0}^{x} (x-t)^{n} (p(p-1)\ldots (p-n))(1+t)^{p-n-1}dt =: T_{n}(x) + R_{n}(x)  .\]

        Покажем что $x\in (-1, 1) \implies \lim\limits_{n \to \infty} R_{n}(x) = 0$.

        \[ \left| \frac{R_{n+1}(x)}{R_{n}(x)}\right| = \frac{ |p-n-1| }{n+1} \left| \frac{\int\limits_{0}^{x} (x-t)^{n+1}(1+t)^{p-n-2}dt }{\int\limits_{0}^{x} (x-t)^{n}(1+t)^{p-n-1} }\right|  .\]
        \[ \frac{ |p-n-1| }{n+1} \left| \frac{\int\limits_{0}^{x} (x-t)^{n}(1+t)^{p-n-1} \frac{x-t}{1+t}dt }{\int\limits_{0}^{x} (x-t)^{n}(1+t)^{p-n-1}dt }\right| .\]

        Заметим, что оба интеграла знакопостоянны. Также, заметим, что $\left|\frac{x-t}{1+t}\right| \le |x|$.

    \[ \left| \frac{R_{n+1}(x)}{R_{n}(x)}\right| \le \frac{ |p-n-1| |x| }{n+1} .\]
    \[ \forall{\eps > 0}\quad \exists{N}\quad \forall{n > N}\quad \left| \frac{R_{n+1}(x)}{R_{n}(x)}\right| \le (1+\eps)|x| .\]

    Так-как $|x| < 1$ фиксированно, всегда можем подобрать $\eps$ что $(1+\eps)|x| < 1$, значит, $R_{n}(x) \to_{n} 0$. 
    \end{proof}
\end{theorem}
\begin{consequence}[Формула Эйлера] \thmslashn

    \[ e^{iz} = \cos z + i\sin z .\]
    \begin{proof} \thmslashn
    
        \[ \sum\limits_{n=0}^{\infty} \frac{i^{n}z^{n}}{n!} = \sum\limits_{n=0}^{\infty} (-1)^{n}\frac{z^{2n}}{(2n)!} + i \sum\limits_{n=0}^{\infty} (-1)^{n} \frac{z^{2n+1}}{(2n+1)!} .\] 
    \end{proof}
\end{consequence}
\begin{consequence} \thmslashn

    \[ \frac{1}{\sqrt{1+x} } = (1+x)^{-\frac{1}{2}} = 1 + \sum\limits_{n=1}^{\infty} \frac{\left( -\frac{1}{2} \right)\left( -\frac{1}{2} - 1 \right)\ldots\left( -\frac{1}{2}-n+1 \right)}{n!}x^{n} = 1 + \sum\limits_{n=0}^{\infty} \frac{(-1)^{n}(2n-1)!!}{(2n)!!}x^{n}  .\]
    \[ \frac{1}{\sqrt{1+x} } = \sum\limits_{n=0}^{\infty} (-1)^{n}\binom{2n}{n} \frac{x^{n}}{4^{n}} .\] 
\end{consequence}
\begin{lemma} \thmslashn

    \[ \arcsin x = \sum\limits_{n=0}^{\infty} (-1)^{n}\binom{2n}{n}\frac{x^{2n+1}}{2n+1} .\] 
    \begin{proof} \thmslashn
    
        Аналогично логорифму и арктангенсу, выводится из $\arcsin' x = \frac{1}{\sqrt{1-x^2} }$.
    \end{proof}
\end{lemma}
\Subsection{8 Функции многих переменных}
\Subsubsection{Дифференцируемые отображения}
\begin{definition} \thmslashn 

    Пусть $f : (E \subset \mathbb{R}^{n}) \mapsto \mathbb{R}^{m}$, $a\in \Int E$, $f$ называется дифференцируемой в точке $a$, если 
    \[ \exists{T\in M_{n \times m}}\quad f(a+h) = f(a) + T(h) + \alpha(h) .\]
    \[ \lim\limits_{n \to \vec{0}} \frac{\alpha(h)}{\|h\|} \to 0  .\]

    $T$ называется дифференциалом $f$ в точке $a$ и обозначается $d_{a}f$.
\end{definition}
\begin{remark} \thmslashn

    Если $T$ существует, то оно определено однозначно.
    \begin{proof} \thmslashn
    
        Зафиксируем $h\in \mathbb{R}^{n}$, такое, что $\|h\| = 1$. 

        \[ f(a+th) = f(a) + T(th) + \alpha(th) = f(a) + tT(h) + \alpha(th) \implies T(h) = \frac{f(a+th)-f(a)}{t} \cdot \frac{\alpha(th)}{t} .\]
        \[ \lim\limits_{t \to 0} \frac{f(a+th)-f(a)}{t} = T(h) .\]

        Это определяет значения $T$ на всех векторах с нормой $1$. Так-как $T$ линейно, этого достаточно:
        \[ T(v) = \|v\|T\left( \frac{v}{\|v\|} \right)  .\] 
    \end{proof}
\end{remark}
\begin{definition} \thmslashn 

    Матрица $T$ называется матрицей Якоби (Якобианом) функции $f$ в точке $a$, и обозначается $f'(a)$.
\end{definition}
\begin{remark} \thmslashn

    Если $f$ дифференцируема в $a$, то $f$ непрерывна в  $a$.

    \begin{proof} \thmslashn
    
        \[ \lim\limits_{h \to 0} f(a+h) = \lim\limits_{h \to 0} f(a) + T(h) + \alpha(h) = f(a) + T(0) + 0 = f(a) .\] 
    \end{proof}
\end{remark}
\begin{remark} \thmslashn

    Если $m=1$, формула принимает следующий вид:
    \[ f(a+h) = f(a) + \left<v, h\right> + \alpha(h)  .\]

    ($\left<a, b\right>$ - скалярное произведение $a$ на $b$).

    Вектор $v$ называется градиентом и обозначается $\grad f$ или $\nabla f$.
\end{remark}
\begin{theorem} \thmslashn

    Дифференцируемость функции в точке равносильна дифференцируемости всех её координатных фукнций в этой точке.
    \begin{proof} \thmslashn

        Пусть $f : \mathbb{R}^{n} \mapsto \mathbb{R}^{m}$ $f_{i}(x)$ - $i$-я координата $f(x)$.

        Необходиомость:

        \[ f(a+h) = f(a) + T(h) + \alpha(h) .\]
        \[ f_{i}(a+h) = f_{i}(a) + T_{i}(h) + \alpha_{i}(h) .\]

        $i$-й столбец матрицы $T$ будет градиентом $i$-й координатной функции.
        
        Достаточность:

        Знаем, что 
        \[ f_{i}(a+h) = f_{k}(a) + T_{i}(h) + \alpha_{i}(h) .\]

        Соберём это всё в вектор. Линейность $T$ не пострадает, $\alpha$ будет покоординатно сходиться к нулю, значит будет сходиться по норме.
    \end{proof}
\end{theorem}
\begin{definition}[Производная по направлению] \thmslashn 

    Пусть есть $h\in \mathbb{R}^{n}$, $\|h\| = 1$, $f : (E \subset \mathbb{R}^{n}) \mapsto \mathbb{R}$, $a\in \Int E$.

    Тогда производная $f$ в точке $a$ по направлению $h$:
    \[ \frac{\partial f}{\partial h}(a) := \lim\limits_{t \to 0} \frac{f(a+th)-f(a)}{t} .\]

    (Проивзодная в нуле функции $g : \mathbb{R} \mapsto \mathbb{R}$, $g(t) = f(a+th)$)
\end{definition}
\begin{theorem} \thmslashn

    Пусть $f : E \mapsto \mathbb{R}$, $a\in \Int E$, $f$ дифференцируема в $a$.

    Тогда $\frac{\partial f}{\partial h}(a) = \left<\nabla f, h\right> = d_{a}f(h)$.

    \begin{proof} \thmslashn
    
        \[ f(a+th) = f(a) + tT(h) + \alpha(th) \implies \frac{\partial f}{\partial h} (a) = \lim\limits_{t \to 0} T(h) + \frac{\alpha(th)}{t} = T(h) .\] 
    \end{proof}
\end{theorem}
\begin{consequence}[Экстремальное свойство градиента] \thmslashn

    Пусть $f : E \mapsto \mathbb{R}$, $a\in \Int E$, $f$ дифференцируема в $a$, $\nabla f(a) \neq 0$.

    Тогда, $\forall{h\in \mathbb{R}^{n}}\quad \|h\| = 1 \implies - \|\nabla f(a)\| \le \frac{\partial f}{\partial h}(a) \le \|\nabla f(a)\|$, причём, равенство достигается только в $h = \pm \frac{\nabla f(a)}{\|\nabla f(a)\|}$.
    \begin{proof} \thmslashn
    
        \[ \left| \frac{\partial f}{\partial h} \right| = \left| \left<\nabla f(a), h\right>\right| \le \|\nabla f(a)\| \|h\| = \|\nabla f(a)\| .\]

        По неравенству Коши-Буняковского. При этом, равенство доистгается если вектора коллинеарны, что вместе с ограничением на норму $h$ даёт только два возможных вектора.
    \end{proof}
\end{consequence}
\begin{definition}[Частная производная] \thmslashn 

    Пусть $f : E \mapsto \mathbb{R}$, $a\in \Int E$.

    Пусть $e_{i}$ - стандартный базис $\mathbb{R}^{n}$.

    Тогда, частная производная по $i$-му аргументу: $\frac{\partial f}{\partial x_{i}}  := f'_{x_{i}} := D_{k}f := \partial_{k}f := \frac{\partial f}{\partial e_{i}} $ 
\end{definition}
\begin{consequence} \thmslashn

    \[ \frac{\partial f}{\partial x_{k}}(a) = \left<\nabla f(a), e_{k}\right> = (\nabla f(a))_{k} \iff \nabla f(a) = \begin{bmatrix} \frac{\partial f}{\partial x_1}(a), \ldots, \frac{\partial f}{\partial x_{n}}(a)   \end{bmatrix}   .\] 
\end{consequence}
\begin{consequence} \thmslashn

    \begin{equation*}
        f' = 
        \begin{bmatrix} 
            \frac{\partial f_1}{\partial x_1} & \ldots & \frac{\partial f_1}{\partial x_{n}}\\
            \ldots & \ldots & \ldots\\
            \frac{\partial f_{m}}{\partial x_{1}} & \ldots & \frac{\partial f_{m}}{\partial x_{n}}  
        \end{bmatrix} 
    \end{equation*}
\end{consequence}
\begin{theorem}[Линейность дифференциала] \thmslashn

    Пусть $f, g : (E \subset \mathbb{R}^{n}) \mapsto \mathbb{R}^{m}$, $f, g$ дифференцируемы в $a\in \Int E$.

    Тогда $f+g$ и $\lambda f$ дифференцируемы в $a$, причём, $d_{a}(f+g) = d_{a}f + d_{a} g$, $d_{a}(\lambda f) = \lambda d_{a}(f)$.
    \begin{proof} \thmslashn
    
        \[ f(a+h) = f(a) + d_{a}f(h) + \alpha(h) .\]
        \[ g(a+h) = g(a) + d_{a}g(h) + \beta(h) .\]
        \[ (f+g)(a+h) = f(a) + g(a) + (d_{a}f + d_{a}g)(h) + (\alpha + \beta)(h) .\]
        
        Аналогично для домножения.
    \end{proof}
\end{theorem}
\begin{theorem} \thmslashn

    Пусть $f : (D \subset  \mathbb{R}^{n}) \mapsto (E \subset \mathbb{R}^{m})$, $g : E \mapsto \mathbb{R}^{\ell}$, $f$ дифференцируема в $a$, $g$ дифференцируема в  $f(a)$, тогда  $g \circ f$ дифференцируема в  $a$, причём, $d_{a}(g \circ f) = d_{f(a)}g \cdot d_{a}f$.
    \begin{proof} \thmslashn
    
        \[ f(a+h) = f(a) + d_{a}f(h) + \alpha(h)\|h\| .\]
        \[ g(b+k) = g(k) + d_{b}g(k) + \beta(k)\|k\| .\] 
        \[ g(f(a+h)) = g(f(a) + d_{a}f(h) + \alpha(h)\|h\|) = g(f(a)) + d_{f(a)}(d_{a}f(h) + \alpha(h)\|h\|) + \beta(k)\|k\| .\]
        \[ g(f(a+h)) = g(f(a)) + d_{f(a)}d_{a}f(h) + \gamma(h).\]
        \[ \gamma(h) = d_{b}(\alpha(h))\|h\| + \beta(k)\|k\|.\]

        Рассмотрим пределы $\gamma(h)$ по слагаемым:

        \[ \lim\limits_{h \to 0} \frac{d_{b}\alpha(h) \|h\|}{\|h\|} \le \|d_{b}\|\|\alpha(h)\| \to 0 .\]

        \[ k = d_{a}f(h) + \alpha(h)\|h\| .\]
        \[ \|k\| \le \|d_{a}f\|\|h\| + \|\alpha(h)\|\|h\| \le C \|h\| \implies \frac{\|\beta(k)\| \|k\|}{\|h\|} \le C \|\beta(k)\| \to 0.\]
    \end{proof}
\end{theorem}
\begin{theorem}[О дифференцировании произведения векторной функции на скалярную] \thmslashn

    Пусть $E \subset \mathbb{R}^{n}$, $a\in \Int E$, $\lambda : E \mapsto \mathbb{R}$, $f : E \mapsto \mathbb{R}^{m}$, $\lambda, f$ дифференцируемы в $a$. Тогда, $\lambda f$ дифференцируема в точке $a$ и  $d_{a}(\lambda f)(h) = d_{a}\lambda(h)f(a) + \lambda(a)d_{a}f(h)$
    \begin{proof} \thmslashn
    
        \begin{equation*}
            \begin{split}
                (\lambda f)(a+h) - (\lambda f)(a) 
                &= (\lambda(a+h) - \lambda(a))f(a+h) + \lambda(a)(f(a+h)-f(a))\\ 
                &= (d_{a}\lambda(h) + o(\|h\|))(f(a) + o(\|h\|)) + \lambda(a)(d_{a}f(h) + o(\|h\|))\\
                &= d_{a}\lambda(h)f(a) + d_{a}\lambda(h)o(\|h\|) + f(a)o(\|h\|) + o(\|h\|^2) + \lambda(a)d_{a}f(h) + \lambda(a)o(\|h\|)\\
                &\to d_{a}\lambda(h)f(a) + \lambda(a)d_{a}f(h) + o(\|h\|)
            \end{split}
        \end{equation*}
    \end{proof}
\end{theorem}
