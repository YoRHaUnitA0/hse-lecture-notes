\SectionLecture{Лекция 5}{Игорь Энгель}
\begin{definition} \thmslashn 

    Система линейных (алгебраичех) уравнений - сисетма условий вида
\begin{equation*}
    \begin{cases}
        a_{11}x_1 + \ldots + a_{1n}x_{n} = b_1\\
        a_{21}x_1 + \ldots + a_{2n}x_{n} = b_2\\
        \vdots\\
        a_{m1}x_1 + \ldots + a_{mn}x_{n} = b_{n}
    \end{cases}
\end{equation*}

Где $a_{ij}, b_{i}\in R$, где $R$ - кольцо.

$x_{i}$ называются переменными.

Решение системы: $\{(x_1, \ldots, x_{n}) \ssep x_{i}\in R \text{ и выполнены все условия системы}\} $.
\end{definition}

Если у нас есть система, можем взять два любых уравнения $E_{i}, E_{j}$ из неё, взять два коэффициента $\lambda,  \mu\in R$, и получить новое уравнение $\lambda E_{i} + \mu E_{j}$. Каждый элемент из решения системы так-же удовлетворяет этому уравнению.

\begin{definition} \thmslashn 

    Две системы называются равносильными, если равны (как множества) их решения.
\end{definition}

\begin{lemma} \thmslashn 

Пусть есть два уравнения из системы $E_{i}, E_{j}$, тогда если заменить $E_{j}$ на $E_{j} + \lambda E_{i}$, $\lambda\in R$, то новая система будет равносильна.
\begin{proof}
    Добавим в систему уравнение $E_{j} + \lambda E_{i}$. Любое решение $E_{i}, E_{j}$ является решением этого уравнения. Удалим уравнение $E_{j}$, могли появиться новые решениея. Но исчезунть не могли.

    Повторим процесс в обратную сторону, заменив $E_{j} + \lambda E_{i}$ на $E_{j} + \lambda E_{i} -\lambda E_{i} = E_{j}$, так-как исчезнуть решения не могли, системы равносильны.
\end{proof}
\end{lemma}
\begin{lemma} \thmslashn

    Заменой строк местами можно получить равносильную систему
\end{lemma}
\begin{lemma} \thmslashn

    Если уравнение $E_{i}$ домножить на $\lambda\in R^{*}$, получиться равносильная система.
\end{lemma}
\begin{definition} \thmslashn 

    Матрицей с коэффициентами в $R$ размерности $m \times n$  называется таблица из $m \times n$ из элементов $R$. Их можество обозначается $M_{m \times n}(R)$.
\end{definition}
\begin{definition} \thmslashn 

Матрица $m \times n$ 
\begin{equation*}
    \begin{bmatrix} 
        a_{11}& a_{12}& \ldots& a_{1n}\\
        a_{21}& a_{22}& \ldots& a_{2n}\\
        \vdots& \vdots& \vdots& \vdots\\
        a_{m1}& a_{m2}& \ldots, a_{mn}
    \end{bmatrix} 
\end{equation*}

Называется матрицей системы
\end{definition}
\TODO{расширенная}

\Subsection{Метод Гаусса}
    Если $a_{11} \neq 0$ (в поле), прибавив первую строку системы к $i>1$-й с коэффициентом $- \frac{a_{i1}}{a_{11}}$ можно убрать вхождение $x_1$ в строку $i$.

    Если $\exists{i}\quad a_{i1} \neq 0$, можно поменять её местами с первой строкой, и применить предыдущее преобразование.

    Если $\forall{i}\quad a_{i1} = 0$, то можно <<забыть>> про первый столбец матрицы.

\begin{definition} \thmslashn 

Элемент $a_{ij}$ называется главным элементом строки $i$ матрицы $A$, если он первый ненулевой элемент в этой строке.
\end{definition}
\begin{definition} \thmslashn 

Матрица $A$ имеет ступенчатый вид, если $\forall{i}\quad $ строка $i$ состоит из нулей, либо позиция главного элемента строки $i$ строго больше позиции главного элемента строки $i-1$.
\end{definition}
\begin{theorem} \thmslashn

    Любую матрицу над полем $K$ можно при помощи элементарных преобразований привести к равносильной ей ступенчатой, такой, что главный коэффициент в каждой не нулевой строке равен $1$ а над каждым главным элементом стоят нули.
\end{theorem}

Рассмотрим расширеную матрицу системы $(A|b)$. Привдём её к равносильной ступенчатой методом Гаусса.

Рассмотрим последнее ненулевое уравнение в такой системе. Пусть оно имеет вид $x_{s} + a_{s+1}x_{s+1} + \ldots + a_{m}x_{m} = b \implies x_{s} = b - \sum\limits_{i=s+1}^{m} a_{i}x_{i}$.

Может быть ситуация, когда главный элемент является частью подматрицы $b$. Тогда у системы нет решений. В остальных случаях есть.

\begin{definition} \thmslashn 

Зависимые переменные - переменные, которые в приведённой матрице соответствуют главным элементам.

Остальные переменные называются независимыми.
\end{definition}
\begin{lemma} \thmslashn

    Зависимые переменные однозначно выражается через независимые.
\end{lemma}
\begin{example}[Задача интерполяции] \thmslashn

    Хотим найти $f(x) = \lambda_0 + \ldots + \lambda_{n-1}x^{n-1}$, такой, что $f(x_{i}) = a_{i}$, $\lambda_{i}, x_{i}, a_{i}\in K$, $x_i \neq x_j$.
    \begin{equation*}
        \begin{split}
            \lambda_0 + \lambda_1 x_1 + \lambda_2 x_1^2 + \ldots + \lambda_{n-1}x_1^{n-1} = a_1\\
            \vdots
        \end{split}
    \end{equation*}

    Её матрица:
    \begin{equation*}
        \begin{bmatrix} 
            1 & x_1 & x_1^2 & \ldots & x_1^{n-1}\\
            \vdots & \vdots & \vdots & \vdots& \vdots\\
            1 & x_{n} & x_{n}^2 & \ldots & x_{n}^{n-1}
        \end{bmatrix} 
    \end{equation*}
    Решение методом Гаусса - $\mathcal{O}(n^3)$, формулой интерполяции - $\mathcal{O}(n^2)$.
\end{example}
\begin{example}[Pagerank] \thmslashn

    Пусть есть орграф $G$ символизирующий набор страниц ссылающихся друг на друга.

    Хотим каждой вершине сопоставить $w_{i} = \sum\limits_{j \to i} \frac{1}{\odeg(j)}w_{j}$.

    $\forall{i}\quad w_{i} = 0$ - точно решение. Есть-ли другие?
\end{example}
\Subsection{Операции над матрицами}
\begin{definition} \thmslashn 

    Пусть $A, B\in M_{m \times n}(R)$, то $\exists{C\in M_{m \times n}(R)}\quad C = A+B$. При этом, $C_{ij} = A_{ij} + B_{ij}$.
\end{definition}
\begin{definition} \thmslashn 

    Пусть $A\in M_{m \times n}(R)$, $\lambda\in R$. Тогда $\exists{\lambda A\in M_{m \times n}(R)}\quad (\lambda A)_{ij} = \lambda A_{ij}$.
\end{definition}
\begin{properties} \thmslashn

    \begin{enumerate}
        \item $M_{m \times n}(R)$ - абелева группа по сложению
        \item $A, B\in M_{m \times n}(R)$, $\lambda\in R$. $\lambda (A + B) = \lambda A + \lambda B$.
        \item $A\in M_{m \times n}(R)$, $\lambda, \mu\in R$, $(\lambda + \mu)A = \lambda A + \mu A$.
        \item  $A\in M_{m \times n}(R)$, $\lambda, \mu\in R$, $(\lambda\mu)A = \lambda(\mu A)$.
        \item  $1A = A$.
    \end{enumerate}
\end{properties}
\begin{remark} \thmslashn

    \begin{equation*}
        \begin{bmatrix} x_1\\ \vdots\\ x_{n} \end{bmatrix}\in R^{n} = M_{n \times 1}(R) 
    \end{equation*}

    \begin{equation*}
        \left( a_1, \ldots, a_{n} \right)\in M_{1 \times n}(R) 
    \end{equation*}

\end{remark}
\begin{definition} \thmslashn 

\begin{equation*}
    \left( a_1, \ldots, a_{n} \right) \begin{bmatrix} x_1\\ \vdots\\ x_{n} \end{bmatrix} = a_1x_1 + \ldots + a_{n}x_n  
\end{equation*}
\end{definition}
\begin{definition} \thmslashn 

\begin{equation*}
    \begin{bmatrix} 
        a_{11} & \ldots & a_{1m}\\
        \vdots & \vdots & \vdots\\
        a_{n1} & \ldots & a_{nm}
    \end{bmatrix} 
    \begin{bmatrix} x_1\\ \vdots\\ x_{n} \end{bmatrix}
\end{equation*}
\begin{equation*}
    (Ax)_{i} = \sum\limits_{j=1}^{n} A_{ij}x_{j}
\end{equation*}
\end{definition}
\begin{definition} \thmslashn

    Произведение матриц: $\cdot : M_{m \times n}(R) \times M_{n \times k}(R) \mapsto M_{m \times k}(R)$
    \[ (AB)_{ij} = \sum\limits_{s=1}^{k} A_{is}B_{sj} .\] 
\end{definition}
\begin{properties} \thmslashn

    $A, B, C\in M_{* \times *}(R)$ (размеры любые, но такие, чтобы произведения были определены). $\lambda\in R$
    \begin{enumerate}
        \item $(AB)C = A(BC)$
        \item  $\lambda (AB) = (\lambda A) = A(\lambda B)$
        \item $C(A+B) = CA + CB$
        \item  $0_{M} = \begin{bmatrix} 0 & \ldots & 0\\ \vdots & \vdots & \vdots\\ 0 & \ldots & 0 \end{bmatrix} $. $0_{M}A = A$.
        \item $I = \begin{bmatrix} 1 & 0 & \ldots & 0\\ 0 & 1 & \ldots & 0\\ \vdots & \vdots & \vdots & \vdots\\ 0 & 0 & \ldots & 1 \end{bmatrix} $. $IA = AI = A$.
    \end{enumerate}
\end{properties}
\begin{consequence} \thmslashn

    $R^{n} \mapsto R^{m}$ заданное матрицей $A\in M_{m \times n}(R)$, такое, что $x \to Ax$ является гомоморфизмом групп.
\end{consequence}
\begin{definition} \thmslashn 

$\Ker A = \{x\in R^{n}\ssep Ax = 0\} $.

Ядро матрицы - множество решений однородной системы уравнений соответствующей этой матрице.
\end{definition}
\begin{consequence} \thmslashn

    Если система уравнений $Ax = b$ имеет решение, то есть биекция между множеством её решений и $\Ker A$.
    \begin{proof}
        Возьмём такое $x_0$, что $Ax_0 = b$.

        Пусть $x^{*}$ такое, что $Ax^{*} = b$, тогда $x^{*} - x_0\in \Ker A$.

        Пусть $y\in \Ker A$, тогда $A(y + x_0) = b$.
    \end{proof}
\end{consequence}
\begin{lemma} \thmslashn

    Пусть $x\in \Ker A$, $\lambda\in R$. Тогда $\lambda x\in \Ker A$.
\end{lemma}
\begin{definition}[Векторное пространство] \thmslashn 

    Векторным пространством называется четвёрка $\left<V, K, +, \cdot\right>$, где $V$ называется множеством векторов, $K$ - полем скаляров, $+ : V \times V \mapsto V$, $\cdot : K \times V \mapsto V$. При этом, операции удовлетворяют следующим свойствам:
    \begin{enumerate}
        \item $\left<V, +\right>$ - абелева группа
        \item $\lambda(u+v) = \lambda u + \lambda v$
        \item $\left( \lambda + \mu \right)u = \lambda u + \mu u $ 
        \item $\lambda (\mu u) = (\lambda \mu) u$
        \item  $1 u = u$
    \end{enumerate}
\end{definition}
\begin{example} \thmslashn

    \begin{enumerate}
        \item[0] $\{0\} $ 
        \item $\left<K, K, +, \cdot \right>$ 
        \item $\left<K^{n}, K, +, \cdot \right>$, $\left<M_{m \times n}(K), K, +, \cdot \right>$ 
        \item $X$ - множество. $K^{X}$ - множество функций из $X$ в $K$.
        \item $\left<C\left[0, 1\right], \mathbb{R}, +, \cdot \right>$
    \end{enumerate}
\end{example}
\begin{definition} \thmslashn 

Векторным подпространством $U$ пространства $\left<V, K, +, \cdot \right>$, $U \subset V$.\\
$u_1, u_2\in U$, $\lambda\in K$.
\[ u_1 + u_2\in U .\]
\[ \lambda u_1\in U .\]
\[ 0\in U .\] 
\end{definition}
\begin{example} \thmslashn
    
    $\Ker A$ - подпространство в $K^{n}$ ($A\in M_{m \times n}(K)$).

    $C^{1}[0, 1] \le C[0, 1]$ 

$C[0, 1] \ge \{f\in C[0, 1]\ssep f(\frac{1}{2}) = 0\} $ 

$K[x] \ge K_{\le n}[x] = \{f\in K[x] \ssep \deg f \le n\} $  
\end{example}
\begin{definition}[Линейная комбинация] \thmslashn 

    Пусть $v_1, \ldots, v_{n}\in V$, $\lambda_1, \ldots, \lambda_{n}\in K$.

    Их линейной комбинацией называется $\lambda_1v_1 + \ldots + \lambda_{n}v_{n}$
\end{definition}
\begin{definition} \thmslashn 

    Пусть $X \subset V$ (подмножество).

    Пространством, порождённым $X$ (линейной оболочкой $X$) называется:
    \begin{enumerate}
        \item наименьшее по включению подпространство содержащие $X$
        \item $\left<X\right> = \{\lambda_1x_1 + \ldots \lambda_{n}x_{n}\ssep x_{i}\in X\} $
    \end{enumerate}
\end{definition}
\begin{definition} \thmslashn 

    Если $\left<X\right> = V$, то $X$ пораждает $V$. (элементы $X$ - образующие $V$)
\end{definition}
\begin{definition} \thmslashn 

Пусть есть набор векторов $v_1, \ldots, v_{n}\in V$.

Набор линейное зависим, если 
\[ \exists{\lambda_1, \ldots, \lambda_{n} \text{ хотя-бы один из которых не $0$}}\quad \lambda_1v_1 + \ldots + \lambda_{n}v_{n} = 0 .\] 
\end{definition}
\begin{definition} \thmslashn 

Набор векторов $v_1, \ldots, v_{n}\in V$ называется базисом $V$, если он образует $V$ и при этом линейно независим.
\end{definition}
