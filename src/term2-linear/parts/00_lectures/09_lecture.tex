\SectionLecture{Лекция 9}{Игорь Энгель}
\begin{theorem} \thmslashn

    Пусть $U \le K^{n}$, такое, что $\codim U = d \iff \dim U = n - d$.

    Тогда $U$ можно задать $d$ уравнениями.
    \begin{equation*}
        \exists{A\in M_{d \times n}(K)}\quad U = \{x\in K^{n}\ssep Ax = 0\} = \Ker A 
    \end{equation*}
    \begin{equation*}
        \exists{f : K^{n} \mapsto K^{d}}\quad U = \Ker f 
    \end{equation*}
    \begin{proof} \thmslashn
    
        Выберем базис $U$ - $e_{d+1}, \ldots, e_{n}$.

        Дополним до базиса $K^{n}$ с помощью $e_1, \ldots, e_{d}$.

        Тогда $\forall{i > d}\quad f(e_{i}) = 0$. $\forall{i \le d}\quad f(e_{i}) = k_{i}$, где $k$ - стандартный базис $K^{d}$.

        Включение $U \subset \Ker f$ очевидно.

        При этом, $\dim \Im f = d \implies \dim \Ker f = n - d \implies U = \Ker f$.
    \end{proof}
\end{theorem}
\begin{theorem} \thmslashn

    Пусть $A\in M_{m \times n}(K)$, то $\exists{D\in M_{m \times m}, C\in M_{n \times n} \text{ - обратимые}}\quad A = D \begin{bmatrix} E_{r} & 0\\ 0 & 0 \end{bmatrix}C $.
    \begin{proof} \thmslashn
    
        Пусть $e$, $f$ - стандартные базисы $K^{n}, K^{m}$.

        Пусть $L : K^{n} \mapsto K^{m}$ - такое, что $[L]_{e}^{f} = A$.

        Возьмём базисы $e'\in K^{n}, f'\in K^{m}$, такие, что $[L]_{e'}^{f'} = \begin{bmatrix} E_{r} & 0\\ 0 & 0 \end{bmatrix} $

        Тогда $C = (D_{e'}^{e})^{-1}$, $D = D_{f'}^{f}$

    \end{proof}
\end{theorem}
\begin{theorem}[Ранговое разложение] \thmslashn

    Пусть $A\in M_{m \times n}(K)$. $\rk A = r$.

    Тогда $\exists{B\in M_{m \times r}, C\in M_{r \times m}}\quad A = BC $
    \begin{proof} \thmslashn
    
        Возьмём $V = \Im A$. Возьмём $v$ - базис $\Im A$. Возьмём $e, f$ - стандартные базисы $K^{n}, K^{m}$.

        Пусть возмьём такие $L_1, L_2$, такие, что $(x \to Ax) = L_2 \circ L_1$. Тогда $A = [L_2]_{v}^{f}[L_1]_{e}^{v}$.
    \end{proof}
    \begin{proof} \thmslashn
    
        Предствим $A$ как $D \begin{bmatrix} E_{r} & 0\\ 0 & 0 \end{bmatrix} C$.

        Заметим, что $\begin{bmatrix} E_{r} & 0\\ 0 & 0 \end{bmatrix} = \begin{bmatrix} 1\\ \vdots\\ 1\\ 0\\ \vdots\\ 0 \end{bmatrix} \cdot \begin{bmatrix} 1 & \ldots & 1 & 0 & \ldots & 0 \end{bmatrix}  $.

        Тогда $A = \left( D \begin{bmatrix} 1\\ \vdots \\ 1\\ 0\\ \vdots\\ 0 \end{bmatrix}  \right) \left( \begin{bmatrix} 1 & \ldots & 1 & 0 & \ldots & 0 \end{bmatrix}C  \right) $
    \end{proof}
\end{theorem}
\begin{statement} \thmslashn

    Пусть $L_1 : U \mapsto V$, $L_2 V \mapsto W$.

    Тогда $\rk L_2 \circ L_1 \le \min(\rk L_1, \rk L_2)$.
    \begin{proof} \thmslashn
    
       \[ \rk L_2 \circ L_1 = \dim \Im L_1 \circ L_2 \subset \Im L_2 \implies \rk L_2 \circ L_1 \le \rk L_2 .\]
       \[ \dim \Im L_1 = \rk L_1 \implies \dim L_2(\Im L_1) \le \dim \Im L_1 \implies \rk L_2 \circ L_1 \le \rk L_1 .\qedhere\] 
    \end{proof}
\end{statement}
\begin{consequence} \thmslashn

    Пусть $T : U' \mapsto U$, $L : U \mapsto V$, $S : \mapsto V'$.

    Тогда $\rk S \circ L \circ T = \rk L$.
    \begin{proof} \thmslashn
    
        \[ L = S^{-1}\circ S\circ L\circ T\circ T^{-1} \implies \rk L \le \rk S \circ L\circ T .\] 
    \end{proof}
\end{consequence}
\begin{theorem} \thmslashn

    \[ \rk A + B \le \rk A + \rk B .\] 
    \begin{proof} \thmslashn
    
        \[ \Im A + B \subset \Im A + \Im B .\] 
    \end{proof}
\end{theorem}
\begin{statement} \thmslashn

    \[ \rk A + B \ge |\rk A - \rk B| .\]
    \begin{proof} \thmslashn
    
        Пусть $\mathbb{E}_{m \times n}^{r}\in M_{m \times n}(K) = \begin{bmatrix} E_{r} & 0\\ 0 & 0\end{bmatrix}$
        
        Без обграничения общности, $\rk A \ge \rk B$.
        .
        Выберем такую систему координат, что $B = \mathbb{E}_{m \times n}^{r}$.

        В $A$ есть $\rk A$ независимых столбцов. В результате прибавления $B$ изменилось не более $\rk B$ столбцов. Уберём их из рассмотрения, точно осталось $\rk A - \rk B$ независимых столбцов.
    \end{proof}
\end{statement}
\begin{definition} \thmslashn 

    Пусть $A\in M_{m \times n}$. Тогда $\rk_{\text{row}} A = \dim \left<\text{строки $A$}\right>$
\end{definition}
\begin{definition} \thmslashn 

    Пусть $A\in M_{m \times n}$. Тогда транпонированная матрица $A^{T}\in M_{n \times m}$ - такая матрица, что $A_{ij}^{T} = A_{ji}$.
\end{definition}
\begin{statement} \thmslashn

    \[ \rk_{\text{row}} A = \rk A^{T} .\] 
\end{statement}
\begin{properties} \thmslashn

    \begin{enumerate}
        \item $(A+B)^{T} = A^{T} + B^{T}$ 
        \item $(\lambda A)^{T} = \lambda A^{T}$ 
        \item $(AB)^{T} = B^{T}A^{T}$ 
        \item $A$ - обратимая. Тогда $(A^{T})^{-1} = (A^{-1})^{T}$
    \end{enumerate}
\end{properties}
\begin{statement} \thmslashn

    \[ \rk_{\text{row}} A = \rk A .\] 
    \begin{proof} \thmslashn
    
        Пусть $A = C \mathbb{E}_{m \times n}^{r}D$.

        Тогда $A^{T} = D^{T}\mathbb{E}_{n \times m}^{r}C^{T}$.

        Заметим, что $\rk A^{T} = \rk A$.
    \end{proof}
\end{statement}
\begin{statement} \thmslashn

    Каждое элементарное преобразование можно представить как домножение на матрицу.
    \begin{proof} \thmslashn
    
        $E_{ij}(\lambda) = E_{n} + \lambda e_{ij}$ - к $i$-й строке прибавить $j$-ю строку домноженную на $\lambda$
        
        $P_{(ij)} = E_{n} - e_{ii} - e_{jj} + e_{ij} + e_{ji}$ - поменять местами $i$-ю и  $j$-ю строки.

        $D_{i}(\lambda) = E_{n} - e_{ii} + \lambda e_{ii}$ - Домножение $i$-й строки на $\lambda\in K^{*}$

        $E_{ij}(\lambda)^{-1} = E_{ij}(-\lambda)$, $P_{(ij)}^{-1} = P_{(ij)}$, $D_{i}(\lambda)^{-1} = D_{i}\left( \lambda^{-1} \right) $
    \end{proof}
\end{statement}
\begin{remark} \thmslashn

    $AP_{(ij)}$ - поменять $i$-й и $j$-й столбец местами.
    
    $AD_{i}(\lambda)$ - домножить $i$-й столбец на $\lambda$.

    $AE_{ij}(\lambda)$ - добавить в $j$-му столбцу $i$-й столбец умноженный на $\lambda$.
\end{remark}
\begin{definition} \thmslashn 

    Пусть $\sigma\in S_{n}$. Тогда $(P_{\sigma})_{ij} = \begin{cases}
        0 & \sigma(j) \neq i\\
        1 
    \end{cases}$
\end{definition}
\begin{definition} \thmslashn 

    Матрица $A$ называется нижнетреугольной и обозначается $A\in LT_{n}(K)$, если $\forall{i < j}\quad A_{ij} = 0$.

    Матрица $A$ называется верхнетреугольной, $A\in UT_{n}(K)$, если $\forall{i > j}\quad A_{ij} = 0$.
\end{definition}
\begin{statement} \thmslashn

    Пусть $A,B\in LT_{n}(K)$.

    Тогда
    \begin{enumerate}
        \item $A + B\in LT_{n}$
        \item $\lambda A\in LT_{n}(K)$ 
        \item $AB\in LT_{n}(K)$ 
            \begin{proof} \thmslashn
            
                \[ (AB)_{ij} = \sum\limits_{k=1}^{n} A_{ik}B_{kj} = \sum\limits_{k=j}^{i} A_{ik}B_{kj} \implies (j > i \implies (AB)_{ij} = 0) .\] 
            \end{proof}
        \item $A^{-1}\in LT_{n}(K)$
            \begin{proof} \thmslashn
            
               \begin{lemma} \thmslashn
               
                   $\forall{A\in LT_{n}}\quad $, такая, что $A$ обратима, $A$ раскладывается в виде произведения $F_{i}$ - элементарные преобразования вида $E_{st}(\lambda)$, $s>t$ и $D_{s}(\lambda)$
                   \begin{proof} \thmslashn
                   
                       Если матрица обратима, то все элементы на главной диагонали не $0$, и переставлять сточки не надо.

                       Применим метод Гаусса, сначала поскладываем строки, получиться диагональная матрица. Потом можем домножить $i$-ю строку на $\frac{1}{A_{ii}}$, получим единичную. Все матрицы которые применяли обратимы, значит, разложили в произведение.
                   \end{proof}
               \end{lemma}

               Если $A$ обратима, то $A$ - произведение матриц элементарных проеобразований, тогда $A^{-1}$ - произвдение обратных к ним, а они тоже нижнетреугольные.
            \end{proof}
            \begin{proof} \thmslashn
            
                Без ограничения общности, на главной диагонали $A$ стоят единицы.

                Представим $A = E_{n} + N$. У $N$ на диагонали и выше нули.

                Тогда $N^{n} = 0$ (заметим, что $N(e_1)\in \left<e_2, \ldots, e_{n}\right>$, $N(e_{2})\in \left<e_3, \ldots, e_{n}\right>, \ldots, N(e_{n}) = 0$). Тогда, на кадой итерации мы теряем хотя-бы один вектор. После $n$ шагов везде нули.

                Тогда
                \[ A^{-1} = (E_{n} + N)^{-1} = E_{n} - N + N^2 - N^3 + \ldots + (-1)^{n-1}N^{n-1} .\qedhere\] 
            \end{proof}
    \end{enumerate}
\end{statement}
\begin{theorem} \thmslashn

    Любую обратимую матрицу $A\in M_{n \times n}(K)$ можно разложить как $PA = LU$, $L\in LT_{n}(K)$, $M\in UT_{n}(K)$, $P$ - матрица перестановки.
    \begin{proof} \thmslashn
        
        \TODO
    \end{proof}
\end{theorem}
\begin{consequence} \thmslashn

    Обратимая матрица $A\in M_{n \times n}(K)$ имеет $LU$ разложение, когда подматрицы вида 
    
    \[ A_{i} =  \begin{bmatrix} A_{11} & A_{12} & \ldots & A_{1i}\\ A_{21} & A_{22} & \ldots & A_{2i}\\ \vdots & \vdots & \vdots & \vdots\\ A_{i1} & A_{i2} & \ldots & A_{ii} \end{bmatrix} .\]
    
    обратимы. Причём, если зафиксировать элементы диагонали $L$, то такое разложение единственно.
    \begin{proof} \thmslashn
    
        \TODO    
    \end{proof}
\end{consequence}
