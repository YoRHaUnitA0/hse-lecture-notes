% !TEX encoding = UTF-8 Unicode
\documentclass[11pt, oneside]{article}   	% use "amsart" instead of "article" for AMSLaTeX format
\usepackage{amssymb}
\usepackage{amsmath}
\usepackage{cRussian}
\usepackage{cPicture}
\usepackage{cAlgebra}
%\usepackage{cFonts}
%\usepackage{cTikz}
\title{Алгебра 3}
\author{Igor Engel}
\date{}

\begin{document}
\maketitle
\section{}
$X$ - множество, на котором задана бинарная операция  $(\cdot ): X \times X \mapsto X$\\
$(\cdot )$ - ассоциативна, если $\forall{a,b,c\in X}\quad (a\cdot b)\cdot c = a\cdot (b\cdot c)$\\
$(\cdot )$ - коммутотивна, если $\forall{a, b\in X}\quad a\cdot b = b\cdot a$\\
У $(\cdot )$ есть нейтральный элемент, если $\exists{1\in X}\quad \forall{x\in X}\quad 1\cdot X$\\
$x\in X$ обратим, если $\exists{x^{-1}\in X}\quad x \cdot x^{-1} = x^{-1} \cdot x = 1$. $x^{-1}$ называется обратным к $x$.\\
$x^{n} = \underbrace{x\cdot x\cdot \ldots \cdot  x}_n$ \\
Нейтральный элемент единственнен.\\
\begin{definition}
    Пара $\left<X, \cdot \right>$ называется моноидом, если:
    \begin{itemize}
        \item $\cdot$ - ассоциативна
        \item Сещуствует нейтральный элемент
    \end{itemize}
\end{definition}
\begin{dlemma}
    Если $X$ - моноид, а  $x, y\in X$ - обратимые элементы, и $x\cdot y$ - обратимо, то $(x\cdot y)^{-1} = y^{-1} \cdot x^{-1} $.
    \begin{proof}
        Рассмотрим произведение
        \[ (x\cdot y) \cdot y^{-1} \cdot x^{-1} .\]
        \[ x \cdot y \cdot y^{-1} \cdot x^{-1} = x \cdot 1 \cdot x^{-1} = 1  .\] 
    \end{proof}
\end{dlemma}
\begin{dlemma}
    Если $X$ - моноид, то обраиный элемент единственнен
\end{dlemma}
\begin{dlemma}
    Если $x$ обратим,  $x^{-1^{-1}} = x$
\end{dlemma}
\begin{definition}
    $\left<G, \cdot\right>$ называется группой, если:\\
    $\left<G, \cdot \right>$ - мониод\\
    Любой $g\in G$ обратим\\
    \begin{itemize}
        \item $X$ - множество. Рассмотрим  $S_X = \{f: X \mapsto X\ssep f \text{- обратима}\}$. Тогда $\left<S_X, \circ\right>$ - группа
    \end{itemize}
\end{definition}
\begin{definition}
    $\left<G, \cdot \right>$ - абелева группа, если:\\
    $\left<G, \cdot \right>$ - группа\\
    $(\cdot) $ - коммутативна\\
    Примеры:
    \begin{itemize}
        \item $\left<Z_{/n}, \cdot \right>$
        \item $\left<\mathbb{Z}|\mathbb{Q}|\mathbb{R}, +\right>$
    \end{itemize}
\end{definition}
\begin{definition}
    Пусть $G$ - группа.\\
    $H \subset G$.\\
    $H$ - подгруппа $G$, если:
     \begin{enumerate}
        \item $\forall{a, b\in H}\quad a\cdot b\in H$
        \item $\forall{a\in H}\quad a^{-1}\in H$
        \item $1\in H$
    \end{enumerate}
    Примеры:\\
    \begin{itemize}
        \item Плоскость $\mathbb{R}^2$, $S_{\mathbb{R}^2} = \{f: \mathbb{R}^2 \mapsto \mathbb{R}^2\ssep f \text{- биекция}\} $. Подгруппа: $\text{Isom}_{\mathbb{R}^2} = \{f\in S_{\mathbb{R}^2} \ssep \forall{\left<x, y\right>\in \mathbb{R}^2}\quad  \|f(x)-f(y)\| = \|x-y\|\} $.\\
        \item Рассмотрим подгруппу внутри $\text{Ison}_{\mathbb{R}^2}$. $H = \{f\in \text{Ison}_{\mathbb{R}^2}\ssep f(x_0) = x_0\} $
    \end{itemize}
\end{definition}
\begin{definition}
    Если $X = \{1, \ldots, n\} $, то $S_X$ называется группой перестановок и обазначатеся $S_n$.\\
    Если $n\ge 3$, то группа перестановок неабелева.
\end{definition}
\begin{definition}
    Пусть $G_1, G_2$ - группы. Рассмотрим группу  $G_1 \times G_2$. Операция этой группы:\\
    \[ \left<g_1, g_2\right>\cdot \left<h_1, h_2\right> = \left<g_1h_1, g_2h_2\right> .\]
    Нейтральный элемент:
    \[ \left<1_1, 1_2\right> .\]
    Обратный к $\left<g_1, g_2\right>$:
    \[ \left<g_1^{-1}, g_2^{-1}\right> .\] 
\end{definition}
\begin{definition}
    Пусть $G$ - группа.  $x\in \mathbb{G}$.\\
    Определим $x^{n}$, $n\in \mathbb{Z}$ :
    \begin{equation*}
        x^{n} = 
        \begin{cases}
            x^{n},& n > 0\\
            1,& n = 0\\
            \left( x^{-1} \right) ^{|n|},&n < 0
        \end{cases}
    \end{equation*}
    \[ x^{n+m} = x^{n}x^{m} .\]
    \[ (x^{n})^{m} = x^{nm} .\] 
\end{definition}
\begin{definition}
    Набор $\left<R, +, \cdot \right>$ - кльцо, если:
    \begin{itemize}
        \item $<R, +>$ - абелева группа\\
        \item  $\forall{a,b,c\in R}\quad (a+b)\cdot c = c\cdot (a+b) = a\cdot c+b\cdot c$
    \end{itemize}
    Кольцо $R$ - ассцоиативное, если $(\cdot)$ - ассоциативна
    Кольцо $R$ - коммутативное, если $(\cdot)$ - коммутативно
    Кольцо $R$ - кольцо с единицей, если у $(\cdot)$ существует нейтральный элемент.
\end{definition}
\begin{dlemma}
    Если $R$ - кольцо, то:
    \[ a\cdot 0 = 0 \cdot a = 0 .\]
    \[ a \cdot (-b) = (-a) \cdot b = -(a\cdot b) .\]
    \[ (-1)\cdot a = a\cdot (-1) = -a\text{, если $R$ - кольцо с единицей} .\]
    \begin{proof}
        \[ a\cdot 0 = a\cdot (0+0) = a\cdot 0 + a\cdot 0 \implies a\cdot 0 = 0 .\]
        \[ a \cdot  \left( b+(-b) \right) = a\cdot 0 =0 .\]
        \[ a \cdot (b+(-b)) = a\cdot b + a \cdot -b = 0 \implies a\cdot (-b) = -(a\cdot b) .\] 
    \end{proof}
\end{dlemma}
\begin{dlemma}
    Если $R$ - коммутатвное кольцо и $b\in R$ обратим, то $\frac{a}{b} = a\cdot b^{-1}$.
\end{dlemma}
\begin{definition}
    $R$ - коммутативное ассоциативное кольцо является полем, если:
    \[ \forall{r\in R\setminus \{0\} }\quad \text{$r$ - обратимо} .\]
    Примеры:
    \begin{itemize}
        \item $\left<\mathbb{Q}|\mathbb{R}, +, \cdot \right>$ 
        \item $\left<\mathbb{Z}_{/p}, +, \cdot \right>$, если $p\in \mathbb{P}$
    \end{itemize}
\end{definition}
\end{document} 
