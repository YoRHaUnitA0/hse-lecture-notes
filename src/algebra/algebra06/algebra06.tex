% !TEX encoding = UTF-8 Unicode
\documentclass[11pt, oneside]{article}   	% use "amsart" instead of "article" for AMSLaTeX format
\usepackage{amssymb}
\usepackage{amsmath}
\usepackage{cRussian}
\usepackage{cPicture}
\usepackage{cTheorem}
\usepackage{cAlgebra}
%\usepackage{cTikz}
\title{Алгебра 6}
\author{Igor Engel}
\date{}

\begin{document}
\maketitle
\section{}
    \begin{theorem}
        \[ \forall{g\in G}\quad \exists!{f: \mathbb{Z} \mapsto  G}\quad f(1) = g  .\] 
        $f$ - гомомоморфизм
        \begin{proof}
            Докажем единственность:
            \[ f(0) = 0 .\]
            \[ f(1) = g .\]
            \[ f(2) = f(1+1) = g^2 .\]
            \[ f(n-1) = g^{n-1} .\]
            Существование теперь тривиально 
        \end{proof}
    \end{theorem}
    \begin{theorem}
        \[ \forall{g\in G}\quad \exists!{f: \mathbb{Z}/\ord g \mapsto \left<g\right>} .\] 
    \end{theorem}
    \begin{definition}
        \[ X \subset G .\]
        Если $G = \left<X\right>$, то $X$ порождает $G$. $X$ - порождающее множество $G$.\\
        Если $X$ кончно, то его элементы порождают $G$.\\
        Если $|X| = 1$, то $G$ - циклическая группа.
    \end{definition}
    \begin{dlemma}
        \[ H \subset \mathbb{Z} .\]
        $H$ - циклическая.
        \begin{proof}
            Если $H = \{0\} $ утверждение тривиально.
            \[ \exists{n\in \mathbb{Z}}\quad H = n \mathbb{Z}  .\]
            Подобное утверждение было доказанно ранее.
        \end{proof}
    \end{dlemma}
    \begin{dlemma}
        \[ g^{n} = e .\]
        \[ n\divby (\ord g=m) .\]
        \begin{proof}
            \begin{equation*}
                \begin{split} 
                    &g^{n} = e\\
                    &\iff g^{mq+r} = e\\
                    &\iff g^{mq}g^{r} = e\\
                    &\iff (g^{m})^{q}g^{r} = e\\
                    &\iff eg^{r} = e\\
                    &\iff g^{r} = e\\
                    &\iff r = 0\\
                    &\iff n\divby m \qedhere  
                \end{split}
            \end{equation*}
        \end{proof}
    \end{dlemma}
\section{Смежные классы и Теорема Лагранжа}
    \begin{definition}
        Заведём отношение $\sim_H$: $g_1 \sim_H g_2 \iff \exists{h\in H}\quad g_1 = g_2h$.
    \end{definition}
    \begin{theorem}
        $\sim_H$ - отношение эквивалентности
        \begin{proof}
            Рефелексивность: $e \in H \implies g_1 = g_1e$\\
            Симметричность: $h^{-1}\in H \implies g_2 = g_1h^{-1}$
            Транзитивность:
            \[ g_1 = g_2h_1\]
            \[ g_2 = g_3h_2 .\]
            \[ g_1 = g_2h_2h_1 .\] 
        \end{proof}
    \end{theorem}
    Рассмотрим классы эквивалентности:
    \begin{definition}
        \[ gH = \{gh \ssep h\in H\}  .\]
        $gH$ - левый смежный класс элемента $g$ относительно $H$.
    \end{definition}
    \[ G = \bigsqcup_{g \in G} gH  .\]
    \begin{dlemma}
        \[ f: H \mapsto gH .\]
        \[ f(h) = gh .\]
        $f$ - биекция.
        \begin{proof}
            Построим обратное отображение:
            \[ F(gh) = g^{-1}gh = h .\] 
        \end{proof}
    \end{dlemma}
    \begin{definition}
        Множество всех левых смежных классов: $G/H$.
    \end{definition}
    \begin{definition}
        $G$ - группа. $|G|$ - порядок $G$, это количество элементов в ней.
        $|G/H| = \left[G : H\right]$ - индекс $H$ внутри $G$. 
    \end{definition}
    \begin{theorem}[Теорема Лагранжа]
        $H$ - подгруппа $G$.\\
        $|H|$ конечен\\
        $\left[G : H\right]$ конечен.\\
        \[ |G| = \left[G : H\right]|H| .\]
        \begin{proof}
            $G$ разбивается на классы смежности, которых $ \left[G : H\right]$.\\
            Так-как из $H$ в $gH$ есть биекция, $|H| = |gH|$.\\
        \end{proof}
    \end{theorem}
    \begin{tlemma}
        \[ |G|\divby |H| .\]
    \end{tlemma}
    \begin{tlemma}
        \[ \forall{g\in G}\quad |G| \divby \ord g  .\]
        \begin{proof}
            \[ \ord g = |\left<g\right>| .\]
            $\left<g\right>$ - подгруппа.
        \end{proof}
    \end{tlemma}
    \begin{tlemma}
        \[ |G| = n \implies g^{n} = e .\]
        \begin{proof}
            \[ n\divby \ord g \implies g^{n} = (g^{\ord g})^{\frac{n}{\ord g}} = e .\] 
        \end{proof}
    \end{tlemma}
    \begin{tlemma}[Теорема Эйлера]
        \[ G = \left( \mathbb{Z}/n \right)^{*}  .\] 
        \[ a\in \left(G\right)  .\]
        \[ \varphi(n) = |G| .\]
        \[ a^{\varphi(n)} = 1 .\]
    \end{tlemma}
    \begin{definition}
        $\varphi(n)$ - функция Эйлера.
    \end{definition}
    \begin{dlemma}
        $p$ - простое.\\
        $\varphi(p) = p-1$.
        \begin{proof}
            Все элементы в $\mathbb{Z}/p$ кроме нуля обратимы.
        \end{proof}
    \end{dlemma}
    \begin{dlemma}
        $p$ - простое.\\
        $\varphi(p^{\alpha}) = p^{\alpha} - p^{\alpha-1}$
    \end{dlemma}
    \begin{dlemma}
        Пусть $n = \prod\limits_{i=1}^{\infty} p_i^{\alpha_i}$\\
        $\varphi(n) = \prod\limits_{i=1}^{\infty} \varphi(p_i^{\alpha_i}) = \prod\limits_{i=1}^{\infty} p_i^{\alpha_i} - p_i^{\alpha_i - 1} = n\prod\limits_{i=1}^{\infty}(1-\frac{1}{p})   $
    \end{dlemma}
    \begin{theorem}[Малая теорема Ферма]
        \[ a^{p-1} \equiv 0\mod p .\] 
        \begin{proof}
            Если $a\equiv 0\mod p$ утверждение тривиально.\\
            $\varphi(p) = p-1$, значит утверждение следует из теоремы Эйлера. 
        \end{proof}
    \end{theorem}
    \begin{theorem}
        Пусть $|G| = p$.\\
        Тогда $G$ - циклическая.\\
        \begin{proof}
            У $G$ не может быть собственных подгрупп, значит, любой элемент кроме нейтрального порождает всю группу.
        \end{proof}
    \end{theorem}

\end{document} 
