% !TEX encoding = UTF-8 Unicode
\documentclass[11pt, oneside]{article}   	% use "amsart" instead of "article" for AMSLaTeX format
\usepackage{amssymb}
\usepackage{amsmath}
\usepackage{cRussian}
\usepackage{cPicture}
\usepackage{cAlgebra}
%\usepackage{cFonts}
%\usepackage{cTikz}
\title{Алгебра 1}
\author{Igor Engel}
\date{}

\catcode`@=11
\def\caseswithdelim#1#2{\left#1\,\vcenter{\normalbaselines\m@th
  \ialign{\strut$##\hfil$&\quad##\hfil\crcr#2\crcr}}\right.}% you might like it without the \strut
\catcode`@=12
%
\def\bcases#1{\caseswithdelim[{#1}}
\def\vcases#1{\caseswithdelim|{#1}}
%

\begin{document}
\maketitle
\begin{definition}
    $a\divby b \equiv \exists x\in\mathbb{Z}\quad a = bx$
\end{definition}
\section{Свойства}
    \begin{dlemma}
        $\forall{a, b, c\in\mathbb{Z}}\quad a\divby{b}\wedge b\divby{c} \implies a\divby{c}$ 
        \begin{proof}
            \[ b = cx_1 .\] 
            \[ a = bx_2 = c(x_1x_2) .\qedhere\] 
        \end{proof}
    \end{dlemma}
    \begin{dlemma}
        $\forall{a_1,b_1,a_2,b_2\in \mathbb{Z}}\quad a_1\divby{b_1}\wedge a_2\divby{b_2} \implies a_1a_2\divby{b_1b_2}$
        \begin{proof}
            \[ a_1=b_1x_1 .\]
            \[ a_2=b_2x_2 .\]
            \[ a_1a_2=(b_1b_2)(x_1x_2). \qedhere\] 
        \end{proof}
    \end{dlemma}
    \begin{dlemma}
        $\forall{a,b,c\in \mathbb{Z}}\quad a\divby{c}\wedge b\divby{c} \implies (a+b)\divby{c}$ 
        \begin{proof}
            \[ a=cx_1 .\]
            \[ b=cx_2 .\]
            \[ a+b=cx_1+cx_2=c(x_1+x_2) . \qedhere\]
        \end{proof}
    \end{dlemma}
    \begin{dlemma}
        $\forall{a,b\in \mathbb{Z}}\quad a\divby{b} \implies \pm{a}\divby{\pm{b}}$
        \begin{proof}
            \[ a = bx = -b\cdot(-x) .\]
            \[ -a = b\cdot(-x) = -bx .\qedhere\]

        \end{proof}
    \end{dlemma}
    \begin{dlemma}
        $\forall{a\in \mathbb{Z}}\quad a\divby{a}$ 
        \begin{proof}
            \[ a=a(1) .\qedhere\] 
        \end{proof}
    \end{dlemma}
    \begin{dlemma}
        $\forall{a\in \mathbb{Z}}\quad a\divby{1}$
        \begin{proof}
            \[ a = 1(a) .\qedhere\] 
        \end{proof}
    \end{dlemma}
    \begin{dlemma}
        $\forall{a, b\in \mathbb{Z}}\quad a\divby b\wedge b\divby a \implies a=\pm{b}$
        \begin{proof}
            \[ a = bx_1.\]
            \[ b = ax_2 = bx_1x_2\implies x_1x_2=1 .\]
            \[ x_1x_2=1\implies x_1=x_2=\pm{1} .\]
            \[ x_1=\pm{1}\implies a=\pm{b}.\]
        \end{proof}
    
    \end{dlemma}
\section{Теоремы}
    \begin{theorem}[Теорема о делении с остатком]
        \[ \forall{a \in \mathbb{Z}}\quad\forall{b \in \mathbb{Z}\setminus\{0\}}\quad\exists!q, r \in \mathbb{Z}\quad a=bq + r\wedge r \in \left[0; |b|\right) .\]
        \begin{proof}
            Докажем для $a, b \ge 0$ существование по индукции: Для $a<b$ утверждение тривиально:
            \[ a=b_0+a .\] 
            Из того, что разложение существует для $a$ следует существование для  $a+b$:
            \[ a=bq+r \implies a+b=b\left( q+1 \right) +r .\] 
            Докажем единственность. Предположим что существуют два разложения:
                \[ a=bq+r=bq'+r' .\]
            Вычтем одно из другого:
            \[ b(q-q')+(r-r')=0\implies (r'-r)=b(q-q') .\]
            Если предположить что $q \neq q' $
            \[ 
                \begin{cases}
                   |b(q-q')|\ge |B|\\
                   |(r'-r)|< |B|\\
                \end{cases} 
            \] 
            Что невозможно. Значит $q=q'$:
            \[ (r'-r)=b(q-q')=b(q-q)=0\implies r=r' .\]
            Значит, второе разложение равно первому.
        \end{proof}
    \end{theorem}
\section{НОД}
    \begin{definition}
        Общий делитель чисел $a$ и $b$ - такое число $d$, что $a\divby{d}$ и $b\divby{d}$
    \end{definition}
    \begin{definition}
        $d = \left(a,b\right) = \text{НОД($a$,  $b$)} \equiv \forall{d' \in \mathbb{Z}}\quad a\divby{d}\wedge b\divby{d}\wedge ((a\divby{d'}\wedge b\divby{d'}) \implies d\divby{d'})$
    \end{definition}
    \begin{dlemma}
        Пусть $a, b \in \mathbb{Z}$ и $a \neq 0$ или $b \neq 0$. 
        \[ A \equiv \{z \in \mathbb{Z}\ssep \exists{x, y \in \mathbb{Z}}\quad z=ax+by\} .\]
        Тогда,
        \[ \exists d \in \mathbb{N}\quad A = d\cdot \mathbb{Z} = \{z \in \mathbb{Z}\ssep\exists{x \in \mathbb{Z}\quad z=dx}\} .\]
        \[ d = \left(a, b\right) = \min\limits_{d' \in A \cup \mathbb{N}}d' .\]
        \begin{proof} $d = \min\limits_{d' \in A\cup \mathbb{N}}d'$ и $A \subseteq d\cdot \mathbb{Z}$. Предположим $a, b \neq 0$
            \[ z = ax+by = dq+r\quad q \in \mathbb{Z} \quad r \in \left[0, d\right) .\]
            \[ d = ax'+by' .\]
            \[ r = z-dq = ax+by-aqx'-bqy' = a(x-qx')+b(y-qy') \implies r \in A \implies r \not\in \mathbb{N} \implies r = 0 .\] 
            \[ r=0 \implies z\divby{d} \implies z \in d\cdot \mathbb{Z} . \qedhere\]
        \end{proof}
        \begin{proof} $A = d\cdot \mathbb{Z}$:
            \[ \forall{z \in d\cdot \mathbb{Z}}\quad \exists{c \in \mathbb{Z}}\quad z=dc .\]
            \[ d = ax+by \implies z = dc = acx+bcy \implies z \in A \implies d \cdot  \mathbb{Z} \subseteq A.\]
            \[ A \subseteq d \cdot  \mathbb{Z}\wedge d\cdot \mathbb{Z} \subseteq A \implies A = d \cdot \mathbb{Z} .\qedhere\] 
        \end{proof}
        \begin{proof}
            $d$ - общий делитель $a$ и $b$
            \[ a = a(1)+b(0) \implies a \in A \implies a\divby{d} .\]
            \[ b = a(0)+b(1) \implies b \in A \implies b \divby{d} .\qedhere\] 
        \end{proof}
        \begin{proof}
            $d = \left(a, b\right)$\\
            Пусть $d'$ - общий делитель $a$ и $b$.
             \[ d \in A \implies d = ax + by .\]
             \[ a\divby{d'} \implies \left( ax \right)\divby{d'} .\]
         \[ b\divby{d'} \implies \left( by \right)\divby{d'} .\]
             \[ \left( ax \right) \divby{d'}\wedge\left( by \right)\divby{d'} \implies \left( ax+by \right)\divby{d'} \implies d\divby{d'} \implies d = \left( a, b \right)   .\qedhere\] 
        \end{proof}
    \end{dlemma}
\section{Простые числа}
   \begin{definition}
       Назовём множество простых чисел $\mathbb{P}$, тогда \[ \mathbb{P} \equiv\{p \in \mathbb{N}\setminus\{1\}\ssep\forall{d \in \mathbb{N}}\quad p\divby{d} \implies d \in \{1, p\}   \}  .\] 
   \end{definition}
   \begin{dlemma}
       \[ p \in \mathbb{P} \equiv \forall{a,b \in \mathbb{Z}}\quad \left( ab \right)\divby{p} \implies \bcases{a\divby{p}\cr b\divby{p}}\]        
    \begin{proof} Необходимость.
        \[ d = \left( a,p \right) \in \{1, p\}  .\]
        \[ d = p \implies a\divby{p} .\] 
        \[ d = 1 \implies \exists{x,y \in \mathbb{Z}}\quad 1 = ax+py \implies b = abx+pby .\]
        \[ \left( ab \right)\divby{p} \implies \left( abx \right)\divby{p} .\]
        \[ p\divby{p} \implies \left( pby \right)\divby{p} .\]
        \[ \left( abx \right)\divby{p}\wedge \left( pby \right)\divby{p} \implies \left( abx+pby \right)\divby{p} \implies b\divby{p}   .\qedhere\]

    \end{proof}
    \begin{proof} Достаточность.\\
        Если $\exists{n,m \in \mathbb{Z}\cup\left(1; p \right) }\quad nm=p$, то
        \[ \bcases{n\divby{p}\cr m\divby{p}} .\]
        но $n, m < p$, так-что таких  $n, m$ не существует.
    \end{proof}
   \end{dlemma}
\section{ОТА}
    \begin{theorem}[Основная теорема арифметики]
    $\forall{a \in \mathbb{Z}\setminus\{0\}\exists!\epsilon \in \{-1, 1\}, p_1 \ldots p_k \in \mathbb{P} \quad \epsilon\prod\limits_{i=1}^{k}p_1i}$
    \textit{Примечание:} $p_1 \ldots p_k$ единственно с точностью до перестановки\\
    \begin{proof} Существование.
        \[ a > 0 \implies \epsilon=1 .\]
        \[ a < 0 \implies \epsilon=1 .\]
        \[ |a| \in \mathbb{P} \implies k=1, p=a .\]
        \[ |a| \not\in \mathbb{P} \implies \exists{n,m \in \mathbb{Z}\cup \left[1; a\right)}\quad nm=a.\qedhere\]  
    \end{proof}
    \begin{proof} Единственность.
        Если $a=\epsilon\prod_{i=1}^{k}p_i=\epsilon'\prod\limits_{i=1}^{n}q_i$
        \begin{proof}
            $\epsilon = \epsilon'$
            \[ a > 0 \implies \epsilon = \epsilon' = 1 .\]
            \[ a < 0 \implies \epsilon = \epsilon' = -1 .\qedhere\] 
        \end{proof}
        Предположим, что $k \le n$. (Для $k \ge n$ доказательство симметрично)
        \begin{proof} $p_1\ldots p_k = q_1\ldots q_k$\\
            Для $k=0$ утверждение тривиально.\\
            Для $k>0$:
            \[ \prod\limits_{i=1}^{k}p_i\divby{q_1}\wedge{q_1 \in \mathbb{P}} \implies \exists{i} p_i\divby{q_1} .\]
            Тогда, на это $p_i$ можно скоратить, и получить разложение с $k'=k-1$\qedhere
        \end{proof}
        \begin{proof} $k=n$\\
            После сокращения всех $p_i$,  $k=0$. Значит,
            \[ a' = \prod\limits_{i=1}^0p_i = 0 = \prod\limits_{i=1}^{n-k}q_i \implies n-k=0 \implies n=k.\qedhere\]
        \end{proof}
        \let\qed\relax
    \end{proof}
    \end{theorem}
    \begin{tlemma}
        \[ \forall{a, b \in Z}\setminus \{0\}  .\] 
        \[ a = \epsilon_a \prod_{i=1}^{k} p_i^{\alpha_i}  .\]
        \[ b = e_b \prod_{n=1}^{k} p_i^{\beta_i}  .\] 
        \[ a\divby{b} \equiv \forall{i \in \left[1, k\right]}\quad \alpha_i \ge \beta_i.\] 
        \begin{proof} Достаточность.
            \[ c = \epsilon' \prod_{i=1}^{k} p_i^{\alpha_i-\beta_i} \implies bc=a \implies a\divby{b}  .\qedhere\]
        \end{proof}
        \begin{proof} Необходимость.
        \[a\divby{b} \implies a=bc \implies c=\epsilon' \prod_{i=1}^{k} p_i^{\gamma_i} \implies \alpha_i = \beta_i + \gamma_i \implies \alpha_i \ge \beta_i .\qedhere\]
        \end{proof}
    \end{tlemma}
    \begin{tlemma}
        \[ \forall{a, b \in Z}\setminus \{0\}  .\] 
        \[ a = \epsilon_a \prod_{i=1}^{k} p_i^{\alpha_i}  .\]
        \[ b = e_b \prod_{n=1}^{k} p_i^{\beta_i}  .\] 
        \[ \phi_i = \min\left( \alpha_i, \beta_i \right)  .\]
        \[ d=\left( a, b \right)=\prod_{i=1}^{k} p_i^{\phi_i}  .\]
        \begin{proof}
            \[\forall{i \in \left[1, k\right)}\quad \phi_i \le  \alpha_i, \beta_i \implies a\divby{d}, b\divby{d} .\]
            \[ \forall{d' \in Z}, a\divby{d'}, b\divby{d'}\quad d' = \prod_{i=1}^{k} p_i^{\gamma_i}  .\] 
            \[ \forall{i \in [1, k)}\quad \gamma_i \le \alpha_i, \beta_i \implies \gamma_i \le \phi_i \implies d\divby{d'} .\qedhere\] 
        \end{proof}
    \end{tlemma}
\end{document} 
