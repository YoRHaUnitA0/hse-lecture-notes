% !TEX encoding = UTF-8 Unicode
\documentclass[11pt, oneside]{article}   	% use "amsart" instead of "article" for AMSLaTeX format
\usepackage{amssymb}
\usepackage{amsmath}
\usepackage{cRussian}
\usepackage{cPicture}
\usepackage{cTheorem}
\usepackage{cAlgebra}
%\usepackage{cTikz}
\title{Алгебра 9}
\author{Igor Engel}
\date{}

\begin{document}
\maketitle
\section{Нормальные подгруппы}
    \begin{definition}
        $g, h\in G$, тогда $ghg^{-1}$ называется сопряжённым к $h$ при помощи $g$.
    \end{definition}
    \begin{definition}
        $h_1, h_2\in G$ называются сопрядёнными, если $\exists{g\in G}\quad h_2 = gh_1g^{-1}$.
    \end{definition}
    \begin{definition}
        $H \le G$ называется нормальной подгруппой $G$ и обозначается $H \unlhd G$, если
        \[ \forall{h\in H}\quad \forall{g\in G}\quad ghg^{-1}\in H  .\] 
    \end{definition}
    \begin{theorem}
        $H \le  G$, следующие услвоия эквивалентны:
        \begin{enumerate}
            \item $H \unlhd G$
            \item  $\forall{g\in G}\quad gHg^{-1} = H$ 
            \item $\forall{g\in G}\quad gH = Hg$ 
            \item $\forall{g\in G}\quad gH \subset Hg$
        \end{enumerate}
        \begin{proof}
            $1 \implies 2$: переформулируем формулировку нормальности: $H \unlhd G \iff \forall{g}\quad gHg^{-1} \subset H$.\\
            Рассмотрим $H = g^{-1}gHg^{-1}g \subset g^{-1}Hg$, пусть $g=k^{-1}$, тогда $H \subset kHk^{-1}$. Значит, $H = \forall{g}\quad gHg^{-1}$.\\
            $2 \implies 3$: $gHg^{-1} = H \implies gH = Hg$ (домножение на $g$ справа)\\
            $3 \implies 4$: тривиально\\
            $4 \implies 1$: $gH \subset Hg \to gh\in Hg \implies gh = h'g \implies ghg^{-1} = h'$.
        \end{proof}
    \end{theorem}
    \begin{tlemma}
        Любая подгруппа абелевой группы - нормальная.\\
        \begin{proof}
            $ghg^{-1} = gg^{-1}h = h$ 
        \end{proof}
    \end{tlemma}
    \begin{definition}
        Факторгруппа по $H \unlhd \left<G, \times\right>$, называется группа состоящая из смежных классов по $H$ с операцией $\cdot $:
        \[ g_1H\cdot g_2H = (g_1 \times g_2)H .\] 
    \end{definition}
    \begin{dlemma}
        Операция задана корректоно.
        \begin{proof}
            Возьмём два класса: $g_1H$, $g_2H$.\\
            Рассмотрим представителей $g_1h_1$, $g_2h_2$.\\
            Тогда $g_1h_1g_2h_2 = g_1g_2g_2^{-1}h_1g_2h_2 = g_1g_2h_3h_2 = g_1g_2h_4\in g_1g_2H$
        \end{proof}
    \end{dlemma}
    \begin{dlemma}
        Факторгруппа - группа.
        \begin{proof} Докажем свойства:\\
            Ассоциативность: $(g_1H\cdot g_2H)\cdot g_3H = (g_1g_2)g_3H = g_1(g_2g_3)H = g_1H(g_2H\cdot g_3H)$
            Нейтральный элемент: $eH = H$.\\
            Обратный элемент: $(gH)^{-1} = g^{-1}H$.
        \end{proof}
    \end{dlemma}
    \begin{theorem}[Теорема о изоморфизме]
        Пуcть $f: G \mapsto G_1$ - гомоморфизм.\\
        $G/\Ker f \cong f(G)$ (гомоморфный образ группы изомофрен факторгруппе по ядру гомоморфизма)\\
        Причём, изоморфизм имеет вид $g\Ker f \mapsto f(g)$.\\
        \begin{proof}
            Пусть $H=\Ker f$\\
            Корректоность: $h\in H$, $f(gh) = f(g)f(h) = f(g)$.\\
            Гомоморфизм: $\hat{f}(g_1g_2H) = f(g_1g_2) = f(g_1)f(g_2) = \hat{f}(g_1H)\hat{f}(g_2H)$.\\
            Инъективность: $gH\in \Ker \hat{f} \iff f(g) = e \iff gH = H$\\
            Сюръективность: $x\in \Ima f$, $f(g) = x$, тогда $f(gH) = x$.
        \end{proof}
    \end{theorem}
    \begin{definition}
        Группа $G$ называется простой, если в ней нет нетривиальных нормальных подгрупп.
    \end{definition}
    \begin{theorem}
        Есть список всех конечных простых групп.
    \end{theorem}
\end{document} 
