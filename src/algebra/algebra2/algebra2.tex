% !TEX encoding = UTF-8 Unicode
\documentclass[11pt, oneside]{article}   	% use "amsart" instead of "article" for AMSLaTeX format
\usepackage{amssymb}
\usepackage{amsmath}
\usepackage{cRussian}
\usepackage{cPicture}
\usepackage{cAlgebra}
%\usepackage{cFonts}
%\usepackage{cTikz}
\title{Алгебра 2}
\author{Igor Engel}
\date{}

\begin{document}
\maketitle
\section{Определения} 
    $\left( a,b \right)$ - НОД($a$,  $b$).
\section{Линейные диофантовы уравнения}
     \[ ax+by=c .\]
    \begin{definition}
        $a$ и  $b$ - взаимно простые, если  $\left( a, b \right) = 1$
    \end{definition}    
    \begin{dlemma}
        Если $n$ и  $m$ - взаимно простые, то $\forall{a} am\divby{n} \implies a\divby{n}$ 
        \begin{proof}
            \[ nx+ny=1 .\]
            \[ anx+any=a .\]
            \[ anx\divby{n}\land amy\divby{n} \implies a\divby{n} .\] 
        \end{proof}
    \end{dlemma}
    \begin{theorem}
        $ax+by=c$ имеет решения только если  $c\divby(d=\left( a,b \right))$
        \begin{proof}
            \[ d = ax'+by'.\] 
            \[ c=\frac{c}{d}\left( ax'+by' \right) \implies x=\frac{c}{d}x'\land y=\frac{c}{d}y'  .\] 
        \end{proof}
    \end{theorem}
    \[ ax+by=c=ax'+by' .\]
    \[ a(x-x')+b(y-y')=0 .\]
    \[ a(x-x')=-b(y-y') .\]
    \[ \frac{a}{d}(x-x')=-\frac{b}{d}(y-y') .\]
     $a$ и  $b$ - взаимно простые. Значит
     \[ \frac{a}{d}t = y-y' .\]
     \[ \frac{a}{d}(x-x')=-\frac{b}{d}t\frac{a}{d} .\]
     \[ x-x'=-\frac{b}{d}t .\] 
     \[ x' = x+\frac{b}{d}t .\]
     \[ y' = y-\frac{a}{d}t .\]
\section{Алгоритм Евкилда}
\[ a=r_{-1}=bq_1+r_1 .\] 
\[ b=r_{0}=r_1q_2+r_2 .\] 
\[ \left(a, b\right) = \left( b, r_1 \right) = \left( r_1, r_2 \right) = \ldots\]
\[ r_i \ge 0 .\]
\[ r_i > r_{i+1} .\]
Бесконечно убываяющая последовательность неотрицательных чисел приходит к нулю. Значит, при определённом $i$,  $r_i=0$, тогда $r_{i-2} = qr_{i-1}+r_i=qr_{i-1}$. $r_{i-1} = \left( a, b \right) $.
Множество  $\left( a,b \right)\cdot \mathbb{Z} = \left( b, r_1 \right) \cdot \mathbb{Z} = \ldots $ 
Значит, любое $r_i \in \left( a, b \right) \cdot \mathbb{Z}$.
    \[ r_{-1} = 1a+0b .\]
    \[ r_0 = 0a+1b .\]
Пусть $r_i = ax+by$,  $r_{i+1} = ax'+by'$, тогда 
\[ r_{i+2} = r_i -q_{i+2}r_{i+2} = a(x-x'q_{i+2})+b(y-y'q_{i+2} .\]
Больше всего итераций если $q_i=1$.
 \[f_0 = r_{k+1}=0 .\]
 \[f_1 = r_k=1 .\]
 \[f_2 = r_{k-1}=r_k+r_{k+1} .\]
\[ f_i = f_{i-1} + f_{i-2} .\]
\begin{theorem}
    $f_n \ge \varphi^{n-2}$. $n>1$. $\varphi$ - золотое сеченией, больший корень  $x^2=x+1$. $\varphi>1$.\\
    \begin{proof}
        
    \[ f_1 = 1 \ge  \frac{1}{\varphi} < 1 .\]
    \[ f_2 = 1 \ge \varphi^0 = 1 .\] 
    \[ f_i = f_{i-1}+f_{i-2} \ge \varphi^{n-3}+\varphi^{n-4}=\varphi^{n-4}\left( \varphi+1 \right) = \varphi^{n-4}\cdot \varphi^2=\varphi^{n-2} .\] 
    \end{proof}
\end{theorem}
\[ a=bq_1+r_1 .\] 
\[ b=r_1q_2+r_2 .\]
\[ N \ge a > b \ge 0 .\] 
Количество делений с остатком в алгоритме евклида $\le 1+FLOOR \log_{\varphi}N$
\begin{theorem}
    \[ r_{k+1-i} \ge f_{i} .\]
   \begin{proof}
        \[ r_{k+1}=0 \ge f_0 = 0 .\]
        \[ r_k \ge f_1=1 .\]
        \[ r_{k+1-i} = r_{k+2-i}q_{k+2-i}+r_{k+3-i} \ge r_{k+2-i}+r_{k+3-i} \ge  f_{i-1}+f_{i-2}=f_{i} .\] 
   \end{proof} 
\end{theorem}
Три последовательности: $\varphi^{k-2} \le f_{k} \le r_{k+1-i}$. Последовательность  $r$ достигнет  $a $ раньше чем  $f$ или  $\varphi$.
 \[ a=r_{-1}\ge f_{k+2}\ge \varphi^{k} .\] 
 $k+1$ - количество делений с остатком. Значит, $\log_{\varphi}N \ge k \implies 1+\log_{\varphi}N \ge k+1$.
\section{Сравнение целых чисел}
\[ x^2+1213x+5321=0 .\]
Есть-ли целый решения?\\
Если $x$ - чётное, то  $x^2$ - чётное, $1213x$ - чётное,  $5321$ - нечётное. Значит решений нет.\\
Если  $x$ - нечётное, то все три числа нечётных, и решений нет.\\
Чётность - остаток от делений на два. Будем приплетать другие остатки. Это теория сравнений. 
\begin{definition}
\[n, a, b \in \mathbb{Z}.\]
\[ (a\equiv b\mod n)=(a\equiv b(n)) = ab\divby{n} .\]
\end{definition}
Замечание: $a \equiv b\mod n$ эквивалетно тому, что остатки от деления на $n$ равны.
 \begin{proof}
     \[ a\equiv b\mod n \implies (a-b)=nk \implies a=nk+b .\]
     \[ a=nq_1+r_1 .\]
     \[ b=nq_2+r_2 .\]
     \[ 0 \le r_1, r_2 < n .\]
     \[ a = nq_2+r_2+nk = n(q_2+k)+r_2 \implies r_2=r_1.\qedhere\]
\end{proof}
\subsection{Свойства}
    \[ a\equiv a\mod n .\]
    \[ a \equiv b\mod n \implies b\equiv a\mod n .\]
    \[ a \equiv b\mod n\land b \equiv c\mod n \implies a\equiv c \mod n .\]
    \[ a\equiv b\mod 0 \implies a=b.\]
    \[ \forall{a, b}\quad a \equiv b \mod 1 .\]
    \[ a\equiv b\mod n \implies a\equiv b\mod -n .\]
    \[ a \equiv b\mod n\land c\equiv d \mod n \implies a+c\equiv b+d\mod n\land ac\equiv bd\mod n .\] 
\end{document} 
\begin{proof}
    \[ a=b+nk .\]
    \[ c=d+ne .\]
    \[ ac=\left( b+nk \right) \left( d+ne \right) =bd+nkd+neb+n^2ke=bd+n\left( kd+eb+ke \right)  .\] 
\end{proof}
\subsection{Линейное сравнение на $x$}
\[ ax\equiv b\mod n .\]
\[ ax=b+ny\implies ax-ny=b .\]
Это линейное диофантого уравнение. $x$ существует если  $b\divby{d=\left( a, n \right) }$.
\[ x= x_0 \frac{n}{d}t.\] 
\[ x\equiv x_0\mod \frac{n}{d} .\]
Решение линейного сравнения единственно по модулю $\frac{n}{d}$.\\
Замечание: $\left( a, b \right) =1$ эквивалентно тому, что $ax\equiv b\mod n$ разрешимо для всех $b$.
\section{Система сравнений}
\begin{equation*}
    \begin{cases}
        x\equiv a_1\mod n_1\\
        \ldots\\
        x\equiv a_k \mod n_k
    \end{cases}
\end{equation*}
\begin{theorem}[Китайская теорема об остатках]
    \begin{equation*}
        \begin{cases}
            x \equiv a\mod n\\
            x \equiv b\mod m
        \end{cases}
    \end{quation*}
\end{theorem}
\end{document}
