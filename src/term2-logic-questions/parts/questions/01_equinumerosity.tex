\Section{Равномощность}{Чистякова Полина}

Это самое начало теории множеств (какое отношение оно имеет к матлогике не очень понятно, но  пусть будет).

\Subsection{Билет 01 <<Равномощные множества>>}

Для множеств определено отношение равномощности 
	- <<множество $A$ равномощно множеству $B$ >> значит, что из одного множества в другое можно построить биекцию.
Иными словами это значит, что любому элементу из множества $A$ сопоставляется ровно один элемент из множества $B$.

Пример двух равномощных множеств 
	- в парке гуляют дети. Если каждому ребёнку на входе в парк подарить шарик, то множества детей и шариков в парке будут равномощны (если никакой ребёнок не отпустит/лопнет шарик).
	
Как у любого отношения у равномощности есть свои свойства.
Это отношение эквивалентности.
Это значит, что оно \begin{itemize}
		    	\item
		    		рефлексивно: <<$A$ равномощно $A$>>
		    	\item
		    		симметрично
		    	\item
		    		транзитивно 
		    \end{itemize}
		    \TODO{Доказательства - АЧИВИДНА}
\TODO{Примеры *их там много и они страаашные :с*}

\Subsection{Билет 02 <<Счётные множества>>}

Билет не просто маленький, он \textit{крошечный}... В книге просто куча воды. Кажется, на экзамене это не попадётся. \\


Счётное множество - множество, равномощное множеству натуральных чисел.
Иными словами - мы просто <<пронумеровали>> все элементы множества.
Биекция будет означать, что у каждого элемента есть номер (сюръекция), и у каждого элемента не больше одного номера (инъекция).

\TODO{надо расписать примеры, ага *но мне лень*}

Самые простые примеры счётных множеств: само \N или множество значений линейной функции от натуральных чисел. (ещё бывает \Z)

Чуть более сложный пример - \Q 

Но его доказательство - следующий билет с: 
\TODO{ссылочка просится :с - можно забить, пусть просится}

\Subsection{Билет 0(3+4) <<Счётность множества рациональных чисел>> + <<Счётность объединения счётного количества счётных множеств>>}

Тут будет ооочень много лемм и теорем, готовьтесь...

\begin{lemma} \thmslashn

	Объединение двух счётных множеств счётно.
	
	\begin{proof} \thmslashn
	
		Рассмотрим два счетных множества A и B; каждое из них можно записать в последовательность:
		\[a_0, \quad a_1, \quad a_2, \quad ...\]
		
		\[b_0, \quad b_1, \quad b_2, \quad ...\]
		
		Теперь можно поочерёдно брать элементы из первой и второй последовательности и записывать в новую (это даст нам $A \cup B$):
		\[a_0, \quad b_0, \quad a_1, \quad b_1, \quad a_2, \quad b_2, \quad a_3, \quad b_3, \quad ...\]
		
		Если $A \cup B = \emptyset$, то мы всё доказали \char`\^\char`\^
		
		Если же это не так, то повторяющиеся элементы мы просто не выписываем.
	\end{proof}
\end{lemma}

\begin{lemma} \thmslashn

	Всякое подмножество счетного множества конечно или счетно.
	\begin{proof} \thmslashn
	
		Пусть у нас есть счётное множество $A$ и его подмножество $A'$. Выпишем множество $A$ в строчку. Затем зачеркнём все элементы, не принадлежащие $A'$. Получим последовательность из всех элементов $A'$: либо конечную (тогда $A'$ конечно), либо бесконечную (тогда $A'$ счётно)
	\end{proof}
\end{lemma}

\begin{lemma} \thmslashn

	Всякое бесконечное множество содержит счётное подмножество.
	\begin{proof} \thmslashn
	
		Выпишем бесконечную последовательность. Возьмём первый элемент случайно (множество не пусто). Дальше будем каждый раз рассматривать дополнение получившейся послдовательности до изначального множества. Оно никогда не кончится (множество бесконечно), значит, мы всегда сможем выписать новый элемент последовательности. Получили бесконечную последовательность. Значит, у изначального множества есть счётное подмножество.
	\end{proof}
\end{lemma}

\begin{lemma} \thmslashn

	Множество рациональных чисел \Q счетно
	\begin{proof} \thmslashn
	
		Докажем сначала отдельно про положительные и отрицательные части \Q. Тогда по одной из предыдущих лемм их объединение будет счётно.
		
		Неотрицательное рациональное число задается парой чисел — числителем и знаменателем. Числитель может быть произвольным натуральным числом, а знаменатель произвольным положительным натуральным числом. Выпишем все такие числа в виде таблицы, бесконечной вниз и вправо:
		\begin{equation*}
			\begin{matrix}
				0/1 & 1/1 & 2/1 & 3/1 & \ldots \\
				0/2 & 1/2 & 2/2 & 3/2 & \ldots \\
				0/3 & 1/3 & 2/3 & 3/3 & \ldots \\
				\vdots & \vdots & \vdots & \vdots & \ddots 
			\end{matrix}
		\end{equation*}
		В этой таблице выписаны все числа (а некоторые даже повторяются......)
		
		Числа из этой таблицы теперь уже легко выписать в последовательность. Например, можно идти по диагоналям (вниз-влево). Сначала выпишем единственное число на первой диагонали (0/1), потом два числа на второй (1/1, 0/2), потом три числа на третьей и так далее:
		\[0/1, 1/1, 0/2, 2/1, 1/2, 0/3, 3/1, 2/2, 1/3, 0/4, \ldots \]
		
		Другими словами, мы сначала выписываем все числа с суммой числителя и знаменателя 1, потом — с суммой 2, потом 3 и так далее. Если мы встречаем число, которое уже выписывали - просто пропускаем его. 
		
		Доказательство для отрицательной части \Q аналогично.
	\end{proof}
\end{lemma}

\begin{theorem} \thmslashn

	Объединение конечного или счётного числа конечных или счётных множеств конечно или счётно.
	\begin{proof} \thmslashn
	
		Пусть есть счётное количество счётных множеств $A_1, A_2, A_3, \ldots$. Выпишем их в табличку:
		\begin{equation*}
			\begin{matrix}
				A_0: & a_{00} & a_{01} & a_{02} & a_{03} & \ldots \\
				A_1: & a_{10} & a_{11} & a_{12} & a_{13} & \ldots \\
				A_2: & a_{20} & a_{21} & a_{22} & a_{23} & \ldots \\
				A_3: & a_{30} & a_{31} & a_{32} & a_{33} & \ldots \\
				\vdots & \vdots & \vdots & \vdots & \vdots & \ddots 
			\end{matrix}
		\end{equation*}
	\end{proof}
\end{theorem}

\begin{theorem} \thmslashn

	Декартово произведение двух счётных множеств A × B cчётно.
	\begin{proof} \thmslashn
	
		Декартово произведение - множество упорядоченных пар вида $(a, b) \ssep a \in A, b \in B$.
		
		Разделим пары на группы - в каждой группе первый элемент пары совпадает. Тогда получим счётно объединение счётных множеств (у нас будет <<$|A|$ штук>> множеств по <<$|B|$ штук>> элементов в каждом)
	\end{proof}
\end{theorem}

\Subsection{Билет 05 <<Добавление счётного множества>>}

\begin{theorem} \thmslashn

	Если множество $A$ бесконечно, а множество $B$ конечно или счётно, то множество $A \cup B$ равномощно $A$.
	\begin{proof} \thmslashn
	
		НУО $A \cap B = \emptyset $ - иначе вместо $B$ берём $B \setminus A$.
		
		Мы знаем, что в $A$ есть счётное подмножество $A_0$. Тогда есть биекция из $A_0 \cup B$ в $B$ (потому что оба множества счётные - биекция черз натуральные числа). Тогда есть биекция из $A \cup B = (A \setminus A_0) \cup (A_0 \cup B)$
	\end{proof}
\end{theorem}

\Subsection{Билет 06 <<Равномощность отрезка [0,1] множеству всех бесконечных последовательностей из 0 и 1>>}
Надеюсь, здесь нужно только то, что в названии билета (\TODO{написать всякие интервалы, полуинтервалы и тд - АЧИВИДНА2})
\begin{theorem} \thmslashn

	*Вставь сюда название билета*
	\begin{proof} \thmslashn
	
		Мы знаем, что $\forall x \in [0,1]$ существует запись $x$ в виде бесконечной двоичной дроби. (\TODO{сюда бы картинку из samples/...}) Но тогда некоторым точкам будут соответсвовать 2 последовательности (например, 0, 1001111... и 0, 101000...). Тогда выкинем все последовательности, заканчивающиеся бесконечным рядом единиц (их счётное число, поэтому так можно)
	\end{proof}
\end{theorem}

\Subsection{Билет 07 <<Равномощность квадрата отрезку>>}

\begin{theorem} \thmslashn

	*Название билета*
	\begin{proof} \thmslashn
	
		Мы знаем, что каждому числу из [0, 1] соответствует одна бесконечная последовательность из 0 и 1. Тогда [0, 1] × [0, 1] соответсвует пара таких последовательностей. Биекция между парой и последовательностью:
		\[(a_0a_1a_2a_3\ldots, b_0b_1b_2b_3\ldots) \rightarrow a_0b_0a_1b_1a_2b_2a_3b_3...\qedhere\]
	\end{proof}
\end{theorem}

\Subsection{Билет 08 <<Теорема Кантора (несчётность отрезка)>>}

\begin{theorem} \thmslashn

	 Множество бесконечных последовательностей нулей и единиц несчётно.
	\begin{proof} \thmslashn
	
		Пусть оно счётно. Пронумеруеми выпишем:
		\begin{equation*}
			\begin{matrix}
				a_0 & = & a_{00} & a_{01} & a_{02} & a_{03} & \ldots \\
				a_1 & = & a_{10} & a_{11} & a_{12} & a_{13} & \ldots \\
				a_2 & = & a_{20} & a_{21} & a_{22} & a_{23} & \ldots \\
				\vdots & = & \vdots & \vdots & \vdots & \vdots & \ddots 
			\end{matrix}
		\end{equation*}
		Теперь посмотрим на последовательность, имеющую вид $b_i = 1 - a_{ii}$. Это последовательность из нашего множества. С другой стороны, она не совпадает с любой другой последовательностью (возьмём номер и посмотрим на нужный член). Ой.
	\end{proof}
\end{theorem}

\Subsection{Билет 09 <<Теорема Кантора-Бернштейна>>}

\begin{definition} \thmslashn

	Множество A имеет мощность не большую, чем множество B, если A равномощно некоторому подмножеству множества B (возможно, совпадающему с B).
\end{definition}

\begin{theorem} \thmslashn

	Если мощность A не больше, чем у B, и одновременно мощность B не больше, чем у A, то A и B равномощны.
	\begin{proof} \thmslashn
	
		Это эквивалентно утвеждению: <<Если для множеств A и B существует инъекция из A в B и инъекция из B в A, то существует и биекция между A и B.>> 
		
		*биекция между множеством A и подмножеством множества B = инъекция из A в B* 
		
		Нарисуем это!! Слева множество $A$, справа - $B$. И проведём все стрелочки-функции $f$ и $g$. Получим возможно бесконечный ориентированный двудольный граф. 
		
		Посмотрим на компоненты связности (если забить на оринтированность). Они бывают либо циклом, либо цепочкой, бесконечной в одну сторону, либо цепочкой, бесконечной в обе стороны.
		
		Почему? Потому что из любой вершины мы точно можем выйти. Но не в любую вершину мы можем войти (отсюда один конец у цепочки). Если мы пришли в вершину, из которой вышли - это цикл,и он конечный, при этом чётной длины.
		В цикле - у нас целых 2 варианта биекции!! (разделим по функциям). В дважды бесконечной цепочке - тоже. А в лишь однажды бесконечной всё определено за нас :с
	\end{proof}
\end{theorem}

\Subsection{Билет 10 <<Теорема Кантора (общая формулировка)>>}

\begin{theorem} \thmslashn

	Никакое множество X не равномощно множеству своих подмножеств.
	\begin{proof} \thmslashn
	
		Пусть не так. Пусть существует биекция $f: X \rightarrow 2^X$. Рассмотрим $Y = \{ x \ssep x \notin f(x) \}$. До противоречия осталось совсем немного! $Y \subset X$. Тогда $\exists y \in X : f(y) = Y$. Тогда
		\[y \notin Y \iff y \notin f(y) \text{- потому что} Y = f(y)\]
		
		С другой стороны, тогда $y \in Y$, потому что $Y$ так строился. 
	\end{proof}
\end{theorem}



\Subsection{Билет 11 <<Операции над мощностями>>}

Мощности конечных множеств — натуральные числа, и их можно складывать, умножать, возводить в степень.

\begin{definition} \thmslashn

	Сумма мощностей множеств - мощность их объединения (если они не пересекаются) или непересекающихся равномощных им в ином случае.
\end{definition}

\begin{definition} \thmslashn

	Произведение мощностей множеств - мощность их Декартового произведения.
\end{definition}

\begin{definition} \thmslashn

	Возведение в степень ($|A|^{|B|}$) - мощность множества $A^B = \{f \ssep f: B \mapsto  A\}$
\end{definition}

\[a + b = b + a\]
\[a + (b + c) = (a + b) + c\]
\[a × b = b × a\]
\[a × (b × c) = (a × b) × c\]
\[(a + b) × c = (a × c) + (b × c)\]
Просто поставь нам биекцию вместо равно!

\begin{theorem} \thmslashn

	\[a^{b+c} = a^b × a^c\]
	
	\begin{proof} \thmslashn
	
		Из чего состоит $A^{B+C}$? Его элементами являются функции со значениями в $A$, определённые на $B + C$. Такая функция состоит из двух частей: своего сужения на $B$ (значения на аргументах из $B$ остаются теми же, остальные отбрасываются) и своего сужения на $C$. Тем самым для каждого элемента множества $A^{B+C}$ мы получаем пару элементов из $A^B$ и $A^C$. Это и будет искомое взаимно однозначное соответствие.
	\end{proof}
\end{theorem}

\TODO{В книге док-в нет, надо самим писать :с}
\[(ab)^c = a^c × b^c\]
\[(a^b)^c = a^{b × c}\]


Свойства мощностей:
\[\aleph_0 + n = \aleph_0 \text{ для конечного } n \]
\[\aleph_0 + \aleph_0 = \aleph_0\]
\[\aleph_0 × \aleph_0 = \aleph_0\]

Какие-то красивые формулы, следующие из свойств операций:
\[\cont × \cont = 2^{\aleph_0} × 2^{\aleph_0} = 2^{\aleph_0 + \aleph_0} = 2^{\aleph_0} = \cont\]
\[\cont^{\aleph_0} = (2^{\aleph_0})^{\aleph_0} = 2^{\aleph_0×\aleph_0} = 2^{\aleph_0} = \cont\]


















