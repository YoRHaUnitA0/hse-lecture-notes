\Section{Исчисление высказываний}{Игорь Энгель}

\Subsection{Билет 29: <<Вывод. Доказательство. Исчисление высказываний>>}

Исчисление высказываний - формальная система, состоящия из $11$ схем аксиом и одного правила вывода.

Формулы исчисления высказываний (<<утверждения>>) состоят из формальных переменных а так-же связок $\land$ (И), $\lor$ (ИЛИ), $\neg$ (НЕ) и $\to$ (импликация, <<следует>>).

\begin{definition}[Аксиомы исчисления выссказываний] \thmslashn 
    \begin{enumerate}
        \item $A \to (B \to A)$
        \item $(A \to (B \to C)) \to ((A \to B) \to (A \to C))$
        \item $(A \land B) \to A$
        \item $(A \land B) \to B$
        \item $A \to (B \to (A \land B))$
        \item $A \to (A \lor B)$
        \item $B \to (A \lor B)$
        \item $(A \to C) \to ((B \to C) \to ((A \lor B) \to C))$
        \item $\neg A \to (A \to B)$
        \item $(A \to B) \to ((A \to \neg B) \to \neg A)$
        \item $A \lor \neg A$
    \end{enumerate}
\end{definition}

Это так называемые <<схемы аксиом>> - на месте $A, B, C$ могут стоять любые формулы исчисления высказываний. Аксиома всегда является теоремой.

\begin{definition}[Modus ponens (MP)] \thmslashn 

    Если $A$ и $A \to B$ - теоремы исчисления высказываний, то $B$ - теорема исчисления высказываний.
\end{definition}


\begin{definition}[Вывод] \thmslashn 

Выводом называется конечная последовательность формул, в которой каждая либо является аксиомой, либо следует из предыдущих по MP.

\end{definition}

\begin{definition}[Вывод из множества формул] \thmslashn 

    Пусть $\Gamma$ - множество из формул утверждения высказываний.

    Выводом из $\Gamma$ называется конечная последовательность формул, в которой каждая либо является аксиомой, либо принадлежит $\Gamma$, либо следует из предыдущих по MP.

\end{definition}

\begin{definition}[Доказательство] \thmslashn 

    Доказательство формулы $A$ - вывод, в котором $A$ является последней формулой.
\end{definition}
\Subsection{Билет 30: <<Теорема о корректности исчисления высказываний>>}
\begin{theorem} \thmslashn

    Любая теорема исчисления выссказываний - тавтология.
    \begin{proof} \thmslashn
    
        Не сложно проверить, что все аксиомы - тавтологии.

        Докажем корректность MP от противного:

        Знаем, что $A$ и $A \to B$ - тавтологии, предположим что $B$ - нет. Возьмём означивание на котором $B$ неверно. Тогда, $A$ верно а $B$ неверно, значит $A \to B$ неверно. Но $A \to B$ - тавтология, значчит оно верно. Противориче. Значит, $B$ - тавтология.

        Таким образом, все начальные утверждения - тавтологии, MP сохраняет свойство тавтологии, значит, все теоремы - тавтологии.
    \end{proof}
\end{theorem}
\Subsection{Билет 31: <<Теорема о полноте исчисления высказываний>>}

\Subsubsection{Схема доказательства}

\begin{itemize}
    \item \hyperref[logic_completeness_1]{Теорема о полноте}
        \begin{enumerate}
            \item Если $B$ выводится с добавлением либо $A$ либо не $\neg A$ (в обоих случаях), то $B$ \hyperref[logic_collapse]{выводится без дополнений}.
            \item Если булева формула истинна на наборе переменных, то она \hyperref[logic_truth_table]{выводится из истинности соответствующих литералов}
                \begin{enumerate}
                    \item $\{\land, \lor, \to, \neg\}$ \hyperref[logic_truth_table_base]{выводятся} из аргументов по таблице истинности
                        \begin{itemize}
                            \item \hyperref[logic_DT]{Лемма о дедукции} для работы с понятием <<выводится из>>
                            \begin{enumerate}
                                \item $D \to D$ - \hyperref[logic_ID]{ТИВ}
                                \item В одну сторону - добавляем формулу в вывод и MP
                                \item В другую - превращем каждую формулу $C_{i}$ в выводе в формулу $(A \to C_{i})$, и префиксируем её выводом (разбор случаев: $C_{i} = A$, $C_{i}\in \Gamma$, $C_{i}$ - аксиома и $C_{i}$ выводится из предыдущих (этот самый сложный, через аксиому 2)).
                            \end{enumerate}
                        \end{itemize}
                    \item Индукция по построению
                \end{enumerate}
        \end{enumerate}
\end{itemize}

\Subsubsection{Доказательство}

Докажем сначала несколько вспомогательных лемм:

\begin{lemma} \label{logic_ID} \thmslashn

    Для любой формулы $D$, $D \to D$ - ТИВ (теорема исчисления высказываний).
    
    \begin{proof} \thmslashn
    
        \begin{enumerate}
            \item Аксиома 2 ($A \to (B \to C) \to ((A \to B) \to (A \to C))$):\\ $\underbrace{D}_{A} \to (\underbrace{(D \to D)}_B \to \underbrace{D}_C) \to ((\underbrace{D}_A \to \underbrace{(D \to D)}_B) \to (\underbrace{D}_A \to \underbrace{D}_C))$
            \item Аксиома 1 ($A \to (B \to A)$):\\
                $\underbrace{D}_A \to (\underbrace{(D \to D)}_B \to \underbrace{D}_A)$
            \item MP от $1$ и $2$: $(D \to (D \to D)) \to (D \to D)$
            \item Аксиома 1: $A \to (B \to A)$:\\
                $\underbrace{D}_{A} \to (\underbrace{D}_{B} \to \underbrace{D}_{A} )$
            \item MP от $3$ и $4$: $D \to D$ \qedhere
        \end{enumerate}
    \end{proof}
\end{lemma}

\begin{lemma}[О дедукции (DT)] \label{logic_DT} \thmslashn

    Пусть $\Gamma \vdash A$ означает <<существует вывод формулы $A$ из множества $\Gamma$>>

    Тогда $\Gamma \vdash (A \to B) \iff \Gamma \cup \{A\} \vdash B $.
    \begin{proof} \thmslashn
    
        Необходимость ($\implies$):

        Если $\Gamma \vdash (A \to B)$, то и $\Gamma \cup \{A\} \vdash (A \to B) $. Тогда, возьмём вывод формулы $A \to B$ из $\Gamma$, добавим к нему утверждение $A$ (можем добавить, так-как входит в множество из котрого выводим), применим MP к последним двум утвержденям, получим $B$. Значит, существует вывод $B$.

        Достаточность ($\impliedby$):

        Возьмём вывод формулы $B$ из множества $\Gamma \cup \{A\}$, назовём его $C_{i}$, заметим, что $C_{n} = B$.

        Тогда, выводом формулы $A \to B$ из $\Gamma$ будет $B_1, (A \to C_1), B_2 (A \to  C_2), \ldots, B_{n}, (A \to C_{n})$. Тогда, последняя формула - $(A \to B)$.

        $B_{n}$ - такие последовательности, которые позволяют добавлять формулы $(A \to C_{n})$ в вывод. Разберём случаи:

        \begin{enumerate}
            \item $C_{i} = A$ : формула $A \to A$ выводима по предыдущей лемме, тогда $B_{i}$ - её вывод из аксиом (кроме последнего формулы, место котрой займёт сама формула $A \to A$).
            \item $C_{i}\in \Gamma$, либо $C_{i}$ - аксиома: через аксиому 1: $B_{i} = C_{i}, C_{i} \to (A \to C_{i})$. Тогда, $A \to C_{i}$ получается через MP.
            \item $C_{i}$ выводится из предыдущих по MP: тогда, в оригинальном выводе были формулы $C_{j}$, $C_{j} \to C_{i}$. Значит, в новом выводе будут формулы $A \to C_{j}$, $A \to (C_{j} \to C_{i})$. выведем из них $A \to C_{i}$:
                \begin{enumerate}
                    \item посылка 1: $A \to C_{j}$
                    \item посылка 2: $A \to (C_{i} \to C_{j})$
                    \item аксиома 2 ($(A \to (B \to C)) \to ((A \to B) \to (A \to C)$):\\
                        $(\underbrace{A}_{A} \to (\underbrace{C_{j}}_{B} \to \underbrace{C_{i}}_{C}) \to ((\underbrace{A}_{A} \to \underbrace{C_{j}}_{B}) \to (\underbrace{A}_{A} \to \underbrace{C_{i}}_{C}))$
                    \item MP от b и c: $(A \to C_{j}) \to (A \to C_{i})$
                    \item MP от a и d: $A \to C_{i}$.
                \end{enumerate}

                Тогда $B_{i} = a,b,c,d$.
        \end{enumerate}

        Получили вывод формулы $A \to B$ из множества $\Gamma$.
    \end{proof}
\end{lemma}
\begin{lemma} \label{logic_truth_table_base} \thmslashn

    Следующие утверждения - ТИВ для любых формул $P, Q$:

    \begin{enumerate}
        \item
            \begin{enumerate}
                \item $P, Q \vdash (P \land Q)$
                \item $\neg P, Q \vdash \neg (P \land Q)$
                \item $P, \neg Q \vdash \neg (P \land Q)$
                \item $\neg P, \neg Q \vdash \neg (P \land Q)$
            \end{enumerate}
        \item
            \begin{enumerate}
                \item $P, Q \vdash (P \lor Q)$
                \item $\neg P, Q \vdash (P \lor Q)$
                \item $P, \neg Q \vdash (P \lor Q)$
                \item $\neg P, \neg Q \vdash \neg (P \lor Q)$
            \end{enumerate}
        \item
            \begin{enumerate}
                \item $P, Q \vdash (P \to  Q)$
                \item $\neg P, Q \vdash (P \to  Q)$
                \item $P, \neg Q \vdash \neg (P \to Q)$
                \item $\neg P, \neg Q \vdash (P \to Q)$
            \end{enumerate}
        \item
            \begin{enumerate}
                \item $P \vdash \neg(\neg P)$
                \item  $\neg P \vdash \neg P$
            \end{enumerate}
    \end{enumerate}
    \begin{proof} \thmslashn
    
        \TODO{Надеюсь его не надо\ldots Тут просто куча тупых выводов.}
    \end{proof}
\end{lemma}
\begin{lemma} \label{logic_truth_table} \thmslashn

    Если формула $A$ интерпретированная как булева формула принимает истинное значение при истинности литералов $\ell_1, \ldots, \ell_{n}$ (если $x$ - формальная переменная, то литерал это либо $x$ либо $\neg x$), то верно что $\ell_1, \ldots, \ell_{n} \vdash A$, если-же она принимает ложное значение, то верно что $\ell_1, \ldots, \ell_{n} \vdash \neg A$
    \begin{proof} \thmslashn
    
       Доказательство по индукции. Пусть $\Gamma = \{\ell_1, \ldots, \ell_{n}\} $.

       Будем предпалогать что $A$ принимает истинное значение, для ложного доказательство симметрично (\TODO{или надо?}).

       Если $A$ состоит из одной формальной переменной $x_{i}$, то раз $A$ истинно, то $x_{i}$ истинно и присутствует в множестве формул. По лемме мы знаем, что $x_{i} \to x_{i}$, и, соответственно $x_{i} \vdash x_{i} \implies \Gamma \vdash x_{i}$.

       Если $A = \neg B$, то раз $A$ истинно, то $B$ ложно, значит верно $\Gamma \vdash \neg B$, но $\neg B = A$.

       Если $A = B \land C$, то раз $A$ истинно, то $B$ и $C$ истинны, значит $\Gamma \vdash B$ и  $\Gamma \vdash C$. 
       \begin{enumerate}
           \item Вывод 1: $B$
           \item Вывод 2: $C$
           \item Аксиома 5: $B \to C \to (B \land C)$
           \item MP 1 и 3: $C \to (B \land C)$
           \item MP 2 и 4: $B \land C$
       \end{enumerate}

        Если $A = B \lor C$, то раз $A$ истинно то либо $\Gamma \vdash B$ либо $\Gamma \vdash C$. В зависимости от случая, берём либо аксиому 6 либо аксиому 7 и применяем MP.
    \end{proof}
\end{lemma}
\begin{lemma} \label{logic_collapse} \thmslashn

    Для любых формул $A$ и $B$, если $\Gamma \cup \{A\} \vdash B$ и $\Gamma \cup \{\neg A\} \vdash B $, то $\Gamma \vdash B$.
    \begin{proof} \thmslashn

        Можем переформулировать теорему с помощью DT как $\{(A \to B), (\neg A \to B)\} \vdash B $ 
        \begin{enumerate}
            \item посылка 1: $A\to B$
            \item посылка 2: $\neg A \to B$
            \item аксиома 8 ($(A \to C) \to ((B \to C) \to ((A \lor B) \to C)$): \\
                $(\underbrace{A}_{A} \to \underbrace{B}_{C}) \to ((\underbrace{\neg A}_{B} \to \underbrace{B}_{C}) \to ((\underbrace{A}_{A} \lor \underbrace{\neg A}_{B} ) \to \underbrace{B}_{C} ) $
            \item MP от 1 и 3: $(\neg A \to B) \to ((A \lor \neg A) \to B)$
            \item MP от 2 и 4: $(A \lor \neg A) \to B$
            \item Аксиома 11: $A \lor \neg A$
            \item MP от 5 и 6: $B$ \qedhere
        \end{enumerate}
    \end{proof}
\end{lemma}
\begin{theorem} \label{logic_completeness_1} \thmslashn

    Любая тавтология - теорема исчисления высказываний.
    \begin{proof} \thmslashn

        Пусть $A$ - тавтология. Тогда она верна на всех наборах литералов.

        Возьмём наборы $x_1, \ldots, x_{n-1}, x_{n}$ и $x_1, \ldots, x_{n-1}, \neg x_{n}$. На обоих наборах формула истинна, а значит выводится из них. 

        Но раз она выводится из них, то она выводится из $x_1, \ldots, x_{n-1}$. Аналлогично, она выводится из $x_1, \ldots, \neg x_{n-1}$. Можем повторить $n$ раз, и получить $ \emptyset \vdash A$, значит $A$ - ТИВ.
    \end{proof}
\end{theorem}
\Subsection{Билет 32: <<теорема о корректности исчисления высказываний (вторая форма)>>}
\begin{definition} \thmslashn 

    Формула $A$ называется выполнимой, если сущесвует означивание переменных при котором формула $A$  истинна.
\end{definition}
\begin{consequence} \thmslashn

    Формула $A$ тавтология тогда и только тогда, когда $\neg A$ невыполнима. 
\end{consequence}
\begin{definition} \thmslashn 

    Множество формул $\Gamma$ называется совместным, если существует означивание переменных при котором все формулы в $\Gamma$ выполнены одновременно
\end{definition}
\begin{definition} \thmslashn 

    Множество формул $\Gamma$ называется противоречивым если верно $\Gamma \vdash A$ и $\Gamma \vdash \neg A$.
\end{definition}
\begin{consequence} \thmslashn

    Если $\Gamma$ - противоречивое множество, то для любой формулы $B$ верно $\Gamma \vdash B$
    \begin{proof} \thmslashn
    
        \begin{enumerate}
            \item Посылка 1: $A$
            \item Посылка 2: $\neg A$
            \item Аксиома 9: $\neg A \to (A \to B)$
            \item MP от 2 и 3: $A \to B$
            \item MP от 1 и 4: $B$ \qedhere
        \end{enumerate}
    \end{proof}
\end{consequence}
\begin{theorem}[Корректность исчисления высказываний (вторая форма)] \thmslashn

    Любое совместное множество формул непротиворечиво

    \begin{proof} \thmslashn
    
        Пусть $\Gamma$ совместное. Предположим что оно противоречиво.

        Тогда $\Gamma \vdash A$ и $\Gamma \vdash \neg A$.

        Если все формулы из $\Gamma$ выполняются на каком-то означивании, то и все выводимы из $\Gamma$ формулы выполняются на этом означивании (доказывается индукцией по построению формулы) (\TODO{у меня ощущение что на экзе за такое убьют, но в книге ничё больше не написано\ldots}).

        Значит, должны выполняться одновременно $A$ и $\neg A$. Но такое невозможно, значит $\Gamma$ непротивочво.
    \end{proof}
\end{theorem}
\begin{consequence} \thmslashn

    Любая теорема исчисления высказыванний - тавтология.
    \begin{proof} \thmslashn
    
        Пусть $A$ - ТИВ. Тогда $\{\neg A\} \vdash A $. Значит, $\{\neg A\}$ противоречиво, а значит несовместно. Значит $A$ - тавтология.
    \end{proof}
\end{consequence}
\Subsection{Билет 33: <<теорема о полноте исчисления высказываний (вторая форма)>>}

\begin{definition} \thmslashn 

    Пусть $V(\Gamma)$ - множество всех формальных переменных участвующих в формулах $\Gamma$.

    {\color{red}\bf Предполагаем что $V(\Gamma)$ не более чем счётно.}
\end{definition}

\begin{definition} \thmslashn 

    Непротиворечивое множество $\Gamma$ называется полным, если для любой формулы $F$ над переменными из $V(\Gamma)$ верно либо $\Gamma \vdash F$ либо $\Gamma \vdash \neg F$
\end{definition}
\begin{lemma} \thmslashn

    Если $\Gamma$ непротиворечиво, а $\Gamma \cup \{A\}$ противоречиво, то $\Gamma \vdash \neg A$
    \begin{proof} \thmslashn
    
        \TODO{Книга утверждает что трвивально\ldots}
    \end{proof}
\end{lemma}
\begin{lemma} \thmslashn

    Если $\Gamma$ непротиворечиво, то либо $\Gamma \cup \{A\} $ либо $\Gamma \cup \{\neg a\} $ непротиворечиво.
    \begin{proof} \thmslashn
    
        Прдеположим что оба противоречивы. Тогда $\Gamma \vdash \neg A$ и $\Gamma \vdash \neg (\neg A)$. Но $\Gamma$ непротиворечиво.
    \end{proof}
\end{lemma}
\begin{lemma} \thmslashn

    Для любого непротиворечивого множества $\Gamma$ существует такое непротиворечивое полное множество $\Delta$, что $\Gamma \subset \Delta$.
    \begin{proof} \thmslashn
    
        $V(\Gamma)$ конечно или счётно, значит формул над $V(\Gamma)$ счётно. Пронумеруем их.

        Будем последовательно добавлять либо $F$ либо $\neg F$ к $\Gamma$, в зависимости от того, что будет непротиворечивым.

        В результате бесконечного процесса получим полное множество.

        Заметим, что оно непротиворечиво: Предположим что не так, $\Gamma \cup X \vdash A$, $\Gamma \cup X \vdash \neg A$. Но так-как вывод $A$ и $\neg A$ содержит только конечно число членов, можно выбрать такое конечное $n$, что после добавления  $n$ формул, оба этих вывода возможны. Но тогда противоречиво какое-то конечное количество шагов процесса, сохраняющего непротиворечивость на каждом шаге.
    \end{proof}
\end{lemma}
\begin{lemma} \thmslashn

    Для любого полного непротиворечивого множества $\Delta$ существует означивание переменных из $V(\Delta)$, при котором все формулы из $\Delta$ истинны.
    \begin{proof} \thmslashn
    
        Для любой переменной $p\in V(\Delta)$, либо $\Delta \vdash p$ либо  $\Delta \vdash \neg p$. 

        Пусть $P = \{p \ssep p\in V(\Delta), \quad \Delta \vdash p\} \cup \{\neg p \ssep p\in V(\Delta), \quad \Delta \vdash \neg p\}$.

        $P$ соответсвует нужному означиванию. При этом, $\Delta \vdash P$. Значит, если $\Delta \cup P$ противоречиво, что $\Delta$ противоречиво, что невозможно. Значит, при означивании $P$ выполняются все формулы в $\Delta$.
    \end{proof}
\end{lemma}
\begin{theorem}[Полнота исчисления высказываний (вторая форма)] \thmslashn

    Любое непротиворечивое множество совместно.

    \begin{proof} \thmslashn
    
        Пусть $\Gamma$ - непротиворечивое множество. Возьмём полное непротиворечивое множество $\Delta$, такое, что $\Gamma \subset \Delta$. Оно совместно, значит и $\Gamma$ совместно.
    \end{proof}
\end{theorem}
\begin{consequence} \thmslashn

    Любая тавтология - ТИВ.
    \begin{proof} \thmslashn
    
        Пусть $A$ - тавтология. Тогда $\{\neg A\} $ несовместно. Значит оно противоречиво. Значит, $ \emptyset \vdash \neg \neg A$, значит $ \emptyset \vdash A$.
    \end{proof}
\end{consequence}
\Subsection{Билет 34 <<Теорема о компактности. Прмимер с бесконечным двудольным графом.>>}

\begin{theorem}[Теорема о компактности исчисления выссказываний] \thmslashn

    Пусть $\Gamma$ - множество формул, каждое конечное подмножество совместно. Тогда $\Gamma$ совместно.
    \begin{proof} \thmslashn
    
        $\Gamma$ совместно если оно непротиворечиво.
        
        Предположим что $\Gamma$ противоречиво. 
        
        Рассмотрим выводы формул $A$ и $\neg A$.
    
        Они содержат лишь конечное число формул, значит существует противоречивое конечное подмножество. Но оно не может быть совместным. Противоречие.
    \end{proof}
\end{theorem}
\begin{theorem} \thmslashn

    Бесконечный граф двудолен тогда и только тогда, когда каждый конечный подграф двудолен.
    \begin{proof} \thmslashn
    
       Необходимость очевидна: Если есть недвудольный подграф, то в нём есть нечётный цикл, и этот цикл есть в самом графе.

        Достаточность:

        Сопоставим каждой вершине графа формальную переменную. Каждому ребру $i \to j$ сопоставим утверждение $(\neg x_{i} \lor x_{j}) \land (x_{i} \lor \neg x_{j})$. 

        Множество таких утверждений совместно, если граф двудолен. По теореме о компактности, достаточно проверить только для конечных множеств формул.

        Для каждого подмножества $\Gamma'$ возьмём граф содерижащий рёбра соответствующие этим утверждениям, и все инцидентные им вершины. Этот подграф конечем, значит двудолен, значит $\Gamma'$ совместно.
    \end{proof}
\end{theorem}
