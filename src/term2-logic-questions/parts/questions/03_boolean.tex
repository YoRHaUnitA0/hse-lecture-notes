\Section{Булева логика}{Игорь Энгель}
\Subsection{Билет 18: <<Высказывания и операции. Тавтологии.>>}
\begin{definition} \thmslashn 

    Определим множество <<пропозициональных формул>> (высказывинй) следующим обарзом:

    \begin{itemize}
        \item <<пропозициональная переменная>> является высказыванием
        \item Если $A$ - высказывание, то $\neg A$ (НЕ $A$) - высказывание.
        \item Если $A$ и $B$ - высказывания, то $A \land B$ ($A$ И $B$), $A \lor B$ ($A$ ИЛИ $B$), $A \to B$ (из $A$ следует $B$) - высказывания.
    \end{itemize}
\end{definition}
\begin{definition} \thmslashn 

    Пусть высказывание $A$ содержит пропозиональные переменные $x_1, \ldots, x_{n}$.

    Соответствующий высказыванию булевой функцией называется функция $\phi_{A} : \mathbb{B}^{n} \mapsto \mathbb{B}$, где $\mathbb{B} = \{0, 1\} = \mathbb{Z}/2$, заданная индуктивно следуюзим образом:

    \begin{equation*}
        \begin{array}{|c|c|c|c|c|c|} \hline
            A & B & A \land B & A \lor B & A \to B & \neg A \\ \hline
            1 & 1 & 1 & 1 & 0 & 0\\ \hline
            1 & 0 & 0 & 1 & 0 & 0 \\ \hline
            0 & 1 & 0 & 1 & 1 & 1 \\ \hline
            0 & 0 & 0 & 0 & 1 & 1\\ \hline
        \end{array}
    \end{equation*}


\end{definition}

\begin{definition} \thmslashn 
    
    Тавтологией называется высказывание, соответветсвующия которому формула принимает значение $1$ на всех возможных входах. 

\end{definition}

\begin{example}[Примеры тавтологий] \thmslashn

    \TODO{Надо.}
\end{example}

\Subsection{Билет 19: <<Выразимость любой формулы в КНФ и ДНФ>>}

\begin{definition} \thmslashn 

    Формула находится в конъюнктивной нормальной форме (КНФ) если она имеет вид 
    \[ \bigwedge_{i\in \{1, \ldots, n\} } \left(\bigvee_{j\in \{1, \ldots, m_{i}\}} \ell_{k_{ij}}\right) .\]

    Формула находится в дизъюнктивной нормальной форме (ДНФ) если она имеет вид
    \[ \bigvee_{i\in \{1, \ldots, n\} } \left(\bigwedge_{j\in \{1, \ldots, m_{i}\}} \ell_{k_{ij}}\right) .\]

    Дизъюнктом называется формула вида
    \[ \bigvee_{j\in \{1, \ldots, m\} } \ell_{j} .\] 

    Конъюнктом:
    \[ \bigwedge_{j\in \{1, \ldots, m\} } \ell_{j} .\] 

    где $\ell_{i}$ называется литералом, и имеет вид либо $x_{i}$ либо $\neg x_{i}$.
        
\end{definition}

\begin{example} \thmslashn

    Пример КНФ: $(x_1 \lor x_2 \lor \neg x_5) \land (x_3 \lor x_4 \lor x_5)$

    Пример ДНФ: $(x_1 \land x_4) \lor (x_2 \land x_4) \lor \neg x_3$
\end{example}

\begin{theorem} \thmslashn

    Любую булеву функцию можно записать в ДНФ.

    \begin{proof} \thmslashn
    
        Возьмём все наборы переменных на которых функция принимает значение $1$.

        Каждому такому набору сопоставим конъюнкт в который входят все переменные, причём, если во входном наборе переменная имеет значение $1$, то она входит как $x_{i}$, если имеет значение $0$, то как $\neg x_{i}$.

        Очевидно, что каждый конъюнкт примет значение $1$ только на одном входе, и функция примет значение $1$ если хотя-бы один конъюнкт принял значение $1$.

    \end{proof}
\end{theorem}

\begin{theorem} \thmslashn

    Любую булеву функцию можно записать в КНФ

    \begin{proof} \thmslashn
    
        Возьмём все наборы переменных на которых функция принимает значение $0$.

        Каждому такому набору сопоставим дизъюнкт в который переменная $x_{i}$ входит как литерал $x_{i}$ если $x_{i}$ имеет значение $0$ и $\neg x_{i}$ когда $x_{i}$ имеет значение $1$.

        Заметим, что такой дизъюнкт выполняется только тогда, кодга строка не совпадает с той, которой он соответствует.

        Значит, все дизъюнкты будут выполнены тогда и только тогда, когда функция не принимает значение $0$, тоесть, принимает значение $1$.
    \end{proof}
\end{theorem}
\Subsection{Билет 20: <<Полиномы Жегалкина>>}

\begin{definition} \thmslashn 

    Моном - формула вида
    \[ 1 \land \left( \bigwedge_{i\in I} x_{i}\right)  .\]
\end{definition}

\begin{example} \thmslashn

    Примеры мономов: $1$, $x_1$, $x_1x_3$.
\end{example}

\begin{definition} \thmslashn 

    Полином Жегалкина - XOR (сумма в $\mathbb{Z}/2$) мономов.

    \begin{equation*}
        \begin{array}{|c|c|c|} \hline
            a & b & a \oplus b\\ \hline
            1 & 1 & 0\\ \hline
            1 & 0 & 1 \\ \hline
            0 & 1 & 1 \\ \hline
            0 & 0 & 0 \\ \hline
        \end{array}
    \end{equation*}
\end{definition}
\begin{example} \thmslashn

    Примеры полиномов Жегалкина:

    \[ 1 \oplus x_1 \oplus x_1x_2 .\] 

    \[ x_1 \oplus x_2 \oplus x_1x_2 .\] 
\end{example}

\begin{theorem} \thmslashn

    Любую булеву функцию можно однозначно записать полиномом жегалкина.

    \begin{proof} \thmslashn
    
        Докажем существование подходящего полинома:

        Выразим основные связки:
        \begin{equation*}
            \begin{split} 
                \neg x &= 1 \oplus x\\
                x_1 \land x_2 &= x_1 \land x_2 \text{ (моном)}\\
                x_1 \lor x_2 &= x_1 \oplus x_2 \oplus x_1x_2\\
            \end{split}
        \end{equation*}

        Теперь, запишем формулу в ДНФ, раскоре $\lor$, уберём повторяющиеся члены (ессли один член встречается в мономе дважды, то второе вхождение ни на что не влияет и надо оставить одно, если один моном встречается дважды, то он отменяет себя в сложении по модулю $2$, и надо убрать оба вхождения).

        Докажем единственность:

        Всего существует $|\mathbb{B}|^{|\mathbb{B}^{n}|} = 2^{2^{n}}$ булевых функций от $n$ переменных.

        Заметим, что существует $2^{n}$ различных мономов - каждый моном либо включает либо не включает одну из $n$ переменных.

        Значит, всего существует $2^{2^{n}}$ различных многочленов Жегалкина от $n$ переменных. По принципу Дирихле, каждой функции соответствует ровно один многочлен, так-как существование хотя-бы одного уже доказано.


    \end{proof}
\end{theorem}

\Subsection{Билет 21: <<Критерий Поста>>}

\begin{definition} \thmslashn 

    Булева функция $f$ называется сохраняющей $0$, если
    \[ f(0, \ldots, 0) = 0 .\]

    Обозначим множество таких функций $T_0$.
\end{definition}
\begin{definition} \thmslashn 

    Булева функция $f$ называется сохраняющей $1$, если
    \[ f(1, \ldots, 1) = 1 .\]

    Обозначим множество таких функций $T_1$.
\end{definition}
\begin{definition} \thmslashn 

    Булева функция $f$ называется самодвойственной, если
    \[ f(\neg x_1, \neg x_2, \ldots, \neg x_n) = \neg f(x_1, x_2, \ldots, x_{n}) .\]

    Обозначим множество таких функций $S$.
\end{definition}
\begin{definition} \thmslashn 

    Булева функция $f$ называется монотонной, если
    \[ f(x_1, \ldots, 0, \ldots, x_{n}) = 1 \implies f(x_1, \ldots, 1, \ldots, x_{n}) .\]

    (замена $0$ на $1$ не может изменить результат с $1$ на $0$).

    Обозначим множество таких функций $M$.
\end{definition}
\begin{definition} \thmslashn 

    Булева функция $f$ называется линейной, если в её полиноме Жегалкина все мономы имеют не более одной переменной.

    Обозначим множество таких функций $L$.
\end{definition}
\begin{definition} \thmslashn 

    Система связок называется полной, если с её помощью можно выразить любую функцию.
\end{definition}
\begin{lemma} \thmslashn

    $\{\neg, \land, \lor\} $ - полная система связок.
    \begin{proof} \thmslashn
    
        Можно построить ДНФ.
    \end{proof}
\end{lemma}
\begin{theorem}[Критерий Поста] \thmslashn

    Система связок $B$ полная тогда и только тогда, когда в ней для каждого из вышеперечисленных классов есть хотя-бы одна функция не входящая в него:
    
    \begin{equation*}
        \begin{split}
            \exists{f\in B}\quad& f \not\in T_0\\
            \exists{g\in B}\quad& h \not\in T_1\\
            \exists{h\in B}\quad& h \not\in S\\
            \exists{k\in B}\quad& k \not\in M\\
            \exists{r\in B}\quad& r \not\in L
        \end{split}
    \end{equation*}
    \begin{proof} \thmslashn
    
        Рассмотрим функцию $f$. Если $f\in T_1$, то $f(x, \ldots, x) = 1$, иначе $f(x, \ldots, x) = \neg x$.

        Рассмотрим функцию $g$. Если $g\in T_0$, то $g(x, \ldots, x) = 0$, инчае $g(x, \ldots, x) = \neg x$.

        Мы либо получили $\neg$, либо получили $\{1, 0\}$.

        Получим $0$ или $1$ из $\neg$:

        Возьмём функцию $h$. Существует такой набор входов $\eps_{i}$, что
        \[ h(\eps_1, \ldots, \eps_{n}) = h(\neg \eps_1, \ldots, \neg \eps_{n}) .\]

        Тогда $h(\eps_1(x), \ldots, \eps_{n}(x) = h(\eps_1(\neg x), \ldots, \eps_{n}(\neg x))$, где $\eps_{i}(x)$ - $x$ если $\eps_{i} = 1$, иначе $\neg x$.

        Тогда такая функция будет константной, другую константу можно получить применив $\neg$.

        Получим $\neg$ из $\{0, 1\} $:

        Возьмём функцию $k$. Существует такой набор уходов $\eps_{i}$, что
        \[ k(\eps_1, \ldots, \eps_{i-1}, 0, \eps_{i+1}, \ldots, \eps_{n}) = 1 .\] 
        \[ k(\eps_1, \ldots, \eps_{i-1}, 1, \eps_{i+1}, \ldots, \eps_{n}) = 0 .\]

        Тогда $k(\eps_1, \ldots, \eps_{i-1}, x, \eps_{i+1}, \eps_{n}) = \neg x$.

        Теперь точно есть $\{\neg, 0, 1\} $.

        У функции $r$ есть хотя-бы один член состоящий из конъюнкции хотя-бы двух переменных. Пусть, без ограничения общности, в нём присутствуют переменные $x_1$, и $x_2$. Тогда

        \[ r(x_1, x_2, 1, \ldots, 1) = x_1x_2[\oplus x_1][\oplus x_2][\oplus 1] .\]

        Члены в квадратных скобках могут присутствтовать или отсутствовать в зависимости от формулы.

        Если присутствует член $\oplus 1$, его можно убрать применив к результату $\neg$.

        \[ x_1x_2 = x_1 \land x_2 .\]
        \[ x_1x_2\oplus x_1 = x_1 \land \neg x_2 .\] 
        \[ x_1x_2\oplus x_2 = \neg x_1 \land x_2 .\] 
        \[ x_1x_2\oplus x_1\oplus x_2 = x_1 \lor x_2 = \neg \left( \neg x_1 \land \neg x_2 \right)  .\]

        Заметим, что применяя $\neg$ можно из любого варианта получить $x_1 \land x_2$. Значит, мы выразили $\{\neg, \land\}$, из этого можно по правилам Де-Моргана выразить $\lor$, значит мы получили полную систему связок. 
    \end{proof}
\end{theorem}
