% !TEX encoding = UTF-8 Unicode
\documentclass[11pt, oneside]{article}   	% use "amsart" instead of "article" for AMSLaTeX format
\usepackage{amssymb}
\usepackage{amsmath}
\usepackage{cRussian}
\usepackage{cPicture}
\usepackage{cTheorem}
%\usepackage{cFonts}
%\usepackage{cTikz}
\title{Мат. Анализ 3}
\author{Igor Engel}
\date{}

\begin{document}
\maketitle
\section{Предел последовательности}
    
\begin{theorem}[Теорема об арифметических действиях с пределами (продолжение)]
    Пусть $\lim x_n=a$ и  $\lim y_n = b$. Тогда:
    \[ \lim \left( x_n+y_n \right) =a+b .\]
    \[ \lim \left( x_ny_n \right) =ab .\]
    \[ \lim\left( x_n-y_n \right)=a-b  .\]
    \[ \lim \frac{x_n}{y_n} = \frac{a}{b}\text{, если $b\neq 0$} .\]
    \[\lim |x_n|=|a| .\] 
    \[ y_n \neq 0\land b\neq 0 \implies \lim\left( \frac{x_n}{y_n} \right) = \frac{a}{b} .\]
    \begin{proof}
        \[ \lim (x_n-y_n) = \lim(x_n+(-1)\cdot y_n) = a+(-1)b=a-b .\]
        Для доказательсвта отношения достаточно доказать, что $\lim \frac{1}{y_n} = \frac{1}{b}$
        \[ \left|\frac{1}{y_n}-\frac{1}{b}\right|=\frac{|y_n-b|}{|y_n| |b|} .\]
        Подставим $\varepsilon = \frac{|b|}{2}$, тогда $\exists{N}\quad \forall{n > N}\quad |y_n-b|<\frac{|b|}{2}$.\\
        Тогда, $|y_n| > \frac{|b|}{2}$. Тогда
        \[ \frac{|y_n-b|}{|y_n| |b|} < \frac{|y_n-b|}{|b| / 2\cdot |b|} = \frac{2|y_n-b|}{b^2} < \varepsilon'.\]
        Подставим $\varepsilon'=\frac{b^2\varepsilon}{2}$, тогда найтётся $ \tilde{N}$ удовлетворяющие условию предела.\\
        Докажем про модуль. 
        \[ | |x_n|-a| \le |x_n-a| \iff -|x_n-a| \le |x_n|-|a|\le |x_n-a|.\]
        \[ |x_n|=|(x_n-a)+a| \le |x_n-a|+|a| .\]
        \[ |a| \le  |x_n|+|x_n-a|\text{, симметрично} .\]
        \[ \lim x_n=a \implies \lim (x_n-a) = 0 \implies \lim |x_n-a| = 0 \implies \lim |x_n|-|a| = 0 \implies \lim |x_n| = |a| .\qedhere\] 
    \end{proof} 
\end{theorem}
\begin{example}
\[ \lim \frac{n^2-3n+5}{3n^2+4n-6}=\lim \frac{1-\frac{3}{n}+\frac{5}{n^2}}{3+\frac{4}{n}-\frac{6}{n^2}} .\]
\[ \frac{\lim (1-\frac{3}{n}+\frac{5}{n^2})}{\lim (3+\frac{4}{n}-\frac{6}{n^2})}.\]
\[ \frac{1-\lim \frac{3}{n} + \lim \frac{5}{n^2}}{3+\lim \frac{4}{n} + \lim \frac{6}{n^2}} = \frac{1}{3} .\] 
    
\end{example}
\begin{definition}
    Последовательность $x_n$ монотонно возрастает, если $\forall{n}\quad x_n \le x_{n+1}$
\end{definition}
\begin{definition}
    Последовательность $x_n$ монотонно убывает, если $\forall{n}\quad x_n \ge x_{n+1}$
\end{definition}
\begin{definition}
    Последовательность $x_n$ монотонна, если она монотонно возрастает или монотонно убывает
\end{definition}
\begin{theorem}
    \begin{enumerate}
        \item Если последовательность монотонно возрастает и ограниченна сверху, то она имеет предел
        \item Если последовательность монотонно убывает и ограниченна снизу, то она имеет предел
        \item Пусть последовательность монотонна, тогда она имеет предел тогда и только тогда когда она ограниченна.
    \end{enumerate}
    \begin{proof}
        Пусть $\{x_n\} $ - ограниченное множество, значит у него есть супремум. Пусть  $\sup \{x_n\} = b$. Докажем что $\lim x_n = b$.\\
        Возьмём  $\varepsilon$. $b-\varepsilon$ не является верхней гранью множества, занчит $\exists{N}\quad x_N>b-\varepsilon$
        \[ \forall{n>N}\quad b\ge x_n>b-\varepsilon \implies b-\varepsilon < x_n < b+\varepsilon \iff \lim x_n = b .\qedhere\] 
    \end{proof}
    Второй пункт доказывается симметрично.
    \begin{proof}
        Необходимость: Если последовательность имеет предел, она ограниченна.\\
        Достаточность следует из предыдущих пунктов\\
    \end{proof}
\end{theorem}
\section{Бесконечные пределы}
    \begin{definition}
        $\lim x_n =+\infty$ - вне любого луча $(E; \infty)$ находится лишь конечное число членов последовательности.
        \[ \forall{E}\quad \exists{N>0}\quad \forall{n>N}\quad x_n > E .\] 
        $\lim x_n=-\infty$ - вне любого луча $(-\infty; E)$ находится лишь конечное число членов последовательности.
        \[ \forall{E}\quad \exists{N>0}\quad \forall{n>N}\quad x_n<E .\] 
        $\lim x_n = \infty$ - вне любой пары лучей $(-\infty; -E)\cup (E; \infty)$ находится лишь конечное число членов последовательности
        \[ \forall{E}\quad \exists{N>0}\quad \forall{n>N}\quad |x_n|>E .\] 
    \end{definition}
    \begin{dlemma}
        Если $\lim x_n = +\infty$, или $\lim x_n = -\infty$, то $\lim x_n = \infty$\\
        Обратное неверно.
    \end{dlemma}
    \begin{dlemma}
        \[ \lim x_n = \infty \iff \lim |x_n| = +\infty  .\] 
    \end{dlemma}
    \begin{definition}
        $x_n$ - бесконечно большая, если  $\lim x_n = \infty$.
    \end{definition}
    \begin{theorem}
        Если  $x \neq 0$, то $x_n$ - бесконечно большая тогда, и только тогда, когда  $\frac{1}{x_n}$ - бесконечно малая.
        \begin{proof}
            Необходимость:\\
            Возьмём $\varepsilon$, и подставим  $E = \frac{1}{\varepsilon}$ в $\lim x_n = \infty$.\\
            \[\exists{N}\quad \forall{n>N}\quad |x_n|>E=\frac{1}{\varepsilon} \iff \left|\frac{1}{x_n}\right| < \varepsilon .\]
            Достаточность:\\
            Возьмём $E>0$, подставим  $\varepsilon=\frac{1}{E}$ в $\lim \frac{1}{x} = 0$ :
            \[ \exists{N}\quad \forall{n>N}\quad \left|\frac{1}{x}\right|<\varepsilon=\frac{1}{E} \iff  |x_n|>E .\qedhere\] 
        \end{proof}
    \end{theorem}
\section{Число $e$}
\begin{theorem}[Неравенство Бернулли]
    \[ n\in \mathbb{N}, x>-1 .\]
    \[ (1+x)^{n}\ge 1+nx .\]
    \[ (1+x)^{n}=1+nx \implies x=0\lor n=1 .\] 
    \begin{proof}
        Для $n=1$ утверждение очевидно.\\
        \[ (1+x)(1+x)^{n} \ge (1+x)(1+nx) .\]
        \[ (1+x)(1+nx)= 1+x+nx+nx^2 \ge 1+(n+1)x . \qedhere\] 
    \end{proof}
\end{theorem}
\begin{tlemma}
    Если $t>1$, то  $\lim t^{n} = +\infty$ 
    \begin{proof}
        \[ t = 1+x\text{, где $x>0$} .\]
        \[ t^{n} = (1+x)^{n} \ge 1+nx>nx>E .\]
        \[ n>\frac{E}{x} .\] 
    \end{proof}
\end{tlemma}
\begin{tlemma}
    Если $|t|<1$, то  $\lim t^{n} = 0$ 
    \[ \frac{1}{t} = 1+x, x>0 .\]
    \[ \left|\frac{1}{t^{n}}\right|=(1+x)^{n}>1+nx>nx .\]
    \[ \left|t^n\right| < \frac{1}{nx}<\varepsilon .\]
    \[ N > \frac{1}{\varepsilon x} .\] 
\end{tlemma}
Рассмотрим последовательности $x_n = \left(1+\frac{1}{n}\right)^{n} < y_n = (1+\frac{1}{n})^{n+1}$ 
\begin{theorem}
    $x_n$ строго монотонно возрастает, а  $y_n$ строго монотонно убывает. 
    \begin{proof}
        \begin{equation*}
            \begin{split}
                \frac{y_{n-1}}{y_n} &= \frac{(1+\frac{1}{n-1})^{n}}{(1+\frac{1}{n})^{n+1}}\\
                &= \frac{\frac{n^{n}}{(n-1)^{n}}}{\frac{(n+1)^{n+1}}{n^{n+1}}}\\
                &= \frac{n^{2n+1}}{(n-1)^{n}(n+1)^{n+1}}\\ 
                &= \frac{n^{2(n+1)}}{(n^2-1)^{n+1}} \cdot \frac{n-1}{n}\\ 
                &= \left(1+\frac{1}{n^2-1}\right)^{n+1}\cdot \frac{n-1}{n}\\
                &> \frac{n-1}{n}\cdot \frac{n}{n-1} = 1 \qedhere 
            \end{split}
        \end{equation*}
    \end{proof} 
    \begin{proof}
        \begin{equation*}
            \begin{split}
                \frac{x_n}{x_{n-1}} &= \frac{\left( 1+\frac{1}{n} \right)^{n} }{\left( 1+\frac{1}{n-1} \right)^{n-1} }\\
                                    &= \frac{\left( n^2-1 \right)^{n} }{n^{2n}} \cdot  \frac{n}{n-1}\\
                                    &= \left( 1-\frac{1}{n^2} \right)^{n} \cdot  \frac{n}{n-1}\\ 
                                    &> \left( 1-\frac{1}{n} \right) \cdot \frac{n}{n-1}\\
                                    &= \frac{n-1}{n}\cdot \frac{n}{n-1} = 1
            \end{split}
        \end{equation*}
        
    \end{proof}
    $x_n$ - ограничена сверху, значит у неё есть предел. $\lim x_n = e$.
\end{theorem}
\begin{tlemma}
    $x_n < e$, а  $y_n>e$.
\end{tlemma}
\begin{tlemma}
    \[ 2 = x_1 < e < y_5 < 3 .\] 
\end{tlemma}
\begin{dlemma}
    \[ y_n - x_n = \left(1+\frac{1}{n}\right)x_n - x_n = \frac{x_n}{n} > \frac{2}{n} .\] 
\end{dlemma}
\begin{theorem}
    Пусть $x_n>0$ и  $\lim \frac{x_{n+1}}{x_n} < 1$, тогда $\lim x_n = 0$
     \begin{proof}
         \[ q = \lim \frac{x_{n+1}}{x_n} .\] 
         Подставим $\varepsilon = \frac{q+1}{2}$. Тогда,  $\exists{N}\quad \forall{n>N}\quad \frac{x_{n+1}}{x_n} < \varepsilon$.\\
         Возьмём $n\ge N$: $x_n = \frac{x_n}{x_{n-1}}\cdot \frac{x_{n-1}}{x_{n-2}} \cdot \ldots \cdot \frac{x_{N-1}}{x_N} \cdot  x_N < \left( \frac{q+1}{2} \right)^{n-N}x_N = \left(\frac{q+1}{2}\right)^{n}\frac{x_n}{c} \iff \lim x_n = 0 $
    \end{proof}
\end{theorem}
\end{document} 
