% !TEX encoding = UTF-8 Unicode
\documentclass[11pt, oneside]{article}   	% use "amsart" instead of "article" for AMSLaTeX format
\usepackage{amssymb}
\usepackage{amsmath}
\usepackage{cRussian}
\usepackage{cPicture}
\usepackage{cTheorem}
%\usepackage{cTikz}
\title{Мат. Анализ 11}
\author{Igor Engel}
\date{}

\begin{document}
\maketitle
\section{}
\begin{definition}[Обозначеия внутри конспекта]
    
    \[ m_{\lambda}(a,b) := \lambda a + (1-\lambda) b.\]
    \[m_{\lambda}^{f}(a,b) := m_{\lambda}(f(a), f(b)).\]
    Заметим, что 
    \begin{equation*}
        \begin{split}
            m_{\lambda}(a,b) + m_{\lambda}(c,d) 
            &= \lambda a + (1-\lambda)b + \lambda c + (1-\lambda) d\\
            &= \lambda(a+c) + (1-\lambda)(b+d) = m_{\lambda}(a+c, b+d)
        \end{split}
    \end{equation*}
    \begin{equation*}
        \begin{split}
            \alpha m_{\lambda}(a,b)
            &= \alpha(\lambda a) + \alpha((1-\lambda)b)\\
            &= \lambda(\alpha a) + (1-\lambda)(\alpha b) = m_{\lambda}(\alpha a, \alpha b)
        \end{split}
    \end{equation*}
    \begin{equation*}
        \begin{split}
            m_{\lambda}^{f}(a,b) + m_{\lambda}^{g}(a,b)
            &= m_{\lambda}(f(a), f(b)) + m_{\lambda}(g(a), g(b))\\ 
            &= m_{\lambda}(f(a)+g(a), f(b)+g(b)) = m_{\lambda}^{f+g}(a,b) 
        \end{split}
    \end{equation*}
    \begin{equation*}
        \begin{split}
            \alpha m_{\lambda}^{f}(a, b) 
            &= \alpha m_{\lambda}(f(a), f(b))\\
            &= m_{\lambda}(\alpha f(a), \alpha f(b)) = m_{\lambda}^{\alpha f}(a, b)
        \end{split}
    \end{equation*}
\end{definition}
\begin{theorem}[Переформулировка выпуклости]
    $f : \left<a, b\right> \mapsto \mathbb{R}$.
    \[ f \text{ выпуклая } \iff \forall{x < z < y\in \left<a, b\right>}\quad (y-x)f(z) \le (y-z)f(x)+(z-x)f(y)   .\] 
    \begin{proof}
        \[ f(m_{\lambda}(x, y)) \le  m_{\lambda}^{f}(x, y) .\]
        \[ z = m_{\lambda}(x,y) = \lambda x + (1-\lambda)y  .\]
        \[ \lambda = \frac{y-z}{y-x} .\]
        \[ 1-\lambda = \frac{z-x}{y-x} .\]
        \[ f(z) \le \frac{y-z}{y-x}f(x) + \frac{z-x}{y-x}f(y) .\]
        \[ (y-x)f(z) \le (y-z)f(x) + (z-x)f(y) .\qedhere\]
    \end{proof}
\end{theorem}
\begin{theorem}[Ариф. Действия с выпуклыми функция]
    $f, g: \left<a, b\right> \mapsto \mathbb{R}$ -   выпуклые функции.\\
    $f+g$ -  выпуклая.\\
    $\forall{\alpha > 0}\quad \alpha f$ - выпуклая.
    \begin{proof}
        \begin{equation*}
            \forall{x, y\in \left<a, b\right>}\quad \forall{\lambda\in (0, 1)}\quad \begin{cases}
                f(m_{\lambda}(x,y)) \le m_{\lambda}^{f}(x, y)\\
                g(m_{\lambda}(x,y)) \le m_{\lambda}^{g}(x, y)
            \end{cases}
        \end{equation*}
        \begin{equation*}
            \begin{split}        
                \forall{x, y\in \left<a,b\right>}\quad \forall{\lambda\in (0,1)}\quad  &f(m_{\lambda}(x,y)) + g(m_{\lambda}(x,y))\\
                &\le m_{\lambda}^{f}(x,y) + m_{\lambda}^{g}(x, y) = m_{\lambda}^{f+g}(x,y) 
            \end{split}
        \end{equation*}
        \begin{equation*}
            \begin{split}
                \forall{x, y\in \left<a, b\right>}\quad \forall{\lambda\in (0, 1)}\quad \alpha f(m_{\lambda}(x,y)) \le \alpha m_{\lambda}^{f}(x, y) = m_{\lambda}^{\alpha f}(x, y)\qedhere 
            \end{split}
        \end{equation*}
    \end{proof}
\end{theorem}
\begin{theorem}[Лемма о трёх хордах]
    $f: \left<a, b\right> \mapsto \mathbb{R}$ - выпуклая.\\
    Тогда, если $x < z < y$ из $\left<a, b\right>$, то:
    \[ \frac{f(z)-f(x)}{z-x} \le \frac{f(y)-f(x)}{y-x} \le \frac{f(y)-f(z)}{y-z} .\]
    Каждое из трёх неравенств равносильно выпуклости.\\
    Если поставить строгие знаки, то выпуклость строгая.
    \begin{proof}
         Неравенство $1$:
         \begin{equation*}
             \begin{split}
                 &(y-x)(f(z)-f(x)) \le (z-x)(f(y)-f(x))\\
                 \iff &(y-x)f(z) - (y-x)f(x) \le (z-x)f(y)-(z-x)f(x)\\
                 \iff &(y-x)f(z) \le (z-x)f(y) + ((y-x)-(z-x))f(x)\\
                 \iff &(y-x)f(z) \le (z-x)f(y) + (y-z)f(x)
             \end{split}
         \end{equation*}
         Неравенство $2$:
         \begin{equation*}
             \begin{split}
                 &(y-z)(f(y)-f(x)) \le (y-x)(f(y)-f(z))\\
                 \iff &(y-x)f(z) \le (y-z)f(x)+((y-x)-(y-z))f(y)\\
                 \iff &(y-x)f(z) \le (y-z)f(x) + (z-x)f(y)
             \end{split}
         \end{equation*}
         Неравенство $3$:
         \begin{equation*}
             \begin{split}
                 &(y-z)(f(z)-f(x)) \le (y-x)(f(y)-f(z))\\
                 \iff &((y-z)-(z-x))f(z) \le (y-z)f(x) + (z-x)f(y)\\
                 \iff &(y-x)f(z) \le (y-z)f(x)+(z-x)f(y) \qedhere
             \end{split}
         \end{equation*}
    \end{proof}
\end{theorem}
\begin{theorem}
    $f: \left<a, b\right> \mapsto \mathbb{R}$ - выпуклая, $x\in \left( a, b \right) $.\\
    Тогда в $x$ существует конечная $f_{\pm}'(x)$, и $f_{-}'(x) \le f_{+}'(x)$.
    \begin{proof}
       \[ u < v < x < y < z\in \left<a, b\right> .\]
       \[ \frac{f(u)-f(x)}{u-x} \le \frac{f(v)-f(x)}{v-x} \le \frac{f(y)-f(x)}{y-x} \le \frac{f(z) - f(x)}{z-x} .\]
       Из первого неравенства следует, что $\frac{f(u)-f(x)}{u-x}$ возрастает по $u$, из второго неравенства следует, что она ограничена $\frac{f(y)-f(x)}{y-x}$.\\
       Тогда сщуествует конечная $f_{-}'(x) = \lim\limits_{u \to x-} \frac{f(u)-f(x)}{u-x} \le \frac{f(y)-f(x)}{y-x}$.\\
       Аналогично, $\frac{f(y)-f(x)}{y-x}$ возрастает по $y$, и онграничена снизу $f_{-}'(x)$.\\
       Значит, существует конечная $f_{-}' \le f_{+}'(x) = \lim\limits_{y \to x+} \frac{f(y)-f(x)}{y-x}$.
    \end{proof}
\end{theorem}
\begin{tlemma}
    $f : \left<a, b\right> \mapsto \mathbb{R}$ - выпуклая, тогда $f$ непрерывна на $\left( a, b \right) $.\\
    \begin{proof}
        Рассмотрим $x\in \left( a, b \right) $.\\
        Тогда существует конечная $f'_{\pm}(x)$, значит функция непрерывна слева в $x$ и справа в $x$.
    \end{proof}
\end{tlemma}
\begin{theorem}
    $f : \left<a, b\right> \mapsto \mathbb{R}$, дифференцируемая.\\
    Тогда $f$ - выпуклая тогда и только тогда $\forall{x, x_0\in \left<a, b\right>}\quad f(x) \ge (x-x_0)f'(x_0)+f(x_0)$.
    \begin{proof}
        Необходиомсть:
        При $x>x_0$ :
        \[ f'(x_0) = f_{-}'(x_0)\le  \frac{f(x)-f(x_0)}{x-x_0} .\]
        \[ (x-x_0)f'(x_0) + f(x_0) \le f(x) .\]
        При $x<x_0$ :
        \[ f'(x_0) = f_{+}'(x_0) \ge \frac{f(x)-f(x_0)}{x-x_0} .\]
        \[ (x-x_0)f'(x_0) + f(x_0) \le  f(x) .\]
        При $x=x_0$ равенство очевидно.\\
        Достаточность:
            \[ x < x_0 < y .\]
            \[ f(x) \ge (x-x_0)f'(x_0) + f(x_0) .\]
            \[ f(y) \le (y-x_0)f'(x_0) + f(x_0) .\]
            \begin{equation*}
                \begin{split}
                    (y-x_0)f(x) + (x_0-x)f(y) &\ge (y-x_0)(x-x_0)f'(x_0) + (y-x_0)f(x_0)\\
                                              &+(y-x_0)(x_0-x)f'(x_0)+(x_0-x)f(x_0)\\
                    &= ((y-x_0)+(x_0-x))f(x_0) = y-x)f(x_0) \qedhere
                \end{split}
            \end{equation*}
    \end{proof}
\end{theorem}
\begin{theorem}[Критерий выпуклости]
    $f : \left<a, b\right> \mapsto \mathbb{R}$, дифференцируемая на $\left( a, b \right) $ и непрерывная на $\left<a, b\right>$.\\
    Тогда:
    \begin{enumerate}
        \item $f$ выпуклая на $\left<a, b\right>$ тогда и только тогда, когда $f'$ возрастает.
        \item Если $f$ дважды дифференцируема, $f$ выпукла тогда и только тогда, когда $f'' \ge 0$ на $\left( a, b \right)$.
        \begin{proof}
            Необходимость:\\
            При $x>y$:
            \[ f'(y) \le \frac{f(x)-f(y)}{x-y} \le f'(x) .\]
            Значит, с ростом $x$ растёт $f'(x)$.
            Достаточность:
            \[ \frac{f(u)-f(v)}{u-v} \le \frac{f(w)-f(v)}{w-v} .\]
            По теореме Лагранжа: $\exists{p\in (u, v)}\quad \exists{q\in (v, w)}\quad $
            \[ f'(p) = \frac{f(u)-f(v)}{u-v} \le \frac{f(w)-f(v)}{w-v} = f(q) .\]
            Но $p<q$, значит $f'$ возрастает. 
        \end{proof}
    \end{enumerate}
\end{theorem}
\begin{theorem}[Неравенство Йенсена]
    $f : \left<a, b\right> \mapsto \mathbb{R}$, выпуклая.\\
    Тогда
    \[ \forall{x_1, \ldots, x_n\in \left<a, b\right>}\quad \forall{\lambda_1, \ldots, \lambda_n >0}\quad \sum\limits_{i=1}^{n}\lambda_i = 1 \implies f(\lambda_1x_1 + \ldots + \lambda_nx_n) \le \lambda_1 f(x_1) + \ldots + \lambda_{n}f(x_n) .\]
    \begin{proof}
        При $n=1$ - тривиально.
        При $n=2$ - опеределение выпуклости.\\
        Переход:
        \[ \sum\limits_{i=1}^{n}\lambda_{i} = 1 - \lambda_{n+1} .\]
        \[ y = \frac{\sum\limits_{i=1}^{n}\lambda_i x_i}{1-\lambda_{n+1}} .\] 
        \[ f((1-\lambda_{n+1})y+\lambda_{n+1}x_{n+1}) \le (1-\lambda_{n+1})f(y) + \lambda f(x_{n+1}).\]
        Раскрывая $f(y)$ по предположению индукции получим нужное утверждение.
    \end{proof}
\end{theorem}
\begin{theorem}[Неравенство о средних]
    $x_1, \ldots, x_n \ge 0$, тогда
    \[ \frac{x_1 + \ldots + x_{n}}{n} \ge \sqrt[n]{x_1 \cdot \ldots \cdot  x_n}  .\]
    \begin{proof}
        Если какой-то $x_i$ равен нулю, то утверждение оченвидно.\\
        \[ \ln \left( \frac{x_1 + \ldots + x_n}{n} \right) \ge \frac{\ln(x_1) + \ldots + \ln(x_n)}{n}  .\]
        Заметим, что логарифм - вогнутая функция, подставим $\lambda_i = \frac{1}{n}$ в неравенство Йенсена с функцией $f(x) = -\ln(x)$. Получим нужное утверждение.
    \end{proof}
\end{theorem}
\begin{definition}[Среднее степенное]
    Среднее степенное порядка $p$, $x_i > 0$:
    \[ M_p := \left( \frac{x_1^{p}+x_2^{p} + \ldots + x_{n}^{p}}{n} \right)^{\frac{1}{p}}  .\]
    \[ M_0 := \lim\limits_{p \to 0} M_p = \sqrt[n]{x_1x_2\ldots x_n}  .\] 
\end{definition}
\begin{dlemma}[Неравенство между средними степенными]
    Если $p < q$, то $M_{p} \le M_{q}$
    \begin{proof}
        Случай $0 < p < q$:
        $f(x) = x^{\frac{q}{p}}$ - выпуклая функция\\
        $y_i = x_i^{p}$, $\lambda_i = \frac{1}{n}$.\\
        \[ \left( \frac{x_1^{p}+x_2^{p}+\ldots+x_n^{p}}{n} \right)^{\frac{q}{p}} \le  \frac{x_{1}^{q}+x_2^{q}+\ldots+x_n^{q}}{n}  .\]
        Извлечаем корень степени $q$, получаем слева степень $\frac{1}{p}$, справа $\frac{1}{q}$, что соответвествтует $M_p \le M_{q}$.\\
        Случай $0 = p < q$:
        \[ \sqrt[n]{x_1x_2\ldots x_n} \le \left(\frac{x_1^{q}+x_2^{q}+\ldots x_n^{q}}{n}\right)^{\frac{1}{q}}  .\]
        Возводем в степень $q$, получаем неравенство о средних для $x_i^{q}$.\\
        Случай $p < q = 0$:
        Аналогично предыдущему, возводим в степень $p$, возведение в отрицательную сторону перевёрнет знак, и подойдёт неравество о средних для $x_i^{p}$.\\
        Случай $p < 0 < q$:
        \[ M_p < M_0 < M_q .\] 
    \end{proof}
\end{dlemma}
\section{Интегральное исчесление фукнции одной переменной}
    \subsection{Первообразная и непрерывный интеграл}
         \begin{definition}
             Первообразная функции $f : \left<a, b\right> \mapsto \mathbb{R}$, тогда $F : \left<a, b\right> $ называется первообразной для $f$, если $F$ дифференцируема, и $\forall{x\in \left<a, b\right>}\quad F'(x) = f(x)$.
         \end{definition}
         \begin{theorem}
             $f\in C\left<a, b\right>$, тогда у $f$ есть первообразная.\\
             Пока без доказательтва.
         \end{theorem}
         \begin{theorem}
             $f,F: \left<a, b\right> \mapsto \mathbb{R}$, $F$ -  первообразная $f$.\\
             \begin{enumerate}
                 \item Тогда $F+C$ - первообразная $f$ 
                 \item $\Phi$ -  первообразная $f$, то $\Phi = F + C$
             \end{enumerate}
             \begin{proof}
                 Пункт $1$ : $(F+C)' = F'+C' = F' = F$ \\
                 Пункт $2$ : Рассмотрим $(\Phi - F)' = \Phi' - F' = f - f =0$, значит $\Phi - F$ - постоянная, и $\Phi = F + C$.
             \end{proof}
         \end{theorem}
         \begin{definition}
             Множество всех первообразных для $f$ называется неопределённым интеграло $f$, и обозначается следующим образом:
             \[ \int f .\]
             \[ \int f(x)dx.\] 
         \end{definition}
         \begin{dlemma}
             \[ \int f = \{F+C \ssep C\in \mathbb{R}\}  .\]
             Где $F$ - произвольная первообразная.
         \end{dlemma}
         \begin{theorem}[Таблица интегралов]
             \[ \int 0 = C .\]
             \[ \int x^{p}dx = \frac{x^{p+1}}{p+1} + C \text{, $p \neq -1$} .\]
             \[ \int \frac{dx}{x} = \ln |x| + C.\]
             \[ \int a^{x}dx = \frac{a^{x}}{\ln a} + C \text{, $a \neq 1$}.\]
             \[ \int \sin x dx = -\cos x + C .\]
             \[ \int \cos x dx = \sin x + C .\] 
            \[ \int \frac{dx}{\cos x} = \tg x + C .\]
            \[ \int \frac{dx}{\sin x} = -\ctg x + C .\]
            \[ \int \frac{dx}{\sqrt{1-x^2} } = \arcsin x + C .\]
            \[ \int \frac{dx}{1+x^2} = \arctan x + C .\]
            \[ \int \frac{dx}{\sqrt{x^2\pm 1} } = \ln(x+\sqrt{x^2\pm 1}) + C  .\]
            \[ \int \frac{dx}{x^2-1} = \frac{1}{2}\ln\left|\frac{x-1}{x+1}\right| + C .\]
            \begin{proof}
                \[ \ln(-x)' = \frac{1}{-x}(-x)' = \frac{1}{x} \]
                \[ \left( \ln(x+\sqrt{x^2+1}  \right)' = \frac{1}{x+\sqrt{x^2+1} }\cdot (x'  + (\sqrt{x^2+1})') = \frac{1}{x+\sqrt{x^2+1} }\left(1 + \frac{x}{\sqrt{x^2+1} }\right) = \frac{1}{x^2+1}  .\]
                %FIXME
                \[ \left( \frac{1}{2}\ln\left|\frac{x-1}{x+1}\right| \right)' =   .\] 
            \end{proof}
         \end{theorem}
         \begin{theorem}[Ариф. действия с непор. интегралами]
            Пусть $f, g : \left<a, b\right> \mapsto \mathbb{R}$, $f$ и $g$ имеют первообразные.\\
            Тогда:
            \[ \int (f+g) = \int f + \int g .\]
            \[ \int \alpha f = \alpha \int f \text{, при $\alpha \neq 0$} .\]
            \begin{proof}
                Пусть $F, G$ - первообразные $f,g$.\\
                Тогда $(F+G)' = F'+G' = f+g$.\\
                $(\alpha F)' = \alpha F' = \alpha f$.
            \end{proof}
         \end{theorem}
         \begin{tlemma}[Линейность интеграла]
             $f, g : \left<a, b\right> \mapsto \mathbb{R}$, $f$ и $g$ имеют первообразные, $\alpha, \beta\in \mathbb{R}$, при $|\alpha| + |\beta| \neq 0$.\\
             Тогда:
             \[ \int (\alpha f + \beta g) = \alpha\int f + \beta \int g .\]
         \end{tlemma}
         \begin{theorem}[Замена переменной в неопр. интеграле]
             $f: \left<a, b\right> \mapsto \mathbb{R}$ имеет первообразную $F : \left<a, b\right> \mapsto \mathbb{R}$.\\
             $\varphi : \left<c, d\right> \mapsto \left<a, b\right>$ дифференцируема.\\
             Пусть $g(t) = f(\varphi(t))\varphi'(t)$, $g : \left<c, d\right> \mapsto \mathbb{R}$.\\
             Тогда:
             \[ \int f(\varphi(t))\varphi'(t)dt = F(\varphi(t)) + C .\]
             \begin{proof}
                 \[ F(\varphi(t))' = F'(\varphi(t))\varphi'(t) = f(\varphi(t))\varphi'(t) .\] 
             \end{proof}
         \end{theorem}
         \begin{tlemma}
             Если $F$ - первообразная $f$, то
             \[ \int f(\alpha x + \beta) = \frac{F(\alpha x + \beta)}{\alpha} + C.\]
             \begin{proof}
                 $\varphi(x) = \alpha x + \beta$.\\
                 $\varphi'(x) = \alpha$.\\
                 \[ \int f(\alpha x + \beta)\alpha = F(\alpha x + \beta) .\] 
             \end{proof}
         \end{tlemma}
         \begin{theorem}[Формула интегрирования по частям]
             $f,g: \left<a, b\right> \mapsto \mathbb{R}$ дифференцируемые и $f'g$ имеет первообразную.\\
             Тогда $fg'$ имеет первообразную.\\
             \[ \int fg' = fg - \int f'g .\]
             \[ \int fg' + \int f'g = fg .\] 
             \begin{proof}
             \[ H = \int f'g.\]
             \[ (fg-H)' = (fg)' - f'g = f'g + fg' - f'g = fg' .\] 
             \end{proof}
         \end{theorem}
    \subsection{Площади и определённый интеграл}
        \begin{definition}
            $\mathcal{F}$ - совокупность всех ограниченных подномжеств $\mathbb{R}^2$
        \end{definition}
        \begin{definition}
            $\sigma : \mathcal{F} \mapsto \left[0; +\infty\right]$ - площадь, если $\sigma\left( \left[a, b\right]\times \left[c, d\right] \right) = (b-a)(d-c)$, при $a\le b$, $c\le d$ и если $A, B\in \mathcal{F}$, $A\cap B = \emptyset$, то $\sigma(A \cup B) = \sigma\left(A\right) + \sigma(B) $.\\

        \end{definition}
        \begin{dlemma}
            Если $E \subset \tilde{E}$, то $\sigma(E) \le \sigma(\tilde{E})$.\\
            \begin{proof}
                \[ \tilde{E} = E \cup (\tilde{E} \setminus E) .\]
                \[ \sigma(\tilde{E}) = \sigma(E) + \sigma(\overline{E}) \ge \sigma(E) .\] 
            \end{proof}
        \end{dlemma}
        \begin{definition}
        $\sigma : \mathcal{F} \mapsto  [0, +\infty)$ назвоём квазиплощадью, если:
            \begin{enumerate}
                \item $\sigma\left( [a, b] \times [c, d] \right) = \left( b-a \right) \left( d-c \right)  $
                \item Если $E \subset  \tilde{E}$, то $\sigma(E) \le \sigma(\tilde{E})$
                \item $E\in \mathcal{F}$, $\ell$ - горизонтальная или вертикальная прямая.\\
                    Обозначим за $E_-$ чусть множества, лежащую слева от прямой, или на ней.\\
                    $E_{+} = E \ E_{-}$, $\sigma(E) = \sigma(E_{-}) + \sigma(E_{+})$.
            \end{enumerate}
        \end{definition}
        \begin{dlemma}
            $A$ - подмножество вертикального или горизонтального отрезка, то $\sigma(A) = 0$.\\
            \begin{proof}
                Если $A$ - отрезок, то $A = \left[a, a\right] \times \left[c, d\right]$, и площадь равна нулю.\\
                Площадь подмножества не больше площади самого отрезка.
            \end{proof}
        \end{dlemma}
        \begin{dlemma}
            Куда относятся точки на $\ell$ не имеет значения.\\
            \begin{proof}
                Пусть $\tilde{E}_{-} = E_{-} \cup \ell$.\\
                Тогда $\sigma(\tilde{E}_{-}) = \sigma(E_{-}) + \sigma(\tilde{E}_{-} \setminus E) = \sigma(E)$
            \end{proof}
        \end{dlemma}
        
\end{document} 
