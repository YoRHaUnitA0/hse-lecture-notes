% !TEX encoding = UTF-8 Unicode
\documentclass[11pt, oneside]{article}   	% use "amsart" instead of "article" for AMSLaTeX format
\usepackage{amssymb}
\usepackage{amsmath}
\usepackage{cRussian}
\usepackage{cPicture}
\usepackage{cTheorem}
\usepackage{cTikz}
\usepackage{xcolor}
\DeclareMathOperator{\const}{const}

\definecolor{func}{HTML}{0b8000}
\definecolor{func2}{HTML}{bd0000}
\definecolor{diff}{HTML}{3700ff}
\definecolor{diff2}{HTML}{8b00db}
\definecolor{dx}{HTML}{1da100}
\definecolor{x0}{HTML}{800800}

\newcommand{\dk}{\textcolor{diff}{k}}
\newcommand{\df}{\textcolor{diff}{f'}}
\newcommand{\dg}{\textcolor{diff2}{g'}}
\newcommand{\rf}{\textcolor{diff}{f_{+}'}}
\newcommand{\lf}{\textcolor{diff}{f_{-}'}}
\newcommand{\ds}{\textcolor{diff}{\varphi}}
\newcommand{\dl}{\textcolor{diff2}{\psi}}
\newcommand{\dx}{\textcolor{dx}{dx}}
\newcommand{\xz}{\textcolor{x0}{x_0}}
\newcommand{\yz}{\textcolor{x0}{y_0}}
\newcommand{\ff}{\textcolor{func}{f}}
\newcommand{\fg}{\textcolor{func2}{g}}

\newcommand{\px}{\textcolor{func}{x}}
\newcommand{\py}{\textcolor{func}{y}}
\title{Мат. Анализ 8}
\author{Igor Engel}
\date{}

\begin{document}
\maketitle
\section{Дифференциальное исчисление}
    \subsection{Дифференцируемость и производная}
        \begin{definition}
            $\ff: \left<a,b\right> \mapsto \mathbb{R}$, $\xz\in \left<a, b\right>$.\\
            $\ff$ дифференцируема в точке $\xz$, если 
            \[ \exists{\dk\in \mathbb{R}}\quad \ff(\px) = \ff(\xz) + \dk(\px-\xz) + o(\px-\xz)\quad \px \to \xz .\]
            \[ \exists{\dk\in \mathbb{R}}\quad \ff(\xz+\dx) = \ff(\xz) + \dk\dx + o(\dx)\quad \dx\to 0 .\] 
        \end{definition}
        \begin{definition}
            Производная $\ff$ в точке $\xz$:
            \[ \df(\xz) = \lim\limits_{\px \to \xz} \frac{\ff(\px)-\ff(\xz)}{\px-\xz} .\] 
            \[ \df(\xz) = \lim\limits_{\dx \to 0} \frac{\ff(\xz+\dx)-\ff(\xz)}{\dx}  .\] 
        \end{definition}
        \begin{theorem}[Критерий дифференцируемости]
            $\ff: \left<a, b\right> \mapsto \mathbb{R}$, $\xz\in \left<a, b\right>$.\\
            Следующие условия равносильны:
            \begin{enumerate}
                \item $\ff$ дифференцируема в точке $\xz$
                \item Существует конечная $\df(\xz)$
                \item $\exists{\ds: \left<a, b\right> \mapsto \mathbb{R}}\quad \ff(\px)-\ff(\xz)=\ds(\px)(\px-\xz)$, $\ds$ непрерывна в точке $\xz$.
                \item $\exists{\ds: \left<a, b\right> \mapsto \mathbb{R}}\quad \ff(\xz+\dx)-\ff(\xz)=\ds(\xz+\dx)\dx$, $\ds$ непрерывна в точке $\xz$
            \end{enumerate}
            Если эти условия выполняются, то $\dk=\df(\xz) = \ds(\xz)$.
            \begin{proof}
            $\dk \implies \df$:\\
            \[ \ff(\px)-\ff(\xz) = \dk(\px-\xz) + o(\px-\xz) \implies \frac{\ff(\px)-\ff(\xz)}{\px-\xz} = \dk + \frac{o(\px-\xz)}{\px-\xz}\to \dk .\]
              \[ \ff(\xz+\dx) - \ff(\xz) = \dk\dx + o(\dx) \implies \frac{\ff(\xz+\dx) - \ff(\xz)}{\dx} = \dk + \frac{o(\dx)}{\dx} \to \dk .\] 
              Тогда производная существует, конечна, и равна $\dk$.\\
              $\df \implies \ds$:\\
              \begin{equation*}
                  \ds(\px) = \begin{cases}
                      \frac{\ff(\px)-\ff(\xz)}{\px-\xz} & x \neq \xz\\
                      \df(\px)
                  \end{cases}
              \end{equation*}
              \begin{equation*}
                  \ds(\xz+\dx) = \begin{cases}
                      \frac{\ff(\xz+\dx)-\ff(x_{0})}{\dx} & \dx \neq 0\\
                      \df(\xz)
                  \end{cases}
              \end{equation*}
              Проверим что $\ds$ непрерывна:\\
              \[ \lim\limits_{\px \to \xz} \ds(\px) = \lim\limits_{\px \to \xz} \frac{\ff(\px)-\ff(\xz)}{\px-\xz} = \df(\xz) = \ds(\xz) .\]
              \[ \lim\limits_{\dx \to 0} \ds(\xz+\dx) = \lim\limits_{\dx \to 0} \frac{\ff(\xz+\dx)-\ff(\xz)}{\dx} = \df(\xz) = \ds(\xz) .\] 
              $\ds \implies \dk$:\\
              $\ds$ непрерына, значит, $\ds(\px) = \ds(\xz) + o(1)$.
              \[\ff(\px) = \ff(\xz) + \ds(\px)(\px-\xz) = \ff(\xz) + (\ds(\xz) + o(1))(\px-\xz) = \ff(\xz) + \ds(\xz)(\px-\xz) + o(\px-\xz)\]
              \[ \ff(\xz+\dx) = \ff(\xz)+\dx\cdot \ds(\xz+\dx) = \ff(\xz) + \dx\left( \ds(\xz)+o(1) \right) = \ff(\xz)+\ds(\xz)\cdot \dx + o(\dx) .\]
              Значит, $\dk=\ds(\xz)$.
          \end{proof}
      \end{theorem}
      \begin{definition}
          Производная справа:
          \[ \rf(\xz) = \lim\limits_{\px \to \xz+} \frac{\ff(\px)-\ff(\xz)}{\px-\xz} .\]
          \[ \rf(\xz) = \lim\limits_{\dx \to 0+} \frac{\ff(\xz+\dx)-\ff(\xz)}{\dx}  .\] 
          Аналогично слева.
      \end{definition}
      \begin{dlemma}
          \[ \rf(\xz) = \lf(\xz) \implies \df(\xz) = \rf(\xz) = \lf(\xz) .\] 
      \end{dlemma}
      \begin{definition}
          \[ \ff(\xz+\dx) = \ff(\xz) + \dk\dx + o(\dx) .\]
          Отображение $\dx \mapsto  \dk\dx$ называется дифференциалом.
      \end{definition}
      \begin{theorem}
          Если $\ff$ дифференцируема в $\xz$, то $\ff$ непрерывна в $\xz$.
          \begin{proof}
              \[ \ff(\px+\dx) = \ff(\xz) + \dk\dx+o(\dx) .\]
              \[ \lim\limits_{\px \to \xz} \ff(\px) = \ff(\xz) + \lim\limits_{\dx \to 0} \dk\dx + \lim\limits_{\dx\to 0} o(\dx) = \ff(\xz)  .\qedhere\] 
          \end{proof}
      \end{theorem}
      \begin{theorem}[Арифметические действия с производными]
          $\ff,\fg: \left<a, b\right> \mapsto \mathbb{R}$, $\xz\in \left<a, b\right>$, $\ff,\fg$ - дифференцируемы в $\xz$.\\
          Тогда:
          \[ (\ff\pm \fg)'(\xz) = \df(\xz) \pm \dg(\xz) .\]
          \begin{proof}
              \[ \lim\limits_{\px \to \xz} \frac{(\ff\pm \fg)(\px)-\left( \ff\pm \fg \right)(\xz) }{\px-\xz} = \lim\limits_{\px \to \xz} \frac{\ff(\px)-\ff(\xz)+\fg(\px)-\fg(\xz)}{\px-\xz} = \df(\xz)+\dg(\xz) .\] 
          \end{proof}
          \[ (\ff\fg)'(\xz) = \df(\xz)\fg(\xz) + \ff(\xz)\dg(\xz) .\]
          \begin{proof}
              \begin{equation*}
                  \begin{split}
                      \lim\limits_{\px \to \xz} \frac{(\ff\fg)(\px)-(\ff\fg)(\xz)}{\px-\xz} 
                      &= \lim\limits_{\px \to \xz} \frac{\ff(\px)\fg(\px)-\ff(\xz)\fg(\xz)}{\px-\xz}\\
                      &= \lim\limits_{\px \to \xz} \frac{\ff(\px)\fg(\px)-\ff(\xz)\fg(\xz)+\ff(\px)\fg(\xz)-\ff(\px)\fg(\xz)}{\px-\xz}\\
                      &= \lim\limits_{\px \to \xz} \frac{(\ff(\px)\fg(\px)-\ff(\px)\fg(\xz))+(\ff(\px)\fg(\xz)-\ff(\xz)\fg(\xz))}{\px-\xz}\\
                      &= \lim\limits_{\px \to \xz} \ff(\px) \frac{\fg(\px)-\fg(\xz)}{\px-\xz} + \lim\limits_{\px \to \xz} \fg(\xz) \frac{\ff(\px)-\ff(\xz)}{\px-\xz}\\
                      &= \ff(\xz)\dg(\xz)+\fg(\xz)\df(\xz)
                  \end{split}
                  %FIXME: Insert drawing
              \end{equation*}
          \end{proof}
          \[ (cf)'(\xz) = c\df(\xz) .\]
          \begin{proof}
              $\fg(\px) = c \implies \df(\px) = 0 \implies (\ff\fg)'(\xz) = \ff(\xz)\dg(\xz)+\fg(\xz)\df(\xz)=c\df(\xz)$.
          \end{proof}
          \[ \fg(\xz) \neq 0 \implies \left(\frac{\ff}{\fg}\right)'(\xz) = \frac{\df(\xz)\fg(\xz) -\ff(\xz)\dg(\xz)}{\fg^2(\xz)}  .\]
          \begin{proof}
              \begin{equation*}
                  \begin{split}
                      \lim\limits_{\px \to \xz} \frac{\left( \frac{\ff}{\fg} \right)(\px) - \left( \frac{\ff}{\fg} \right)(\xz)  }{\px-\xz}
                      &= \lim\limits_{\px \to \xz} \frac{\frac{\ff(\px)}{\fg(\px)} - \frac{\ff(\xz)}{\fg(\xz)}}{\px-\xz}\\
                      &= \lim\limits_{\px \to \xz} \frac{1}{\fg(\px)\fg(\xz)} \cdot \lim\limits_{\px \to \xz} \frac{\ff(\px)\fg(\xz)-\ff(\xz)\fg(\px)}{\px-\xz}\\
                      &= \frac{1}{\fg^2(\xz)} \cdot \lim\limits_{\px \to \xz} \frac{\ff(\px)\fg(\xz)-\ff(\xz)\fg(\px)+\ff(\xz)\fg(\xz)-\ff(\xz)\fg(\xz)}{\px-\xz}\\
                      &= \frac{1}{\fg^2(\xz)} \cdot  \left(\lim\limits_{\px \to \xz} \fg(\xz) \frac{\ff(\px)-\ff(\xz)}{\px-\xz} -  \lim\limits_{\px \to \xz} \ff(\xz) \frac{\fg(\px)-\fg(\xz)}{\px-\xz}\right)\\
                      &= \frac{\fg(\xz)\df(\xz)-\ff(\xz)\dg(\xz)}{\fg^2(\xz)}
                  \end{split}
              \end{equation*}
          \end{proof} 
      \end{theorem}
      \begin{theorem}[Производная композиции]
          $\ff:\left<a, b\right> \mapsto \mathbb{R}$, $\fg: \left<c, d\right> \mapsto \left<a, b\right>$, $\xz\in \left<c, d\right>$, $\fg$ дифференцируема в $\xz$, $\ff$ в $\fg(\xz)$.
          \[ (\ff\circ \fg)(\px) = \df(\fg(\px))\dg(\px) .\]
          \[ \frac{d(\ff\circ \fg)}{\dx} = \frac{d\ff}{d\fg}\frac{d\fg}{\dx} .\]
          \begin{proof}
              \[ \yz = \fg(\xz) .\] 
              \[ \fg(\px)-\fg(\xz) = \dl(\px)(\px-\xz) \text{, $\dl$ непрерывна в  $\xz$} .\]
              \[ \ff(\py)-\ff(\yz) = \ds(\py)(\py-\yz) \text{, $\ds$ непрерывна в $\yz$} .\] 
              \[ \ff(\fg(\px))-\ff(\fg(\xz)) = \ds(\fg(\px))(\dl(\px)(\px-\xz)) .\]
              $\fg(\px)$ непрерывна в $\xz$, значит $\ds(\fg(\px))$ непрерывна в $\xz$, и $\ds(\fg(\px))\dl(\px)$ непрерынва в $\xz$, как произведение непрерывных.
          \end{proof}
      \end{theorem}
      \begin{theorem}[Производная обратной фнукции]
          $\ff: \left<a, b\right> \mapsto \mathbb{R}$, $\ff$ непрерывна, строго монотонна и дифференцируема в $\xz$, при этом $\df(\xz) \neq 0$.\\
          Тогда, $\fg=\ff^{-1}$ дифференцируема в $\yz=\ff(\xz)$, $\dg(\yz) =\frac{1}{\df(\xz)} = \frac{1}{\df(\fg(\xz))}$.
          \begin{proof}
              \[ \ff(\px) - \ff(\xz) = \ds(\px)(\px-\xz) .\]
              Заметим, что $\ds(\px) \neq 0$, так-как в $\px\neq \xz \implies \ff(\px)-\ff(\xz) \neq 0$, а в $\xz$ она равна производной.\\
              \[ \px-\xz = \frac{1}{\ds(\px)}(\ff(\px)-\ff(\xz)) \iff \fg(\py) - \fg(\yz) = \frac{1}{\ds(\px)}(\py-\yz) .\]
              Так-как $\ds(\px) \neq 0$, то $\frac{1}{\ds(\px)}$ непрерывна.\\
              \[ \dg(\yz) = \frac{1}{\ds(\xz)} = \frac{1}{\df(\xz)} = \frac{1}{\df(\fg(\yz))} .\qedhere\] 
          \end{proof}
      \end{theorem}
    \subsection{Производные элементарных функций}
        \[ c' = 0 .\]
        \[ (x^{p})' = px^{p-1} .\]
        \[ (a^{x})' = a^{x}\ln a \text{, $a>0$} .\]
        \[ (e^{x})' = e^{x} .\]
        \[ (\log_a x)' = \frac{1}{x\ln a} \text{, $a>0$,  $a\neq 1$} .\]
        \[ (\ln x)' = \frac{1}{x} .\]
        \[ (\sin x)' = \cos x .\]
        \[ (\cos x)' = \sin x .\]
        \[ (\tan x)' = \frac{1}{\cos^2 x} .\]
        \[ (\ctg x)' = \frac{1}{\sin^2 x} .\] 
        \[ (\arctan x)' = \frac{1}{1+x^2} .\]
        \[ (\arcctg x)' = -\frac{1}{1+x^2} .\]
        \[ (\arcsin x)' = \frac{1}{\sqrt{1-x^2} } .\]
        \[ (\arccos x)' = 0\frac{1}{\sqrt{1-x^2} } .\] 
        \begin{proof}
            \[ (x^{p})' = \lim\limits_{\dx \to 0} \frac{(x+\dx)^{p}-x^{p}}{\dx} = x^{p} \lim\limits_{\dx \to 0} \frac{(1+\frac{\dx}{x})^{p} - 1}{\frac{\dx}{x}} = x^{p}\cdot p\cdot \frac{1}{x} = px^{p-1}  .\]
            \[ (a^{x})' = \lim\limits_{\dx \to 0} \frac{a^{x+h}-a^{x}}{h} = a^{x} \lim\limits_{\dx \to 0} \frac{a^{\dx}-1}{\dx} = a^{x}\ln a .\]
            \[ (\ln x)' = \lim\limits_{\dx \to 0} \frac{\ln(x+\dx)-\ln x}{\dx} = \lim\limits_{\dx \to 0} \frac{\ln\left( 1+\frac{\dx}{x} \right) }{\frac{\dx}{x}}\cdot \frac{1}{x} = \frac{1}{x}  .\]
            \[ (\sin x)' = \lim\limits_{\dx \to 0} \frac{\sin(x+\dx)-\sin x}{\dx} = \lim\limits_{\dx \to 0} \frac{2\left(\sin \frac{\dx}{2}\cos\left( x+\frac{\dx}{2} \right)\right) }{\dx} = \cos x .\]
            Косинус аналогично, остальные по арифметике и обратным.
        \end{proof}
    \subsection{Теорема о среднем}
        \begin{theorem}[Теорема Ферма]
            $\ff: \left<a, b\right> \mapsto \mathbb{R}$, $\xz\in \left( a, b \right)$, и $\ff(\xz) = \max \ff(\px)$ или $\min \ff(\px)$, и $\ff$ дифференцируема в $\xz$, тогда $\df(\xz) = 0$.
            \begin{proof}
                Для определённости, пусть будем минмум.
                \begin{equation*}
                    \begin{cases}
                        \rf(\xz) = \lim\limits_{\px \to \xz+} \frac{\ff(\px) - \ff(\xz)}{\px-\xz} \ge 0\\
                        \lf(\xz) = \lim\limits_{\px \to \xz-} \frac{\ff(\px)-\ff(\xz)}{\px-\xz} \le 0 
                    \end{cases}
                    \implies \df(\xz) = 0
                \end{equation*}

            \end{proof}
        \end{theorem}
        \begin{theorem}[Теорма Ролля]
            $\ff: \left[a, b\right] \mapsto \mathbb{R}$, дифференцируема на $\left( a, b \right) $, непрерывна на всём отрезке, и $\ff(a) = \ff(b)$.\\
            Тогда существует $c\in \left( a, b \right) $, $\ff(c) = 0$.\\
            \begin{proof}
                По теореме Вейерштрасса, $\ff$ достигает максимума и минимума в каких-то точках.\\
                Если максимум или минимум внутри интервала, то по теореме Ферма производная там равна нулю.\\
                Если и максимум и минимум достагается на концевых точках, то функция постоянна на отрезке, и её производная равна нулю.
            \end{proof}
        \end{theorem}
        \begin{theorem}[Теорема Лагранжа (формула конечных приращений)]
            $\ff: \left[a, b\right] \mapsto \mathbb{R}$, дифференцируема на интервале, непрерывна на отрезке.\\
            \[ \exists{c\in \left(a, b\right)}\quad \ff(b)-\ff(a) = \df(c)(b-a) .\]
            \[ \exists{c\in (a, b)}\quad \frac{\ff(b) - \ff(a)}{b-a} = \df(c) .\] 
            \begin{proof}
                \[ \dk = \frac{\ff(b)-\ff(a)}{b-a} .\] 
                \[ \fg(\px) = \ff(\px) - \dk(\px-a) .\]
                \[ \fg(a) = \ff(a) .\]
                \[ \fg(b) = \ff(b) - \frac{\ff(b) - \ff(a)}{b-a}\left( b-a \right) = \ff(a) .\] 
                Применим теорему Ролля: \[ \exists{c\in \left( a, b \right) }\quad \dg(c) = 0 = \df(c)-\dk = \df(c) - \frac{\ff(b)-\ff(a)}{b-a} \implies \df(c)(b-a) = \ff(b) - \ff(a) .\qedhere\] 
            \end{proof}
        \end{theorem}
        \begin{theorem}[Теорема Коши (очередная)]
            $\ff, \fg: \left[a, b\right] \mapsto \mathbb{R}$, $\ff,\fg$ непрерывны на отрезке, дифференцируемы на интервале, $\forall{\px\in \left( a, b \right) }\quad \dg(\px) \neq  0$.\\
            Тогда
            \[ \exists{c\in (a, b)}\quad \frac{\ff(b) - \ff(a)}{\fg(b) - \fg(a)} = \frac{\df(c)}{\dg(c)} .\]
            \begin{proof}
                \[ \dk = \frac{\ff(b) - \ff(a)}{\fg(b) - \fg(a)} .\] 
                \[ h(\px) = \ff(\px) - \dk(\fg(\px)-\fg(a)) .\]
                \[ \exists{c\in \left( a, b \right) }\quad h'(c) = 0 = \df(c) - \dk\dg(c) = \df(c) - \frac{\ff(b) - \ff(a)}{\fg(b)-\fg(a)}\dg(c) \implies \frac{\df(c)}{\dg(c)} = \frac{\ff(b) - \ff(a)}{\fg(b)-\fg(a)}.\qedhere\]
            \end{proof}
        \end{theorem}
        \subsubsection{Следствия из Лагранжа}
            
            $\ff: \left<a, b\right> \mapsto \mathbb{R}$, дифферецируема на интервале, непрерывна на промежутке.\\
            \begin{tlemma}
                Если $\forall{\px\in \left( a, b \right) }\quad |\df(\px)| \le  M$, то 
                \[\forall{\px, \py\in \left( a, b \right) }\quad |\ff(\px)-\ff(\py)| \le M|\px-\py|.\]
                Такая функция называется Липшицевой функйией с константой $M$.
                \begin{proof}
                    По Т. Л. $\exists{c\in \left( \px, \py \right) }\quad |\ff(\py)-\ff(\px)| = \df(c)|\px-\py| \le M|\px-\py|$.
                \end{proof}
            \end{tlemma}
            \begin{tlemma}
                Если $\forall{x\in (a, b)}\quad \df(\px) \ge 0$, то $\ff$ нестрго монотонно возрастает.\\
                \begin{proof}
                    \[ \ff(y) - \ff(\px) = \df(c)(y-x) \ge  0 \iff y \ge  x .\] 
                \end{proof}
            \end{tlemma}
            \begin{tlemma}
                Если $\forall{x\in \left( a, b \right) }\quad \df(\px) > 0$, то $\ff$ строго монотонно возрастает.
            \end{tlemma}
            \begin{tlemma}
                Если $\forall{x\in \left( a, b \right) }\quad \df(\px) \le 0$, то $\ff$ нестрого монотонно убывает.
            \end{tlemma}
            \begin{tlemma}
                Если $\forall{x\in \left( a, b \right) }\quad \df(\px) < 0$, то $\ff$ строго монотонно убывает.
            \end{tlemma}
            \begin{tlemma}
                Если $\forall{x\in \left( a, b \right) }\quad \df(\px) = 0$, то $\ff(\px) = \const$.
            \end{tlemma}
            \begin{theorem}[Теорема Дарбу]
                $\ff: \left[a, b\right] \mapsto \mathbb{R}$, $\ff$ дифференцируема на отрезке.\\
                Пусть $C$ лежит строго между $\df(a)$, $\df(b)$, тогда $\exists{c\in \left( a, b \right) }\quad \df(c) = C$
                \begin{proof}
                    Случай $C=0$.\\
                    Для определённости $\df(a) < 0 < \df(b)$.\\
                    $\ff$ непрерывна, значит принимает максимальное и минимальное значение на отрезке.\\
                    При таких знаках производной, минимум функции точно будет внутри интервала, и в этой точке $\df(\px) = 0$.\\
                    \[ \df(a) = \lim\limits_{\px \to a+} \frac{\ff(\px)-\ff(a)}{\px-a}  .\]
                    Пусть $a$ - точка минимума. Тогда числитель больше нуля, знаменатель больше нуля, предел $\ge 0$, но $\df(a) < 0$.\\
                    Аналогично для $\df(b)$.\\
                    При другой расстановке знаков, внутри будет максимум.\\
                    Если $C\neq 0$, можно из $\ff$ вычесть фукнцию $C\px$, и получится случай для $C=0$.\\
                \end{proof}
            \end{theorem}
            \begin{tlemma}
                Пусть $\ff: \left<a, b\right> \mapsto \mathbb{R}$, $\ff$ непрерывна на промежутке, диффренцируема на интервале, $\df(\px) \neq 0$ на всём интервале. Тогда $\ff$ строго монотонна.
                \begin{proof}
                    Если существет точки в которых производная разных знаков, то между ними была-бы точка с нулевой производной.
                \end{proof}
            \end{tlemma}
            \begin{theorem}[Правило Лопиталя]
                Пусть $-\infty \le a < b \le +\infty$.\\
                $\ff,\fg : \left( a, b \right) \mapsto \mathbb{R}$, дифференцируемы на $\left( a,b \right) $, $\dg(\px) \neq0$.\\
                $\lim\limits_{\px \to b-} \ff(\px) = \lim\limits_{\px \to b-} \fg(\px) = 0$.\\
                Если $\lim\limits_{\px \to b-} \frac{\df(\px)}{\dg(\px)} = \ell\in \overline{\mathbb{R}}$, то $\lim\limits_{\px \to b-} \frac{\ff(\px)}{\fg(\px)} = \ell$
                \begin{proof}
                    По предыдущей лемме, $\fg$ строго монотонно, значит, она либо везде положительна, либо везде отрицательная, и $\fg(\px) \neq 0$.\\
                    Проверим предел по Гейне. Возьмём монотонно возрастающую последовательность $x_n \to b$.\\
                    Тогда $\fg(x_n)$ строго монотонно.\\
                    Тогда, по теореме Штольца:
                    \[ \lim\frac{\ff(x_n)}{\fg(x_n)} = \lim \frac{\ff(x_{n}) - \ff(x_{n-1})}{\fg(x_n) - \fg(x_{n-1})} .\]
                    \[ \exists{c_n\in \left( x_{n-1}, x_n \right) }\quad \frac{\ff(x_n) - \ff(x_{n-1})}{\fg(x_n) - \fg(x_{n-1})} = \frac{\df(c)}{\dg(c)} .\]
                    $c_n \to b$, значит, предел равен $\ell$.\\
                \end{proof}
            \end{theorem}
            \begin{theorem}[Лопиталь для бескончностей]
                Меняем условие $\lim\limits_{x \to b-} \ff(\px) = \lim\limits_{x \to b-} g(\px) = 0$, на $\lim\limits_{x \to b-} g(\px) = +\infty$. Доказательство аналогично Лопиталю для нулей, с заменёнными пределами последовательностей.
            \end{theorem}
    \subsection{Производные высших порядков}
        \begin{definition}
            $f: \left<a, b\right> \mapsto \mathbb{R}$ дифференцируема в окрестности точки $\xz$, тогда $\df$ определена в окретнсоти $\xz$, и если $\df$ дифференцируема, то  $\ff$ дифференцируема дважды, в  $\xz$.\\ 
            Вторая производная $\ff$ - производная $\df$.
        \end{definition}
\end{document} 
