% !TEX encoding = UTF-8 Unicode
\documentclass[11pt, oneside]{article}   	% use "amsart" instead of "article" for AMSLaTeX format
\usepackage{amssymb}
\usepackage{amsmath}
\usepackage{cRussian}
\usepackage{cPicture}
\usepackage{cTheorem}
\DeclareMathOperator{\sgn}{sign}
\DeclareMathOperator{\argmin}{argmax}
\DeclareMathOperator{\argmax}{argmin}
\usepackage{cTikz}
\title{Матан 6}
\author{Igor Engel}
\date{}

\begin{document}
\maketitle
\section{Арифметика пределов}
    \begin{theorem}[Предельный переход в неравенстве]
        \[ f, g: E \mapsto \mathbb{R} .\]
        \[ \forall{x\in E}\quad f(x) \le g(x)  .\]
        $a$ - предельная точка $e$, и существуют $\lim\limits_{x \to a} f(x) = A$ и $\lim\limits_{x \to a} g(x) = B$.\\
        Тогда $\lim\limits_{x \to a} f(x) \le \lim\limits_{x \to a} g(x)$.
        \begin{proof}
            Пусть $\{x_n\} \to a$.\\
            Тогда $\lim f(x_n) = A$, $\lim g\left( x_n \right) $.\\
            \[ \forall{x_n}\quad f(x_n) \le g(x_n) \implies A\le B .\qedhere\] 
        \end{proof}
    \end{theorem}
    \begin{theorem}[Теорема о двух миллиционерах]
        \[ f,g,h: E \mapsto \mathbb{R} .\]
        $a$ - предельная точка $E$.\\
        \[ \forall{x\in E}\quad f(x) \le g(x) \le h(x)   .\]
        Если $\lim\limits_{x \to a} f(x) = \lim\limits_{x \to a} h(x) = A$, то $\lim\limits_{x \to a} g(x) = A$.
        \begin{proof}
            Пусть $\{x_n\} \to a$.\\
            \[ \lim f(x_n) = \lim h(x)n) = A .\]
            \[ A \le \lim g(x_n) \le A \implies \lim g(x_n) = A \implies \lim\limits_{x \to a} g(x) = A .\qedhere\] 
        \end{proof}
    \end{theorem}
    \begin{definition}[Предел слева и справа]
        \[ f: E \mapsto \mathbb{R} .\] 
        \[ E_1 = \left( -\infty; a \right)\cap E .\]
        $a$ - предельная точка $ E_1$.\\
        \[ g = \left.f\right|_{E_1}\]
        \[ \lim\limits_{x \to a-} f(x) = \lim\limits_{x \to a-0} f(x) = \lim\limits_{x \to a} g(x)  .\] 
        \[ E_2 = \left( a, +\infty \right)\cap E .\]
        $a$ - предельная точка $ E_2$.\\
        \[ h = \left.f\right|_{E_2} .\]
            \[ \lim\limits_{x \to a+} f(x) = \lim\limits_{x \to a+0} f(x) = \lim\limits_{x \to a} h(x) .\] 
    \end{definition}
    \begin{example}
        \[ f(x) = \left[x\right] .\]
        \[ \lim\limits_{x \to n+} \left[x\right] = n.\]
        \[ \lim\limits_{x \to n-} \left[x\right] = n - 1.\] 
    \end{example}
    \begin{dlemma}
        \begin{equation*}
            \begin{split}
                \lim\limits_{x \to a+} f(x) &= A\\
                &\iff \forall{\varepsilon>0}\quad \exists{\delta > 0}\quad \forall{x\in E}\quad 0<x-a<\delta \implies |f(x) - A| < \varepsilon 
            \end{split} 
        \end{equation*}
        \begin{equation*}
            \begin{split}
                \lim\limits_{x \to a-} f(x) &= A\\
                &\iff \forall{\varepsilon>0}\quad \exists{\delta > 0}\quad \forall{x\in E}\quad 0<a-x<\delta \implies |f(x) - A| < \varepsilon 
            \end{split} 
        \end{equation*}
    \end{dlemma}
    \begin{dlemma}
        \[ \lim\limits_{x \to a-} f(x) = \lim\limits_{x \to a+} f(x) = A \iff \lim\limits_{x \to a} f(x) = A .\] 
    \end{dlemma}
    \begin{definition}
        $f: E \mapsto \mathbb{R}$ - монотонно возрастает если
        \[ \forall{x, y\in E}\quad x<y \implies f(x) \le  f(y) .\]
        Строго монотонно возрастает если
        \[ \forall{x,y\in E}\quad x<y \implies f(x) < f(y)  .\]
        Аналогично для убывания
    \end{definition}
    \begin{dlemma}
        $f: E \mapsto \mathbb{R}$, $E_1 = \left( -\infty, a \right)\cap E$.\\
        $a$ - предельная точка $ E_1$.\\
        Если $f$ монотонно возрастает и ограниченна сверху, то 
        \[ \lim\limits_{x \to a-} f(x) = \sup\limits_{x\in E_1} f(x) .\]
        Если $f$ монотонно убывает и ограниченна снизу, то
        \[ \lim\limits_{x \to a-} f(x) = \inf\limits_{x\in E_1} f(x)  .\]
        \begin{proof}
            Пусть $\sup\limits_{x\in E_1} f(x) = B < \infty$.\\
            \[ \forall{\varepsilon>0}\quad \exists{y\in E}\quad f(y)>B-\varepsilon .\]
            Пусть $\delta = a-y$, возьмём $0<a-x<\delta \iff y<x<a$\\
            \[ B-\varepsilon < f(y) \le f(x) \le B < B+\varepsilon. \implies \lim\limits_{x \to a-} f(x) = B.\qedhere\]
        \end{proof}
    \end{dlemma}
    \begin{theorem}[Критерий Коши]
        \[ f: E \mapsto \mathbb{R} .\] 
        $a$ - предельная точка $E$.\\
        \[ \lim\limits_{x \to a} f(x) = A < \infty \iff \forall{\varepsilon>0}\quad \exists{\delta>0}\quad \forall{x,y\in E}\quad  \begin{cases} 0<|x-a|<\delta\\ 0<|y-a|<\delta \end{cases} \implies |f(x)-f(y)| < \varepsilon  .\]
        \begin{proof}
            Необходимость:
            \[ \forall{\varepsilon>0}\quad \exists{\delta > 0}\quad \forall{x\in E}\quad 0<|x-a|<\delta \implies |f(x)-A| < \varepsilon   .\] 
            \[ \forall{\varepsilon>0}\quad \exists{\delta > 0}\quad \forall{y\in E}\quad 0<|y-a|<\delta \implies |f(y)-A| < \varepsilon   .\] 
            \[ |f(x)-f(y)| \le |f(x)-A|+|A-f(y)| < 2\varepsilon .\]
            Достаточность:
            Пусть $\{x_n\} \to a$.\\
            Возьмём $\delta>0$ по $\varepsilon$.\\
            По $\delta$ найдётся $N$, такой, что $\forall{n>N}\quad |x_n-a|<\delta $, $\forall{m>N}\quad |x_m-a|<\delta $.\\
            Тогда
            \[ \forall{n,m>N}\quad |f(x_n)-f(x_m)|<\varepsilon  .\]
            Последовательность $f(x_n)$ - фундаментальна, значит у неё есть конечный предел.
        \end{proof}
    \end{theorem}
\section{Непрерывные функции}
    \begin{definition}
        $f: E \mapsto \mathbb{R}$, $a\in E$.\\
        $f$ непрерывна в точке $a$, если выполняется одно из равносильных условий:
        \begin{enumerate}
            \item $a$ - не предельная точка $E$ или $a$ - предельная точка и $\lim\limits_{x \to a} f(x) = f(a)$.
            \item $\forall{U_{f(a)}}\quad \exists{U_a}\quad f(U_a\cap E) \subset U_{f(a)}$
            \item $\forall{\varepsilon>0}\quad \exists{\delta>0}\quad \forall{x\in E}\quad |x-a|<\delta \implies |f(x)-f(a)| < \varepsilon$
            \item Для любой последовательности $\{x_n\in E\} \to a $, $\lim f(x_n) = f(a) $
        \end{enumerate}
    \end{definition}
    \begin{theorem}[Арифметические действия с непрерывными функциями]
        $f, g : E \mapsto \mathbb{R}$, $a\in E$, $f,g$ - непрерывны в точке $a$.\\
        Тогда:
        \begin{enumerate}
            \item $f\pm g$ - непрерывна в $a$
            \item $f\cdot g$ - непрерывна в $a$
            \item $|f|$ непрерывна в $a$
            \item $ \frac{f}{g}$ непрерывна в $a$, если $g(a) \neq 0$
        \end{enumerate}
        \begin{proof}
            Если $a$ - не предельная точка, там непрерынвы все функции.\\
            Если $a$ - предельная точка, то $\lim\limits_{x \to a} f(x) = f(a)$, $\lim\limits_{x \to a} g(x) = g(a)$.
            \[ \lim\limits_{x \to a} f(x)+g(x) = \lim\limits_{x \to a} f(x) + \lim\limits_{x \to a} g(x) = f(a) + g(a) .\]
            Другие пункты аналогично.
        \end{proof}
    \end{theorem}
    \begin{tlemma}
        Многочлены непрерывным во всех точках.
        \begin{proof}
            $C_c(x) = c$ - непрерывна.\\
            $f(x) = x$ - непрерывна.\\
            $F_a(x) = x^{a} = \underbrace{x\cdot x\cdot \ldots\cdot x}_{a} $ - непрерывна.\\
            $P(x) = C_{a_1}(x) + C_{a_1}F_{a}(x) + \ldots C_{a_n}F_n(x)$ - непрервына.
        \end{proof}
    \end{tlemma}
    \begin{tlemma}
        Рациональные функции непрерывны во всех точках, где знаменатель не обращается в $0$.
        \begin{proof}
            $P_1(x)$, $ P_2(x)$ - непрерывны.\\
            $\frac{P_1(x)}{P_2(x)}$ - непрерывна, если $P_2(a) \neq 0$
        \end{proof}
    \end{tlemma}
    \begin{theorem}[Теорема о стабилизации знака]
        $f: E \mapsto \mathbb{R}$, $f$ непрерывна в точке $a$, $a\in E$, $f(a) \neq 0$.
        \[ \exists{U_a}\quad \forall{x\in U_a\cap E}\quad \sgn(f(x)) = \sgn((f(y))   .\]
        \begin{proof}
            Если $a$ не предельная точка, то утверждение эквивалентно $\sgn(f(a)) = \sgn(f(a))$.\\
            Если $a$ предельная точка, то следует из стабилизации знака для передлов.
        \end{proof}
    \end{theorem}
    \begin{theorem}[Теорема о непрервыности композиции]
        $f: E \mapsto G \subset \mathbb{R}$, $g: G \mapsto \mathbb{R}$. $f$ непрерывна в $a$, $g$ непрерывна в $b=f(a)$.\\
        $h(x) = g(f(x))$ непрерывна в $a$.
        \begin{proof}
            \[ \forall{\varepsilon>0}\quad \exists{\delta>0}\quad \forall{y\in G}\quad |y-b|<\delta \implies |g(y)-g(b)| < \varepsilon    .\]
            \[ \exists{\gamma > 0}\quad \forall{x\in E}\quad |x-a|<\gamma \implies |f(x) < f(a)| < \delta  .\]
            \[ y=f(x) .\]
            \[ |x-a|<\gamma \implies |y-b|<\delta \implies |g(f(x))-g(f(a))|<\varepsilon .\qedhere\] 
        \end{proof}
    \end{theorem}
    \begin{tlemma}
        $\lim\limits_{x \to a} f(x) = A$, $g$ непрерывна в точке $A$.\\
        Тогда $\lim\limits_{x \to a} g(f(x)) = g(A)$.\\
        \begin{proof}
            Подправим функцию $f$ в точке $a$, так, чтобы $f(a) = A$. Пределы от этого не изменились.\\
            Теперь $f$ непрерывна в $a$, значит $g(f(x))$ непрерывна в $a$.\\
            Значит, $\lim\limits_{x \to a} g(f(x)) = g(f(a)) = g(A)$.
        \end{proof}
    \end{tlemma}
    \begin{theorem}
        \[ \forall{x\in \left( 0; \frac{\pi}{2} \right) }\quad \sin x < x < \tan x .\] 
        \begin{proof}
            Рассмотрим рисунок:\\
           \begin{center}
               \begin{tikzpicture}[scale=1.5, every node/.style={transform shape}]

                   \def\rad{1};
                   \coordinate (C) at (0,0);
                   \coordinate (shift) at (45:\rad);
                   \draw[thin] (C) circle[radius=\rad];
                   \draw[thin] (0, 0) node[below left]{$O$} -- (1, 0) node[right]{$A$} -- (shift) node[above left]{$B$} -- (0,0);
                   \draw[thin] (0, 0) -- (1, 1) node[above]{$C$} -- (1,0);
                   \draw[thin] let \p1=(shift) in (\x1, \y1) -- (\x1, 0) node[below]{$H$};
               \end{tikzpicture}
           \end{center}
           \[ \sin x = BH .\]
           \[ \tg x = AC .\]
           \[ S_{\triangle OBA} < S_{OBA} < S_{\triangle OCA} .\]
           \[ S_{\triangle OBA} = \frac{1}{2}OA\cdot BH = \frac{\sin x}{2} .\]
           \[ S_{OAB} = \frac{1}{2}xr^2 = \frac{x}{2} .\]
           \[ S_{\triangle OCA} = \frac{1}{2} OA\cdot AC = \frac{\tan x}{2} .\]
           \[\sin x < x < \tan x .\qedhere\] 
        \end{proof}
    \end{theorem}
    \begin{tlemma}
        $|\sin x| \le |x|$. Равно только при $x=0$
    \end{tlemma}
    \begin{tlemma}
        $|\sin x - \sin y| \le |x-y|$\\
        $|\cos x - \cos y| \le |x-y|$.
    \end{tlemma}
    \begin{tlemma}
        $\sin x$ и $\cos x$ всюду непрерывны.\\
        $\tan x$ и $\ctg x$ непрерывны во всех точках определения.
    \end{tlemma}
    \begin{theorem}[Теорема Вейерштрасса]
        $f: \left[a; b\right] \mapsto \mathbb{R}$, непрерывна на всей области определения.\\
        Тогда:
        \begin{enumerate}
            \item $f$ ограничена
            \item $f$ принимает наибольшее и наименьшее значение.
        \end{enumerate}
        \begin{proof}
           Докажем от противного.\\
           Пусть $f$ неограничена.\\
           \[ \forall{n}\quad \exists{x_n}\quad |f(x_n)| > n .\]
           $x_n$ - ограниченная последовательность, существует $x _{n_k}$, $\lim x_{n_k} = c$.\\
           \[ a \le x_n \le b \implies a \le c \le b .\]
           Тогда $f$ непрерывна в точке $c$.\\
           Значит, $\lim f(x_{n_k}) = f(c) $, значит, $f(x_{n_k})$ ограничена.\\
           Но $|f(x_{n_k})| \ge n_k \ge k$, значит, $\lim |f(x_{n_k})| = +\infty$. Противореие, значит функция ограничена.\\
           Пусть $M = \sup f(x) < +\infty$.\\
           Предположим что $M$ не достигается.\\
           \[ \forall{x\in \left[a, b\right]}\quad f(x) < M .\]
           Пусть $g(x) = \frac{1}{M-f(x)} > 0$, $g$ непрерывна на всей области определения.\\
           Значит, она ограниченна. $g(x) \le M_1$:
           \[ \frac{1}{M-f(x)} \le M_1 \implies M - f(x) \ge \frac{1}{M_1} \implies f(x) \le M - \frac{1}{M_1} .\]
           Но $M$ - супремум. Так-что пртиворечие. В другую сторону аналогично.
        \end{proof}
    \end{theorem}
    \begin{theorem}[Теорема Больцмана-Коши]
        $f: \left[a; b\right] \mapsto \mathbb{R}$, непрерывна на всей области определения.\\
        Если $C\in \left[f(a), f(b)\right]$, то $\exists{c\in \left[a; b\right]}\quad f(c)=C$.
        \begin{proof}
            Пусть, без ограничения общности $C=0$, $f(a)< 0 < f(b)$.\\
            Пусть $a_0 = a$, $b_0 = b$.\\
            Рассмотрим функцию в точке $m=\frac{a_0+b_0}{2}$.\\
            Если $f(m) = 0$, то точка найдена.\\
            Если $f(m) > 0$, то $a_1=a_0$, $b_1=m$.\\
            Если $f(m) < 0$, то $a_2=m$, $b_1=b$.\\
            Повторяя итерацию, сойдёмся до нужной точки, так-как отрезки стягивающие.\\
        \end{proof}
    \end{theorem}
    \begin{theorem}
        Непрерывный образ отрезка - отрезок.\\
        Пусть $f: \left[a; b\right] \mapsto \mathbb{R}$, $f$ непрерывна.\\
        Тогда $f(\left[a;b\right]) = \left[\min f(x); \max f(y)\right]$.
        \begin{proof}
            По теореме Вейершатрасса, мнимум и максимум достигаются.\\
            Пусть $p = \argmin f(x)$, $q = \argmax f(x)$.\\
            Путь $p\le q$
            По теореме Боinцмана-Коши:\\
            \[ \forall{C\in \left[f(p); f(q)\right]}\quad \exists{c\in \left[p; q\right]}\quad f(c) = C   .\] 
        \end{proof}
    \end{theorem}
    \begin{theorem}]
        Непрерывный образ промежутка - промежуток.\\
        Пусть $ \left<a; b\right>$ - одно из $\left[a; b\right]$, $(a;b]$, $ \left( a; b \right) $, $[a; b)$.
        $f: \left<a, b\right> \mapsto \mathbb{R}$, $f$ непрерывна на $\left<a, b\right>$
        \begin{proof}
            Пусть $m =\inf f(x)$, $M = \sup f(x)$.\\
            \[ f(\left<a, b\right>) \subset \left[m; M\right] .\] 
            Докажем, что:
            \[ \left( m; M \right) \subset f(\left<a, b\right>) .\]
            Возьмём $C\in \left( m; M \right) $\\
            $C$ - не нижняя грань.\\
            Значит
            \[ \exists{p\in \left<a, b\right>}\quad f(p) < C  .\]
            $C$ - не верхняя грань.\\
            Значит
            \[ \exists{q\in \left<a, b\right>}\quad f(q) > C  .\]
            \[ f(p) < C < f(q) \implies C\in \left[f(p); f(q)\right] .\] 
        \end{proof}
    \end{theorem}
\end{document} 
