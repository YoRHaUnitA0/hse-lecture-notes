% !TEX encoding = UTF-8 Unicode
\documentclass[11pt, oneside]{article}   	% use "amsart" instead of "article" for AMSLaTeX format
\usepackage{amssymb}
\usepackage{amsmath}
\usepackage{cRussian}
\usepackage{cPicture}
\usepackage{cAlgebra}
%\usepackage{cFonts}
%\usepackage{cTikz}
\title{Мат. Анализ}
\author{Igor Engel}
\date{}

\begin{document}
\maketitle
\section{Введение}
    $x \in A$ - x принадлежит A\\
    $a \not\in A$ - x не принадлежит A\\
    $A \subset B \implies \forall{x \in A}\quad x \in B$\\
    $A = B \equiv A \subseteq B\land B \subseteq A$\\
    $A \subsetneq B \implies (A\subset B)\land A\neq B$ - A - собственное подможество B\\
    $\emptyset$ - пустое множество\\
    $\mathbb{N}$ - натуральные числа\\
    $\mathbb{Z}$ - целые числа\\
    $\mathbb{Q}$ - рациональные числа\\
    $\mathbb{R}$ - вещественные числа\\
    $2^A$ - множество всех подмножеств A\\
    $A = \{1, 2\}$\\
    $2^{A} = \{\{\emptyset\}, \{1\}, \{2\}, \{1, 2\}\}$\\
    $\forall$ - для всех\\
    $\exists$ - существует
    \subsection{Задание множеств}
    \begin{itemize}
        \item Перечисление: $\{a, b, c\}$, $\{a_1, a_2, \ldots, a_n\},  \{a_1, a_2, a_3, \ldots\}  $ 
        \item Описание: множество простых чисел - множество таких чисел, которые делятся только на себя и на единицу
        \item Функция: $\{x \in X\ssep \Phi\left( x \right) \} $ - множество всех $x$ из множества $X$, для которых  $\Phi\left( x \right) $ - истина. Выбирать $\Phi$ аккуратно.
    \end{itemize}
    \subsection{Операции с множествами}
        \[ A\cup B = \{x\ssep x \in A \lor x \in B\}  .\]
        \[ A\cap B = \{x\ssep x \in A\land x \in B\}  .\] 
        \[ A\setminus B = \{x\ssep x \in A\land x \not\in B\}   .\] 
        \[ A\triangle B = \left( A\setminus B  \right)\cup\left( B\setminus A \right)   .\]
        \[ \bigcup\limits_{\alpha \in I}A_{\alpha} \equiv \{x \ssep \exists{\alpha \in I}\quad x \in A_{\alpha_i}\} .\]
        \[ \bigcap\limits_{\alpha \in I}A_{\alpha} \equiv \{x \ssep \forall{\alpha \in I}\quad x \in A_{\alpha}\}  .\] 
        \[ \bigcup\limits_{a=1}^kA_{\alpha} \equiv \bigcup\limits_{\alpha \in \left[1; k\right]}A_{\alpha} .\]
    \subsection{Правила Де'Моргана}
        \[ A\setminus\bigcup\limits_{\alpha \in I}B_{\alpha} = \bigcap\limits_{\alpha \in I}A\setminus B_{\alpha}  .\]
    \subsection{}
        \begin{theorem}
            \[ A\cap\left( \bigcup\limits_{\alpha \in I}B_{\alpha} \right)=\bigcup\limits_{\alpha \in I}\left( A\cup B_{\alpha} \right)   .\] 
            \[ A\cup\left( \bigcap\limits_{\alpha \in I}B_{\alpha} \right) = \bigcap\limits_{\alpha \in I}\left( A\cap B_{\alpha} \right)   .\] 
            \begin{proof}
                \[x \in A\cap\left( \bigcup\limits_{\alpha \in I}B_{\alpha} \right) \implies \exists{\alpha \in I}\quad x \in A\land x \in B_{\alpha} \implies \exists{\alpha \in I}\quad x \in \left( A\cup B_{\alpha} \right) \implies x \in \bigcup\limits_{\alpha \in I}\left( A\cup B \right) .\]
                \[ A\cap\left( \bigcup\limits_{\alpha \in I}B_{\alpha} \right) \subset \bigcup\limits_{\alpha \in I}\left( A\cup B \right) .\] 
                Доказательство обратного вхождения, а так-же следующего утверждения симметричны. 
            \end{proof}
        \end{theorem}
    \subsection{Упорядоченная пара}
        \[ \left<x, y\right> = \left<x', y'\right> \equiv x=x'\land y=y' .\]
        \subsubsection{Кортеж (Упорядоченная $n$-ка}
        \[ \left<x_1, x_2, \ldots, x_n\right> = \left<x_1', x_2', \ldots, x_n'\right> \equiv \forall{i \in [1, n]}\quad x_i = x_i' .\] 
    \subsection{Декартово произведение}
        \[ A\times B = \{\left<x, y\right> \ssep x \in A\land y \in B\}  .\] 
    \subsection{Бинарное отношение}
        Б. о. между множествами $A$,  $B$:
         \[ R \subset A\times B .\]
         \[ xRy \equiv \left<x, y\right> \in R .\]
        \subsubsection{Область определения отношения}
        \[ \delta_R \equiv \text{dom}_R .\]
        \[ \delta_R \equiv \{x \in A \ssep \exists{y \in B}\quad xRy\}  .\] 
        \[ \rho_R \equiv \{y \in B \ssep \exists{x \in A}\quad xRy\}  .\] 

        \subsubsection{Обратное отношений}
            \[ R^{-1} \subset B\times A .\]
            \[ R^{-1} = \{\left<y, x\right>\ssep xRy\}  .\]
        \subsubsection{Композиция отношение}
            \[ R_1 \subset A\times B .\]
            \[ R_2 \subset B\times C .\]
            \[ R_1 \circ R_2 = \{\left<x, z\right>\ssep \exists{y \in B}\quad xR_1y\land yR_2z\}  .\] 
        \subsubsection{Примеры}
            \begin{enumerate}
                \item $A = B$ \\
                    Отношение равенства: $R = \{\left<x, x\right>\ssep x \in A\} $ 
                \item $A = B=\mathbb{R}$\\
                    Отношение "$\le$": $R = \{\left<x, y\right>\ssep x \le y\}$ 
                \item $A = B = \mathbb{N}$
                    Отношение "$<$":  $R = \{\left<x, y\right>\ssep x < y\} $ \\
                    $\delta_R = \mathbb{N}$\\
                    $\rho_R = \mathbb{N}\setminus \{1\} $\\
                    $(< \circ <) = \{\left<x, z\right>\ssep x < z-1\} $
                \item Отношение перпендикулярности:\\
                    $\perp \circ \perp \equiv \parallel$
                \item $A$ - множество всех живших мужчин\\
                    $R = \{\left<x, y\right> \ssep x \text{ отец } y\} $ \\
                    $\delta_R =$ все у кого есть отец\\
                    $\rho_R =$ все у кого есть сыновья\\
                    $R \circ R = $ дед по отцовской линии
            \end{enumerate}
    \subsection{Функция}
    $f$ называется функцией и обозначается  $f: A \mapsto B$, если она - отношение между $A$ и  $B$, обладающие следующими свойствами:
         \[ \delta_f = A .\]
         \[ \forall{x \in A}\quad\forall{y, z \in B}\quad xfy\land xfz \implies y=z .\]
         \[ y=f(x) \equiv xfy .\]
         \subsubsection{Последовательность}
         Последовательность: $f : \mathbb{N} \to B$ 

    \subsection{Страшные слова}
        \subsubsection{$A = B$}
            $R$ - рефлексивное если  $\forall{x \in A}\quad xRx$\\ 
            $R$ -  иррефлексивное если $\nexists{x \in A}\quad xRx$\\
            $R$ - симметричное если  $\forall{x, y \in A}\quad xRy \equiv yRx$\\
            $R$ - антисимметричное если  $\forall{x, y \in A}\quad xRy \not\equiv yRx$ \\
            $R$ - транзитивное если $\forall{x, y, z \in A}\quad xRy\land yRz \implies xRz$\\
            $R$ - отношение эквивалентсности если оно симметрично и транзитивное\\
        \subsubsection{Функция}
            $f$ - инъективная, если  $\forall{x, y \in A} f(x) = f(y) \implies x = y$\\ 
            $f$ - сюръективна, если  $\rho_f = B$\\
            $f$ - биективна, если она инъективна и суръективна
        \subsubsection{Примеры}
            \begin{enumerate}
                \item Равенство, сравнение по модулю, паралелльность, подобие треугольников - отношение эквивалентсноти
                \item Рефлексивное + антисимметричное + транзитивное - отношение нестрогого частичного порядка\\
                Например, 
                \[A=\mathbb{R}, R = \le\]
                \[A=2^C,  R = \subset\]
                \item Иррефлексивное + транзитивное: Отношение строгого частичного порядка\\
                    Например: 
                    \[A=\mathbb{R}, R = <\]
                    \[A=2^C, R = \subsetneq\] 
            \end{enumerate} 
    \subsection{Задания}
        \begin{enumerate}
            \item $R$ - строгий частичный порядок. Тогда  $R\cup \{x \in A\ssep\left<x, x\right>\} $ - нестрогий частичные порядок, и наоборот
            \item $f: A \to B$, $f: B \mapsto C$, тогда $(f \circ g) A \mapsto C$, и $(f \circ g) = g(f(x))$
        \end{enumerate}
         
\end{document} 
